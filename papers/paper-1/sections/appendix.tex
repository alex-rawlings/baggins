\section{Prior Predictive Checks for Projected Mass Density}
Here we show the prior predictive check for the projected mass density of the H-1.00 system, as well the corner plots of the latent parameter hyperdistributions. 

\begin{figure*}
    \includegraphics[width=\textwidth]{{graham_density-H-H-3.0-0.001_prior_pred_log10_proj_density_mean}.png}
    \caption{\drafting{Prior predictive estimate for the core-S\'ersic fit. Darker shaded regions correspond to higher Bayesian Credible Intervals (BCIs). The observed data is overlaid. From the prior predictive estimate, it is seen that the model and the assumed parameter hyperdistributions are able to describe the observed data, in that the observed data lies within the range allowed by the prior predictive estimate.}}
    \label{fig:priorpred}
\end{figure*}

\begin{figure*}
    \includegraphics[width=0.7\textwidth]{{graham_density-H-H-3.0-0.001_pair_0}.png}
    \caption{\drafting{Corner plot of hierarchical model hyperparameters. Each of the hyperdistributions are modelled as Gamma distributions with shape parameters $\alpha$ and $\beta$. For example, the hyperdistribution on the core radius $r_\mathrm{b}$ is modelled as $r_\mathrm{b} \sim \text{Gamma}(r_{\mathrm{b},\alpha}, r_{\mathrm{b},\beta})$. Critically, the hyperdistributions are unimodal with highest-density regions located some distance from the distribution edges, indicating the HMC sampling has converged well. The contours correspond to the 50\%, 90\%, 95\%, and 99\% BCIs. }}
    \label{fig:corner_params_1}
\end{figure*}

\begin{figure*}
    \includegraphics[width=0.7\textwidth]{{graham_density-H-H-3.0-0.001_pair_1}.png}
    \caption{\drafting{Same as \autoref{fig:corner_params_1}}}
    \label{fig:corner_params_2}
\end{figure*}

\begin{figure*}
    \includegraphics[width=0.7\textwidth]{{graham_density-H-H-3.0-0.001_pair_2}.png}
    \caption{\drafting{Same as \autoref{fig:corner_params_1}}}
    \label{fig:corner_params_3}
\end{figure*}