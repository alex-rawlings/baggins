The second method is a forward modelling method, where we use Bayesian analysis to construct a simple hierarchical model. 
We construct the hierarchical model using \textsc{Stan}, which is an implementation of Hamiltonian Monte Carlo.
Our likelihood function is constructed using the analytical hardening relations \citep{quinlan96,sesana06}:
\begin{equation}\label{eq:quinlan_H}
    \dv{}{t}\left( \frac{1}{a} \right) = H \frac{G\rho}{\sigma},
\end{equation}
and
\begin{equation}\label{eq:quinlan_K}
    \dv{e}{t} = -K a^{-1} \dv{a}{t}.
\end{equation}
In these equations, $H$ and $K$ are dimensionless constants describing the evolution rate of the semimajor axis and eccentricity of the SMBH binary, respectively, and $\rho$ and $\sigma$ are the stellar density and velocity dispersion within the binary influence radius, respectively.
We wish to determine the uncertainty on the free parameters $H' = HG\rho/\sigma$ and $K$, in addition to the initial conditions for the differential equation system, $a_\mathrm{h}$ and $e_\mathrm{h}$, at a time $t_\mathrm{h}$. 
Let us denote the set of parameters $[H', K, a_\mathrm{h}, e_\mathrm{h}]$ as $\theta$.
To determine the uncertainty in the parameter set $\theta$, we solve the differential equation system analytically for a time period after the SMBH binary becomes hard where the effect of GW emission can be ignored: here taken to be $t_\mathrm{end} = t(a=10\,\mathrm{pc})$. 
Defining a time translation $t' = t - t_\mathrm{h}$, \autoref{eq:quinlan_H} becomes:
\begin{equation}\label{eq:quinlan_H2}
    \frac{1}{a(t')} \simeq H't' + \frac{1}{a_\mathrm{h}},
\end{equation}
and \autoref{eq:quinlan_K} becomes:
\begin{equation}\label{eq:quinlan_K2}
    e(t') \simeq K \ln\left[ \frac{a_\mathrm{h}}{a(t')} \right] + e_\mathrm{h}.
\end{equation}
The parallel to the first method now becomes apparent: in a given set of simulations, each simulation will have its own value of $\theta$, however each realisation of $\theta$ is itself a draw from a common population distribution that we wish to understand.
If we assume that $e_\mathrm{h}$ is normally distributed with some mean $\mu_{e_\mathrm{h}}$ and standard deviation $\sigma_{e_\mathrm{h}}$, then the distribution of $\sigma_{e_\mathrm{h}}$ is precisely the uncertainty in the standard deviation of the binary eccentricity at the time the binary becomes hard.
The primary benefit to using this second, forward modelling approach, is that we obtain an uncertainty estimate on $\sigma_{e_\mathrm{h}}$, which is not obtainable when using the first, inverse modelling approach. 

A probabilistic graphical model is shown in \drafting{FIGURE}.