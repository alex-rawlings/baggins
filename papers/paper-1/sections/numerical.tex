\section{Numerical Simulations}

\subsection{Initial Conditions}

\begin{table*}
    \centering
    \caption{Parameters of Hernquist sphere galaxy models used in the sample.}
    \label{tab:hernquist_pars}
    \begin{tabular}{ccccccc}
        \hline
        Name & $m_\star / (10^5 \Msun)$ & $N_\star/10^6$ & $\varepsilon_\star$/pc & $m_\mathrm{DM} / (10^7 \Msun)$ & $N_\mathrm{DM}/10^6$ & $\varepsilon_\mathrm{DM}$/pc\\
        \hline
        H-1.00 & 1.0 & 1.00 & 3.5 & 3.0 & 2.54 & 200.0 \\
        H-0.50 & 2.0 & 0.50 & 3.5 & 6.0 & 1.27 & 200.0 \\
        H-0.25 & 4.0 & 0.25 & 3.5 & 12.0 & 0.64 & 200.0 \\
        H-0.10 & 10.0 & 0.10 & 7.0 & 30.0 & 0.25 & 200.0 \\
        H-0.05 & 20.0 & 0.05 & 7.0 & 60.0 & 0.13 & 200.0 \\
        \hline
    \end{tabular}
\end{table*}

We simulate the merger of two identical galaxies. 
Each galaxy is modelled as a spherically-symmetric multicomponent system: a stellar component embedded within a DM halo, and a single SMBH at the centre of the galaxy. 
The total stellar mass is set to $M_\star=10^{11}\,\Msun$, the total DM mass is set to $M_\mathrm{DM} \sim 8\times 10^{12}\,\Msun$, and the SMBH mass is set to $M_\bullet \sim 8.8\times10^8\,\Msun$.
Five realisations of the system are generated with varying mass resolutions between the stellar (and DM) particles and the SMBH particle. 
Both the stellar and DM components have a density profile given by an ergodic Hernquist sphere parametrised by scale radius $a$:
\begin{equation}\label{eq:hernquist}
    \rho(r) = \frac{2M}{4\pi} \frac{a}{r(r+a)^3}.
\end{equation}
We use a value of $a_\star=4.0\,\mathrm{kpc}$ for the stellar component, and a value of $a_\mathrm{DM}=300.0\,\mathrm{kpc}$ for the DM component. 
We Monte Carlo sample the phase space distribution function
\begin{equation}\label{eq:eddington}
    f(\mathcal{E}) = \frac{1}{\sqrt{8}\pi^2} \left[ \int_0^\mathcal{E} \dv{\Psi}{\sqrt{\mathcal{E}-\Psi}} \dv{^2\nu_\mathcal{E}}{\Psi^2} + \frac{1}{\sqrt{\mathcal{E}}} \left(\dv{\nu_\mathcal{E}}{\Psi}\right)_{\Psi=0}\right],
\end{equation}
to generate position and velocity coordinates for all stellar and DM particles.
As mentioned, the SMBH is initialised at rest at the origin.
Additionally, we give the SMBH a randomly-drawn value of the dimensionless spin parameter $\chi$ following \drafting{one of the Zlochower papers}, with shape parameters $\alpha=10.5868, \beta=4.66884$. 
This distribution has a mode at $\chi \sim 0.8$. 

We simulate the equal-mass merger between like-mass resolution systems for each mass resolution listed in \autoref{tab:hernquist_pars}.
The progenitor galaxies in each merger are set to an initial separation of $3.0\Rvir$ on quasi-Keplerian orbit, assuming each galaxy is a point-mass containing half its total mass. 
The initial orbital eccentricity is chosen to be almost radial ($e_0 = 0.9999$), and the first pericentre distance $r_\mathrm{peri}$ is set to $10^{-3}\Rvir$. 

\subsection{Simulations}
In typical N-body simulations, one tests the sensitivity of the model by Monte Carlo sampling the phase space distribution function given by \autoref{eq:eddington} a number of times with different random seeds.
This is done to quantify the uncertainty in the model due to the discrete sampling of the phase space of field particles, which induce a Brownian motion to the orbit of the more massive SMBH particle. 
We observe that resampling \autoref{eq:eddington} is in effect displacing the centre of mass (CoM) of the system with the respect to the SMBH.
To avoid the need to run a large number of simulations from the large $3.0\Rvir$ initial separation, we instead \textit{displace the SMBH} with respect to the CoM of the system at a time $t_\mathrm{perturb}$ after the simulation has begun. 
This is done by measuring the Brownian-motion induced displacement in position and velocity space of the SMBH from the CoM for a galaxy in isolation.
After identifying the displacement along each axis to be Gaussian-distributed, we use the variance of this distribution to reposition the SMBH relative to the CoM. 
We thus `recentre' the SMBH on the CoM with a perturbation of $\mathcal{N}_\text{pos}(\mu=0\,\text{pc}, \sigma^2=100\,\text{pc}^2)$ for each position axis, and $\mathcal{N}_\text{vel}(\mu=0\,\text{km\,s}^{-1}, \sigma^2=100\,\text{km}^2\text{\,s}^{-2})$ for each velocity axis. 
In this way, we mimic the effect of Brownian motion for a single run beginning at large separation (the \parent{} run) to generate an arbitrary number of perturbed \child{} runs. 
The perturbation is applied no later than the second-to-last pericentre passage before the SMBH particles form a bound binary system, and ensure the separation of the SMBHs at the time of the perturbation is of the order 1000 larger than the perturbation applied. 

The progenitors are evolved through the \textit{parent phase} from $t(r_0)=0$ with \gadget{} until the SMBHs form a bound binary at $t_\text{stop}$, with a stellar softening of $\epsilon_\star=10\,\text{pc}$ and \gadget integration tolerances \texttt{ErrTolIntAccuracy}=$0.02$ and \texttt{ErrTolForceAcc}=$0.005$.
We then initialise the \textit{child phase} by restarting from the snapshot closest to but earlier than $t=t_\mathrm{perturb}$.
We apply the small perturbation to each SMBH position and velocity vector in the simulation, from which we create ten uniquely-perturbed \child{} runs, allowing us to explore the distribution of parameters describing both the orbital and remnant properties of each \parent{} merger. 
Additionally, we lower the stellar softening to the values presented in \autoref{tab:hernquist_pars} and tighten the integration tolerance \texttt{ErrTolIntAccuracy} to $0.002$ to optimise the run for \ketju{}.
We check for energy conservation between the \parent{} and \child{} runs, and confirm that the total energy is conserved at the sub-percent level. 
In total, we perform 50 unique \child{} simulations using \ketju{}, and use this sample for our analysis. 

Within a relatively short time frame from beginning the \child{} run, the SMBH binary hardens predominantly through dynamical friction and three-body stellar interactions. 
We evolve the binary to the time $t_{a_h}=t(a_h)$, and continue to when $t_\text{safe}=t(a=10\,\text{pc})$. 
We then analytically predict the orbital evolution of the binary by determining the hardening constant $H$ in \autoref{eq:quinlan_H} and eccentricity constant $K$ in \autoref{eq:quinlan_K} by fitting the derivative of $\text{d}(1/a)/\text{d}t$ with a linear function between $t_{a_h}$ and $t_\text{safe}$.
Using $a_0=a_h$ and three different values of $e_0$, taken to be the 5\%, 50\%, and 95\% quantiles of the eccentricity distribution within the same time interval 
$t_{a_h} \text{--} t_\text{safe}$, we predict the expected merger timescale of the system. 
We continue to run the \child runs until one of three conditions are met:
\begin{enumerate}
    \item The SMBHs merge, or
    \item The Hubble time, $13.8\,\text{Gyr}$, is exceeded, or
    \item The eccentricity stabilises to below the analytically-determined eccentricity estimated from the 5\%-quantile eccentricity value, and said analytical-estimate indicates that the merger will occur significantly after the Hubble time.
\end{enumerate}
If it is not clear if the third condition is satisfied, we continue to run the merger until the second condition is met.

