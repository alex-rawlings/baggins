\section{Results}\label{sec:results}

\drafting{
Figures to do
\begin{itemize}
\item Example of forward-modelled semimajor axis and eccentricity
\end{itemize}
and maybe another one of BH merger remnant properties.
}

\drafting{
    WHAT ARE THE MAIN OBSERVATIONS WE SEE and WHAT CONCLUSIONS CAN WE DRAW?
    \begin{enumerate}
        \item compare to \citet{nasim20}
        \item stability of hardening rate -> semimajor axis can be robustly predicted
        \item wide variance of eccentricity -> N-body models offer no quantitative prediction on likelihood of merger, can only comment on what merger looks like \textit{IF} it happens
        \item sample posterior to determine merger timescale
    \end{enumerate}
}


\subsection{Comparison to Previous Work}
\begin{enumerate}
    \item \citet{nasim20} show convergence of eccentricity to follow $\sigma_\mathrm{ecc} \propto 1/\sqrt{N_\star}$
    \item Perform same analysis
    \item Results disagree $\rightarrow$ because of highly radial orbit?? 
\end{enumerate}


\subsection{Hardening Rate}
\begin{enumerate}
    \item Introduce HM for hardening rate
    \item Corner plots of latent parameters, forward folded PPC
    \item Discussion uncertainty on $HG\rho/\sigma$
\end{enumerate}

We enforce a minimum of six simulations per mass resolution for that mass resolution to be analysed with our hierarchical modelling technique. 
For these five positive-constrained latent parameters, we model the latent distribution as a half normal distribution. 
Each location hyperparameter for each of these latent parameters is also drawn from a truncated normal distribution, as a negative expectation value may be excluded on physical grounds.
For $\log_{10}\Sigma_\mathrm{b}$, we assume a normal distribution, with a location hyperparameter centred at $\log_{10}\Sigma_{\mathrm{b}, \mu}=10$. 
For all scale hyperparameters, we use a half normal distribution (as the scale of a distribution must of course always be positive).
To ensure our hyperparameter prior distributions are weakly-informative, the hyperparameters are drawn from the distributions in \autoref{tab:cshypers} for all simulation families. 
The sampled hyperparameter distributions are shown as corner plots in Figures

We ensure our hyperparameter and latent parameter distributions are representative of our observed data by performing prior predictive checks, an example of which is shown for the H-1.00 family in \autoref{fig:priorpred}. 
Critically, both the range of values spanned by the prior distributions and the general shape of the distribution suggests that the data is well represented by the model and the prior distributions on the hyper and latent parameters. \drafting{Not yet, but will be. Some issue with the sampler.} 

We compute the posterior distribution using \textsc{Stan}, ensuring that the sampler exits without encountering any divergences, and that all \drafting{R-hat (is a summary stat, but how to draw? Don't want to confuse with observed radius)} values are within the range \drafting{$[0.99, 1.01]$}.
We first inspect how the posterior distribution appears visually by computing posterior predictive checks, an example of which is shown in \autoref{fig:postpred} for the family H-1.00. 
We find that the data is well described by the 50\% HDI \drafting{introduce what HDI is} of the posterior distribution, as desired. 

As part of the hierarchical modelling sampling process, we can obtain in addition to the posterior distribution on $\Sigma(R)$ the latent parameter distributions from which $\Sigma(R)$ is determined by marginalising out the other parameters.

After considering the robustness of the projected density profile to SMBH Brownian motion, we turn our attention to a similar analysis of the hardening rate of the SMBH binary. 
The hardening rate $H$ in \autoref{eq:quinlan_H} connects the evolution of the inverse semimajor axis of the binary system to the surrounding stellar mass distribution. 
To extract the data for our hierarchical model for hardening rate, we take for each simulation the semimajor axis values between $a=a_\mathrm{h}$ and $a/\mathrm{pc}=10$.
We then fit the differential equation give in \autoref{eq:quinlan_H} to obtain a parameter estimate of the combined quantity $HG\rho/\sigma$.
We fit for the combined property in preference to just $H$ as the temporal-resolution of the semimajor axis $a$ is \drafting{check} $10^4$ times more frequent than the temporal resolution of the snapshots, requiring linear (or higher order) interpolation to be done between snapshots to obtain a continuous function of $G\rho/\sigma$ that can be mapped to the temporal resolution of $a$. 
To avoid these assumptions, we estimate the combined quantity $HG\rho/\sigma$.

\begin{figure}
    \includegraphics[width=0.5\textwidth]{quinlan_posterior_pred_inv_a.png}
    \caption{\drafting{Example plot. This is an old one and needs to be updated, but gives us the idea of what we can talk about}}
    \label{fig:quinlan_H}
\end{figure}


\subsection{Binary Properties}
\begin{enumerate}
    \item introduce HM for binary quantities 
    \item corner plots $a$, $e$, discuss generacy
    \item forward folded PPC for $l$
    \item demonstrate $e$ cannot be robustly predicted (e.g. multimodal marginal distribution)
\end{enumerate}

\begin{figure}
    \includegraphics{compare_binaries.pdf}
    \caption{\drafting{Show an example of the scatter we get from perturbations. These are plotted with t=0 corresponding to when the child run started. Best way to show this, or should a time-shift be done? This way at least we see clearly that different sims become bound at different times.}}
    \label{fig:ketju_scatter}
\end{figure}
\drafting{INtro here} orbital parameters of the SMBH binary, namely the semimajor axis $a$ and the eccentricity $e$. 
It is instructive to first consider the raw data of a particular simulation family, shown in \autoref{fig:ketju_scatter} for the case of H-1.00. 
Immediately apparent is the scatter in eccentricity between different realisations, with some runs (e.g. 006) having $e<0.2$ for the majority of the integration time, whilst other realisations (e.g. 002) have $e>0.9$ for the same time period.
As discussed in \drafting{section}, the efficiency of GW as a hardening mechanism is directly related to the binary eccentricity, and is highly non-linear. 

As seen in \autoref{fig:ketju_scatter}, a higher eccentricity leads to more rapid BH binary mergers as opposed to a lower eccentricity. 
Out of the 50 simulations that we ran, we observe \drafting{XXX} systems that resulted in SMBH binary coalescence. 

Also of note is the rapid oscillation in eccentricity between $\sim100$--$300\,\mathrm{Myr}$. 
This is contrasted by a relatively smooth evolution in semimajor axis, indicating that the SMBH binary orbit is highly sensitive to the surrounding stellar mass distribution.


\subsection{Merger Timescale}
\begin{enumerate}
    \item Sample from marginal distributions $HG\rho/\sigma$, $a_\mathrm{h}$, $e_\mathrm{h}$
    \item Plug in to Quinlan-Peter analytical formula (write this explicitly)
    \item Posterior estimate on $t_\mathrm{merge}$
    \item discuss uncertainty
\end{enumerate}