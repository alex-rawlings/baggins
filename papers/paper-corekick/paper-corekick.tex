% mnras_template.tex 
%
% LaTeX template for creating an MNRAS paper
%
% v3.0 released 14 May 2015
% (version numbers match those of mnras.cls)
%
% Copyright (C) Royal Astronomical Society 2015
% Authors:
% Keith T. Smith (Royal Astronomical Society)

% Change log
%
% v3.0 May 2015
%    Renamed to match the new package name
%    Version number matches mnras.cls
%    A few minor tweaks to wording
% v1.0 September 2013
%    Beta testing only - never publicly released
%    First version: a simple (ish) template for creating an MNRAS paper

%%%%%%%%%%%%%%%%%%%%%%%%%%%%%%%%%%%%%%%%%%%%%%%%%%
% Basic setup. Most papers should leave these options alone.
\documentclass[fleqn,usenatbib]{mnras}

% MNRAS is set in Times font. If you don't have this installed (most LaTeX
% installations will be fine) or prefer the old Computer Modern fonts, comment
% out the following line
\usepackage{newtxtext,newtxmath}
% Depending on your LaTeX fonts installation, you might get better results with one of these:
%\usepackage{mathptmx}
%\usepackage{txfonts}

% Use vector fonts, so it zooms properly in on-screen viewing software
% Don't change these lines unless you know what you are doing
\usepackage[T1]{fontenc}

% Allow "Thomas van Noord" and "Simon de Laguarde" and alike to be sorted by "N" and "L" etc. in the bibliography.
% Write the name in the bibliography as "\VAN{Noord}{Van}{van} Noord, Thomas"
\DeclareRobustCommand{\VAN}[3]{#2}
\let\VANthebibliography\thebibliography
\def\thebibliography{\DeclareRobustCommand{\VAN}[3]{##3}\VANthebibliography}


%%%%% AUTHORS - PLACE YOUR OWN PACKAGES HERE %%%%%

% Only include extra packages if you really need them. Common packages are:
\usepackage{graphicx}	% Including figure files
\usepackage{amsmath}	% Advanced maths commands
%\usepackage{amssymb}	% Extra maths symbols
\usepackage{bm}         % Bold math fonts

%%%%%%%%%%%%%%%%%%%%%%%%%%%%%%%%%%%%%%%%%%%%%%%%%%

%%%%% AUTHORS - PLACE YOUR OWN COMMANDS HERE %%%%%
% drafting macros
\usepackage{xcolor}
\newcommand{\drafting}[1]{
{\leavevmode\color[RGB]{224, 77, 24}#1}
}

% Please keep new commands to a minimum, and use \newcommand not \def to avoid
% overwriting existing commands. Example:
%\newcommand{\pcm}{\,cm$^{-2}$}	% per cm-squared
\newcommand{\ketju}{\textsc{Ketju}}
\newcommand{\mstar}{\textsc{mstar}}
\newcommand{\gadget}{\textsc{gadget-4}}

\newcommand{\Msun}{\ensuremath{\mathrm{M}_{\sun}}}
\newcommand{\kmps}{\ensuremath{\mathrm{km}\,\mathrm{s}^{-1}}}

\newcommand{\dd}[1]{\ensuremath{\mathrm{d}#1}}
\newcommand{\dv}[2]{\ensuremath{\frac{\dd{#1}}{\dd{#2}}}}
\newcommand{\vb}[1]{\ensuremath{\bm{#1}}} %vectors

\graphicspath{{figures/}}

%%%%%%%%%%%%%%%%%%%%%%%%%%%%%%%%%%%%%%%%%%%%%%%%%%

%%%%%%%%%%%%%%%%%%% TITLE PAGE %%%%%%%%%%%%%%%%%%%

% Title of the paper, and the short title which is used in the headers.
% Keep the title short and informative.
\title[Cores from kicks]{Cores from kicks: constraints for observations of gravitational recoil in elliptical galaxies}

% The list of authors, and the short list which is used in the headers.
% If you need two or more lines of authors, add an extra line using \newauthor
\author[A. Rawlings et al.]{
Alexander Rawlings,$^{1}$\thanks{E-mail: alexander.rawlings@helsinki.fi}
et al.
\vspace*{0.1cm}\\%
% List of institutions
$^{1}$
Department of Physics,
Gustaf H\"allstr\"omin katu 2, FI-00014, University of Helsinki, Finland
\\%
$^{2}$
Max-Planck-Institut f\"ur Astrophysik, Karl-Schwarzchild-Str 1, D-85748 Garching, Germany
}

% These dates will be filled out by the publisher
\date{Accepted XXX. Received YYY; in original form ZZZ}

% Enter the current year, for the copyright statements etc.
\pubyear{2023}

% Don't change these lines
\begin{document}
\label{firstpage}
\pagerange{\pageref{firstpage}--\pageref{lastpage}}
\maketitle

% Abstract of the paper
\begin{abstract}
\drafting{Investigate the relationship between core size and kick velocity}
\end{abstract}

% Select between one and six entries from the list of approved keywords.
% Don't make up new ones.
\begin{keywords}
black hole physics -- galaxies: kinematics and dynamics -- methods: numerical -- software: simulations
\end{keywords}

%%%%%%%%%%%%%%%%%%%%%%%%%%%%%%%%%%%%%%%%%%%%%%%%%%

%%%%%%%%%%%%%%%%% BODY OF PAPER %%%%%%%%%%%%%%%%%%


% INTRODUCTION
\section{Introduction}
\drafting{Some general intro here}

% NUMERICAL
\section{Numerical Simulations}
\label{sec:num_sims}

\drafting{Taken from eccentricity paper, leave here as a guide for a similar description about Ketju}
We construct a number of idealised galaxy merger simulations, which we evolve with our new version of the \ketju{} code \citep{mannerkoski2023,rantala2017}.
The dynamics of
SMBHs, and stars in a small region around them, are integrated with an algorithmically regularised integrator \citep{rantala2020}, whereas the dynamics of the remaining particles
is computed with the \gadget{} \citep{springel2021} fast multiple method (FMM) with second order multipoles. Together with hierarchical time integration this allows for
symmetric interactions and manifest momentum conservation. \ketju{} also includes post-Newtonian (PN) correction terms up to order 3.5 between each pair of SMBHs \citep{blanchet2014}. 

\drafting{Rantala 2018 ICs}
We model the merger of two gas-poor elliptical galaxies using collisionless merger simulations.
Our galaxy initial conditions are chosen to match the model IC-3 presented in \citet{rantala2019}. 
The galaxy is thus represented as a multicomponent sphere, including a stellar component of total mass $M_\star\sim 1.38\times10^{11}\,\Msun$ embedded within a DM \drafting{ensure abbreviation stated} component of total mass $M_\mathrm{DM}=2.5\times10^{13}\,\Msun$, and at the centre a SMBH with mass $M_\bullet=2.93\times10^{9}\,\Msun$.
The stellar and DM components each follow a \citet{dehnen1993} profile with shape parameter $\gamma=1$, and scale radius $a_\star=3.9\,\mathrm{kpc}$ and $a_\mathrm{DM}=245\,\mathrm{kpc}$, respectively\footnote{A Dehnen profile with $\gamma=1$ is commonly referred to as a Hernquist profile.}.
The density profile is given by:
\begin{equation}\label{eq:dehnen}
    \rho_\star(r) = \frac{(3-\gamma)M_\star}{4\pi} \frac{a}{r^\gamma (r+a)^{(4-\gamma)}}.
\end{equation}
The mass of a stellar particle is set to $m_\star=5\times10^4\,\Msun$, and the mass of a DM particle is set to $m_\mathrm{DM}=5\times10^6\,\Msun$.

We create six independently Monte Carlo sampled galaxy ICs, and merge each combination of galaxies on a radial orbit.
The merger orbit has an initial separation of $D=30\,\mathrm{kpc}$, a first pericentre distance of $r_\mathrm{peri}=2\,\mathrm{kpc}$, and an initial eccentricity of $e_0=0.97$.
We select a merger combination that results in a rapid coalescence of the BH binary, noting that the merger timescale we observe is driven by stochasticity in the binary eccentricity \citep{nasim2020,rawlings2023}.
We then select the snapshot just prior to merger, and generate 31 `child' simulations, where each child has a unique gravitational recoil kick velocity $v_\mathrm{kick}$ prescribed, ranging from $0\,\kmps$ to $1800\,\kmps$ (inclusive), in $60\,\kmps$ increments.
The upper limit on $v_\mathrm{kick}$ is chosen to match the escape velocity of the centre of the merger remnant.
We additionally run one simulation with a recoil kick above the escape velocity, with $v_\mathrm{kick}=2000\,\kmps$.
Each child simulation is run until the median remnant SMBH velocity relative to the galaxy CoM \drafting{check abbreviation} velocity is less than $10\,\kmps$ for at least $0.1\,\mathrm{Gyr}$.
Interactions between stellar particles are softened with a softening length of $\varepsilon = 2.5\,\mathrm{pc}$, and the \ketju{} region radius is set to $r_{\rm ketju}=3\varepsilon = 7.5\,\mathrm{pc}$.

\drafting{Atte: mergers of unequal mass ratio?}



% RESULTS
\section{Results}\label{sec:results}
\drafting{
    Things to show:
    \begin{enumerate}
        \item DAG for hierarchical model
        \item Density profiles 
        \item Scatter plot $v_k\mathrm{-}r_b$ with best fit function
        \item Relation $v_k\mathrm{-}r_b$ model comparison (elpd?)
        \item MC-sampled distribution of $r_b$
        \item $h_4$ maps
    \end{enumerate}
}


% DISCUSSION
\section{Discussion}
\label{sec:discussion}



% CONCLUSION
\section{Conclusions}
\label{sec:conclusions}

% FINAL BITS
\section*{acknowledgments}
A.R. acknowledges the support by the University of Helsinki Research Foundation.
A.R., acknowledge the support
by the European Research Council via ERC Consolidator Grant KETJU (no. 818930) and the support of the Academy of Finland grant 339127.

The numerical simulations used computational resources provided by
the CSC -- IT centre for Science, Finland.

\section*{Author contributions}
We list here the roles and contributions of the authors according to the Contributor Roles Taxonomy (\href{https://credit.niso.org}{CRediT}). 
\textbf{AR}:

\section*{Software}
\ketju{} \citep{mannerkoski2023,rantala2017},
\gadget{} \citep{springel2021},
NumPy \citep{harris2020},
SciPy \citep{virtanen2020},
Matplotlib \citep{hunter2007},
pygad \citep{rottgers2020}.


%%%%%%%%%%%%%%%%%%%%%%%%%%%%%%%%%%%%%%%%%%%%%%%%%%
\section*{Data Availability}
The data underlying this article will be shared on reasonable request to the corresponding author.




%%%%%%%%%%%%%%%%%%%% REFERENCES %%%%%%%%%%%%%%%%%%

% The best way to enter references is to use BibTeX:

\bibliographystyle{mnras}
\bibliography{ref} % if your bibtex file is called example.bib


% Alternatively you could enter them by hand, like this:
% This method is tedious and prone to error if you have lots of references
%\begin{thebibliography}{99}
%\bibitem[\protect\citeauthoryear{Author}{2012}]{Author2012}
%Author A.~N., 2013, Journal of Improbable Astronomy, 1, 1
%\bibitem[\protect\citeauthoryear{Others}{2013}]{Others2013}
%Others S., 2012, Journal of Interesting Stuff, 17, 198
%\end{thebibliography}

%%%%%%%%%%%%%%%%%%%%%%%%%%%%%%%%%%%%%%%%%%%%%%%%%%

%%%%%%%%%%%%%%%%% APPENDICES %%%%%%%%%%%%%%%%%%%%%

%\appendix

%%%%%%%%%%%%%%%%%%%%%%%%%%%%%%%%%%%%%%%%%%%%%%%%%%


% Don't change these lines
\bsp	% typesetting comment
\label{lastpage}
\end{document}

% End of mnras_template.tex
