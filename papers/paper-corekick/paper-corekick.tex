% mnras_template.tex 
%
% LaTeX template for creating an MNRAS paper
%
% v3.0 released 14 May 2015
% (version numbers match those of mnras.cls)
%
% Copyright (C) Royal Astronomical Society 2015
% Authors:
% Keith T. Smith (Royal Astronomical Society)

% Change log
%
% v3.0 May 2015
%    Renamed to match the new package name
%    Version number matches mnras.cls
%    A few minor tweaks to wording
% v1.0 September 2013
%    Beta testing only - never publicly released
%    First version: a simple (ish) template for creating an MNRAS paper

%%%%%%%%%%%%%%%%%%%%%%%%%%%%%%%%%%%%%%%%%%%%%%%%%%
% Basic setup. Most papers should leave these options alone.
\documentclass[fleqn,usenatbib]{mnras}

% MNRAS is set in Times font. If you don't have this installed (most LaTeX
% installations will be fine) or prefer the old Computer Modern fonts, comment
% out the following line
\usepackage{newtxtext,newtxmath}
% Depending on your LaTeX fonts installation, you might get better results with one of these:
%\usepackage{mathptmx}
%\usepackage{txfonts}

% Use vector fonts, so it zooms properly in on-screen viewing software
% Don't change these lines unless you know what you are doing
\usepackage[T1]{fontenc}

% Allow "Thomas van Noord" and "Simon de Laguarde" and alike to be sorted by "N" and "L" etc. in the bibliography.
% Write the name in the bibliography as "\VAN{Noord}{Van}{van} Noord, Thomas"
\DeclareRobustCommand{\VAN}[3]{#2}
\let\VANthebibliography\thebibliography
\def\thebibliography{\DeclareRobustCommand{\VAN}[3]{##3}\VANthebibliography}


%%%%% AUTHORS - PLACE YOUR OWN PACKAGES HERE %%%%%

% Only include extra packages if you really need them. Common packages are:
\usepackage{graphicx}	% Including figure files
\usepackage{amsmath}	% Advanced maths commands
%\usepackage{amssymb}	% Extra maths symbols
\usepackage{bm}         % Bold math fonts

%%%%%%%%%%%%%%%%%%%%%%%%%%%%%%%%%%%%%%%%%%%%%%%%%%

%%%%% AUTHORS - PLACE YOUR OWN COMMANDS HERE %%%%%
% drafting macros
\usepackage{xcolor}
\newcommand{\drafting}[1]{
{\leavevmode\color[RGB]{224, 77, 24}#1}
}

% Please keep new commands to a minimum, and use \newcommand not \def to avoid
% overwriting existing commands. Example:
%\newcommand{\pcm}{\,cm$^{-2}$}	% per cm-squared
\newcommand{\ketju}{\textsc{Ketju}}                           % code name 
\newcommand{\mstar}{\textsc{mstar}}                           % code name
\newcommand{\gadget}{\textsc{gadget-4}}                       % code name

\newcommand{\Msun}{\ensuremath{\mathrm{M}_{\sun}}}            % Solar Mass
\newcommand{\kmps}{\ensuremath{\mathrm{km}\,\mathrm{s}^{-1}} }% km/s formatting
\newcommand{\Reff}{\ensuremath{R_\mathrm{e}}}                 % effective radius
\newcommand{\rb}{\ensuremath{r_\mathrm{b}}}                   % core radius
\newcommand{\vk}{\ensuremath{v_\mathrm{kick}}}                % kick velocity

\newcommand{\dd}[1]{\ensuremath{\mathrm{d}#1}}                % integral dx
\newcommand{\dv}[2]{\ensuremath{\frac{\dd{#1}}{\dd{#2}}}}     % derivative
\newcommand{\vb}[1]{\ensuremath{\bm{#1}}}                     % vectors

\graphicspath{{figures/}}

%%%%%%%%%%%%%%%%%%%%%%%%%%%%%%%%%%%%%%%%%%%%%%%%%%

%%%%%%%%%%%%%%%%%%% TITLE PAGE %%%%%%%%%%%%%%%%%%%

% Title of the paper, and the short title which is used in the headers.
% Keep the title short and informative.
\title[Cores from kicks]{Cores from kicks: constraints for observations of gravitational recoil in elliptical galaxies}

% The list of authors, and the short list which is used in the headers.
% If you need two or more lines of authors, add an extra line using \newauthor
\author[A. Rawlings et al.]{
Alexander Rawlings,$^{1}$\thanks{E-mail: alexander.rawlings@helsinki.fi}
et al.
\vspace*{0.1cm}\\%
% List of institutions
$^{1}$
Department of Physics,
Gustaf H\"allstr\"omin katu 2, FI-00014, University of Helsinki, Finland
\\%
$^{2}$
Max-Planck-Institut f\"ur Astrophysik, Karl-Schwarzchild-Str 1, D-85748 Garching, Germany
}

% These dates will be filled out by the publisher
\date{Accepted XXX. Received YYY; in original form ZZZ}

% Enter the current year, for the copyright statements etc.
\pubyear{2023}

% Don't change these lines
\begin{document}
\label{firstpage}
\pagerange{\pageref{firstpage}--\pageref{lastpage}}
\maketitle

% Abstract of the paper
\begin{abstract}
\drafting{Investigate the relationship between core size and kick velocity}
\end{abstract}

% Select between one and six entries from the list of approved keywords.
% Don't make up new ones.
\begin{keywords}
black hole physics -- galaxies: kinematics and dynamics -- methods: numerical -- software: simulations
\end{keywords}

%%%%%%%%%%%%%%%%%%%%%%%%%%%%%%%%%%%%%%%%%%%%%%%%%%

%%%%%%%%%%%%%%%%% BODY OF PAPER %%%%%%%%%%%%%%%%%%


% INTRODUCTION
\section{Introduction}
\drafting{Some general intro here}

% NUMERICAL
\section{Numerical Simulations}
\label{sec:num_sims}

\drafting{Taken from eccentricity paper, leave here as a guide for a similar description about Ketju}
We construct a number of idealised galaxy merger simulations, which we evolve with our new version of the \ketju{} code \citep{mannerkoski2023,rantala2017}.
The dynamics of
SMBHs, and stars in a small region around them, are integrated with an algorithmically regularised integrator \citep{rantala2020}, whereas the dynamics of the remaining particles
is computed with the \gadget{} \citep{springel2021} fast multiple method (FMM) with second order multipoles. Together with hierarchical time integration this allows for
symmetric interactions and manifest momentum conservation. \ketju{} also includes post-Newtonian (PN) correction terms up to order 3.5 between each pair of SMBHs \citep{blanchet2014}. 

\drafting{Rantala 2018 ICs}
We model the merger of two gas-poor elliptical galaxies using collisionless merger simulations.
Our galaxy initial conditions are chosen to match the model IC-3 presented in \citet{rantala2019}. 
The galaxy is thus represented as a multicomponent sphere, including a stellar component of total mass $M_\star\sim 1.38\times10^{11}\,\Msun$ embedded within a DM \drafting{ensure abbreviation stated} component of total mass $M_\mathrm{DM}=2.5\times10^{13}\,\Msun$, and at the centre a SMBH with mass $M_\bullet=2.93\times10^{9}\,\Msun$.
The stellar and DM components each follow a \citet{dehnen1993} profile with shape parameter $\gamma=1$, and scale radius $a_\star=3.9\,\mathrm{kpc}$ and $a_\mathrm{DM}=245\,\mathrm{kpc}$, respectively\footnote{A Dehnen profile with $\gamma=1$ is commonly referred to as a Hernquist profile.}.
The density profile is given by:
\begin{equation}\label{eq:dehnen}
    \rho_\star(r) = \frac{(3-\gamma)M_\star}{4\pi} \frac{a}{r^\gamma (r+a)^{(4-\gamma)}}.
\end{equation}
The mass of a stellar particle is set to $m_\star=5\times10^4\,\Msun$, and the mass of a DM particle is set to $m_\mathrm{DM}=5\times10^6\,\Msun$.

We create six independently Monte Carlo sampled galaxy ICs, and merge each combination of galaxies on a radial orbit.
The merger orbit has an initial separation of $D=30\,\mathrm{kpc}$, a first pericentre distance of $r_\mathrm{peri}=2\,\mathrm{kpc}$, and an initial eccentricity of $e_0=0.97$.
We select a merger combination that results in a rapid coalescence of the BH binary, noting that the merger timescale we observe is driven by stochasticity in the binary eccentricity \citep{nasim2020,rawlings2023}.
We then select the snapshot just prior to merger, and generate 31 `child' simulations, where each child has a unique gravitational recoil kick velocity $\vk$ prescribed along the $x$-axis\footnote{As the $x$-axis is in the global coordinate frame, the direction of the kick is essentially random with respect to the angular momentum and inertia tensors. Additionally, the merger remnant displays almost no triaxiality, as seen in the $\vk=0\,\kmps$ case in \autoref{sec:app_triax}.}, ranging from $0\,\kmps$ to $1800\,\kmps$ (inclusive), in $60\,\kmps$ increments.
The upper limit on $\vk$ is chosen to match the escape velocity of the centre of the merger remnant.
We additionally run one simulation with a recoil kick above the escape velocity, with $\vk=2000\,\kmps$.
Each child simulation is run until the median remnant SMBH velocity relative to the galaxy CoM \drafting{check abbreviation} velocity is less than $10\,\kmps$ for at least $0.1\,\mathrm{Gyr}$.
Those simulations where the maximum displacement of the kicked SMBH exceeds $30\,\mathrm{kpc}$ are not included in the analysis, as we expect that observationally a SMBH further than this distance from the centre of its host galaxy would not be identified with that galaxy.
In practice, this limits our sample of analysed simulations to those with $\vk\leq 900\,\kmps$, thus totalling sixteen simulations in our analysis.
Interactions between stellar particles are softened with a softening length of $\varepsilon = 2.5\,\mathrm{pc}$, and the \ketju{} region radius is set to $r_{\rm ketju}=3\varepsilon = 7.5\,\mathrm{pc}$.

\drafting{Atte's runs of mergers of unequal mass ratio?}

\drafting{Kicks randomly orientated?}



% RESULTS
\section{Results}\label{sec:results}
\drafting{
    Things to show:
    \begin{enumerate}
        \item DAG for hierarchical model
        \item Density profiles 
        \item Scatter plot $v_k\mathrm{-}r_b$ with best fit function
        \item Relation $v_k\mathrm{-}r_b$ model comparison (elpd?)
        \item MC-sampled distribution of $r_b$
        \item $h_4$ maps
    \end{enumerate}
}

\drafting{
    Main points to hammer home:
    \begin{enumerate}
        \item Parameters associated with the core (most importantly the core size) have a dependence on the kick velocity, whereas large-scale parameters (e.g. effective radius) do not.
        \item Increasing kick velocity induces a decrease in the overall $\langle h_4 \rangle$ profile
    \end{enumerate}
}


\subsection{Density Profiles}
We fit the six-parameter core-Se\'rsic profile to the projection of each simulated merger remnants:
\begin{equation}\label{eq:cs}
    \Sigma(R) = \Sigma' \left[1 + \left(\frac{\rb}{R}\right)^\alpha \right]^{\gamma/\alpha} \exp\left[-b \left(\frac{R^\alpha + \rb^\alpha}{\Reff^\alpha}\right)^{(1/\alpha n)}\right]
\end{equation}
where
\begin{equation}
    \Sigma' = \Sigma_\mathrm{b} 2^{-\gamma/\alpha} \exp\left[b\left(2^{1/a} \frac{\rb}{\Reff}\right)^{1/n}\right].
\end{equation}
Here $\rb$ is the core radius, $\Sigma_\mathrm{b}$ is the density at the core radius, $\gamma$ is the core slope, $\Reff$ is the effective (half-light) radius, $\alpha$ is the profile transition index, and $n$ is the Se\'rsic index.
We refer to the collective vector of these parameters as $\vb{\theta}$.
We view the simulated merger remnant from fifteen different angles, and use a Bayesian hierarchical model (HM) to fit the model parameters.
As each of the projections are unordered and exchangeable (i.e., there is no distinguishing label for the projections), and each projection is of the same merger remnant, a HM naturally lends itself.
The idea behind the HM is straightforward: for each projection, we infer the six parameters of $\vb{\theta}$, and assume that these values are draws from a common, global distribution of all possible vectors $\vb{\theta}$.
Hence, we are effectively inferring the global hyperparameters $\vb{\theta}^\mathrm{hyp}$ that describe the latent parameters $\vb{\theta}$ that we are interested in.
This idea is demonstrated in the directed acyclic graph (DAG) for the model, shown in \autoref{fig:dag}.

We use weakly-constrained priors for each of the hyperparameters in $\vb{\theta}^\mathrm{hyp}$, where these priors are not necessarily Gaussian (given that distributions of positive-constrained quantities, like radius, should not be described by a distribution with support on the full real space).
The distributions\footnote{Note that the normal distributions $\mathcal{N}$ are parameterised using the standard deviation $\sigma$, as opposed to the variance $\sigma^2$ as is common in many statistics references.} of the hyperparameters are given in \autoref{tab:hyper}.
We use the Hamiltonian Monte Carlo code \textsc{Stan} \citep{standevelopmentteam2018} with four chains each of 2000 posterior sample draws (excluding 1000 burn-in draws) to fit the model parameters, ensuring that:
\begin{enumerate}
    \item the number of diverging draws is less than \drafting{5\%},
    \item the effective draws per transition exceeds 0.001,
    \item the sampler transitions of HMC potential energy is in excess of 0.3, and
    \item the value $\hat{R}$ (comparison of between chain and within chain estimates) is less than 1.05.
\end{enumerate}
Example corner plots of the hyperparameter and latent parameters can be found in \autoref{sec:app_fit}.
\drafting{Do we need specifics of MC parameters and exit conditions?}

\begin{table}
    \caption{Hyperparameter distributions for core-Se\'rsic model.}
    \label{tab:hyper}
    \begin{tabular}{llc}
        \hline
        Hyperparameter & Distribution & Truncation \\
        \hline
        $\mu_{\log_{10}\Sigma_\mathrm{b}}$ & $\mathcal{N}(10, 1)$ & None \\
        $\sigma_{\log_{10}\Sigma_\mathrm{b}}$ & $\mathcal{N}(0, 0.05)$ & $x>0$ \\
        $\lambda_\gamma$ & $\mathrm{Exp}(10)$ & None \\
        $\mu_n$ & $\mathcal{N}(4,2)$ & $0 < x \leq 15$ \\
        $\sigma_n$ & $\mathcal{N}(0,2)$ & $x>0$ \\
        $\sigma_\alpha$ & $\mathcal{N}(0, 20)$ & $x>0$ \\
        $\sigma_{r_\mathrm{b}}$ & $\mathcal{N}(0, 0.2)$ & $x>0$ \\
        $\sigma_{R_\mathrm{e}}$ & $\mathcal{N}(0, 20)$ & $x>0$ \\
        $\mu_{\sigma_{\log_{10}\Sigma}}$ & $\mathcal{N}(0, 1)$ & $x>0$ \\
        $\sigma_{\sigma_{\log_{10}\Sigma}}$ & $\mathcal{N}(0, 0.2)$ & $x>0$ \\
        \hline
    \end{tabular}
\end{table}


\begin{figure}
    \centering
    \includegraphics[width=0.4\textwidth]{dag}
    \caption{Directed acyclic graph of the core-Se\'rsic model of projected mass density within the Bayesian hierarchical framework. Single-line circle nodes represent fit parameters, double-line circles represent measured quantities (the data), and diamond nodes represent deterministic quantities. The particular distribution connecting nodes is written below the corresponding black square, with a subscript `T' indicating a truncated distribution, and the subscript `likelihood' indicating the likelihood function. The $R$ box indicates variables fit for each radial bin, and the projection box indicates variables specific to each projection realisation. Note that we additionally use $\hat{\Sigma}$ to distinguish the calculated value of surface density from the measured value $\Sigma$.}
    \label{fig:dag}
\end{figure}

\begin{figure}
    \centering
    \includegraphics[width=0.4\textwidth]{density}
    \caption{Density profiles with errors for select representative runs.}
    \label{fig:density}
\end{figure}

\begin{figure}
    \centering
    \includegraphics[width=0.4\textwidth]{rb-kick}
    \caption{
        Bayesian estimate of the merger remnant core size $r_b$, scaled to the core size of the pre-merger remnant $r_{b,0}$, as a function of kick velocity $\vk$.
        The core size distributions are shown as violin plots, with the median core size indicated by the central mark.
        Larger kick velocities are correlated with larger core sizes. Additionally, a greater spread in the distribution of core sizes over different viewing projections of the merger remnant is associated with larger kick velocities.}
    \label{fig:vkrb}
\end{figure}

\begin{figure}
    \centering
    \includegraphics[width=0.4\textwidth]{rb_pdf}
    \caption{Probability density function of core radius, where kick velocity is MC sampled from Zlochower relation, and pumped through our fitted model in \autoref{fig:vkrb}.}
    \label{fig:rb_pdf}
\end{figure}


\subsection{Integral field unit kinematics}
We create mock integral field unit (IFU) observations using \drafting{ reference to Cappellari maybe?}. 
Prior to performing the analysis, we reorientate each merger remnant so that the $z$-axis coincides with the minor axis of the reduced inertia tensor.

\drafting{Describe voronoi tesselation}.
For each Voronoi bin, we follow \citet{vandermarel1993} and decompose the line of sight (LOS) velocity into a series of Gauss-Hermite functions, described by the mean radial velocity $V$, the velocity dispersion $\sigma$, asymmetric deviations $h_3$, and symmetric deviations $h_4$.
The line profile is thus described by:
\begin{equation}
    \mathfrak{L} = \frac{1}{\sqrt{2\pi}\sigma} e^{-w^2/2} \left\{ 1 + \sum_{j=3}^4 h_j H_j(w) \right\},
\end{equation}
where $w \equiv (v_\mathrm{LOS} - V)/\sigma$, and the Hermite polynomials $H_3$ and $H_4$ are defined:
\begin{align}
    H_3 &= \frac{1}{\sqrt{3}} (2w^3 - 3w) \nonumber \\
    H_4 &= \frac{1}{\sqrt{24}} (4w^4 - 12w^2 + 3).
\end{align}
A recent comprehensive study of LOS velocity distribution fitting applied to elliptical galaxies can be found in \citet{mehrgan2023}.

We observe a trend of the Voronoi bins with $h_4<0$ extending to larger radii for remnants with larger recoil kicks than for those remnants with smaller recoil kicks.
To quantify this recoil velocity dependence, we calculate a radially-varying mass-weighted $h_4$ parameter analogous to the observational spin parameter $\lambda$:
\begin{equation}
    \langle h_4 \rangle = \frac{\langle R h_4 \rangle}{\langle R \rangle},
\end{equation}
where the $\langle\rangle$ brackets indicate a mass-weighted average (thus equivalent to a flux weighted average in the case of constant mass-to-light ratio).
We observe an overall shift to more negative values of $h_4$ for kick velocities up until $\vk\sim400\,\kmps$ \drafting{check this, currently just read off plot}, above which the radial $h_4$ profile is not sensitive to increasing kick velocity. 
Additionally, all $h_4$ profiles are radially decreasing, obtaining a local minimum at $R/\Reff\simeq 0.2$, and a modestly increasing by $R/\Reff\simeq 0.3$.

\drafting{Why is this 
\begin{enumerate}
    \item decreasing with increasing kicks,
    \item saturating beyond some point,
    \item displaying a dip at 0.2 Re??
\end{enumerate}
}


\begin{figure*}
    \centering
    \includegraphics[width=\textwidth]{IFU_v0000}
    \caption{
        Mock integral field unit kinematic maps of the $\vk=0\,\kmps$ remnant, plotted out to $0.5\,\Reff$. 
        Note the unordered rotation $V$, and the centrally-concentrated velocity dispersion $\sigma$.
        The majority of the merger remnant within this aperture has $h_4>0$, with a small concentration of $h_4<0$ in an annulus about the central regions.
    }
    \label{fig:IFU0}
\end{figure*}

\begin{figure*}
    \centering
    \includegraphics[width=\textwidth]{IFU_v0900}
    \caption{
        Mock integral field unit kinematic maps of the $\vk=900\,\kmps$ remnant, again plotted out to $0.5\,\Reff$. 
        Contrasting to the $\vk=0\,\kmps$ case in \autoref{fig:IFU0}, this IFU map has a much greater extent of $h_4<0$ that also reaches more extreme values of $h_4$ \drafting{check this}. 
    }
    \label{fig:IFU900}
\end{figure*}

\begin{figure}
    \centering
    \includegraphics[width=0.4\textwidth]{h4}
    \caption{
        Plot of the flux-weighted value of $h_4$ as a function of mean bin distance $R$ (scaled to the half mass radius) for each merger remnant simulated.
        The lines are coloured by the kick velocity. 
        Immediately apparent is the decrease in $\langle h_4 \rangle$ with increasing $\vk$.
        All merger remnants have a global minimum in $\langle h_4 \rangle$ at $R/\Reff\simeq0.2$.
    }
    \label{fig:h4}
\end{figure}


% DISCUSSION
\section{Discussion}
\label{sec:discussion}



% CONCLUSION
\section{Conclusions}
\label{sec:conclusions}

% FINAL BITS
\section*{acknowledgments}
A.R. acknowledges the support by the University of Helsinki Research Foundation.
A.R., acknowledge the support
by the European Research Council via ERC Consolidator Grant KETJU (no. 818930) and the support of the Academy of Finland grant 339127.

The numerical simulations used computational resources provided by
the CSC -- IT centre for Science, Finland.

\section*{Author contributions}
We list here the roles and contributions of the authors according to the Contributor Roles Taxonomy (\href{https://credit.niso.org}{CRediT}). 
\textbf{AR}:

\section*{Software}
\ketju{} \citep{mannerkoski2023,rantala2017},
\gadget{} \citep{springel2021},
NumPy \citep{harris2020},
SciPy \citep{virtanen2020},
Matplotlib \citep{hunter2007},
pygad \citep{rottgers2020}.


%%%%%%%%%%%%%%%%%%%%%%%%%%%%%%%%%%%%%%%%%%%%%%%%%%
\section*{Data Availability}
The data underlying this article will be shared on reasonable request to the corresponding author.




%%%%%%%%%%%%%%%%%%%% REFERENCES %%%%%%%%%%%%%%%%%%

% The best way to enter references is to use BibTeX:

\bibliographystyle{mnras}
\bibliography{ref} % if your bibtex file is called example.bib


% Alternatively you could enter them by hand, like this:
% This method is tedious and prone to error if you have lots of references
%\begin{thebibliography}{99}
%\bibitem[\protect\citeauthoryear{Author}{2012}]{Author2012}
%Author A.~N., 2013, Journal of Improbable Astronomy, 1, 1
%\bibitem[\protect\citeauthoryear{Others}{2013}]{Others2013}
%Others S., 2012, Journal of Interesting Stuff, 17, 198
%\end{thebibliography}

%%%%%%%%%%%%%%%%%%%%%%%%%%%%%%%%%%%%%%%%%%%%%%%%%%

%%%%%%%%%%%%%%%%% APPENDICES %%%%%%%%%%%%%%%%%%%%%

\appendix
\section{Triaxiality}\label{sec:app_triax}
We present the triaxiality of all sixteen merger remnants in the analysis at the time when the SMBH has settled.
We determine the ratios $b/a$ and $c/a$, where $a$, $b$, and $c$ are the eigenvalues of the reduced inertia tensor $I_\mathrm{red}$ and $c\leq b\leq a$. 
The tensor $I_\mathrm{red}$ is determined by binning the stellar component of the remnant into 20 radial shells from $10^{-3} R_\mathrm{vir}$ to $10^{-1} R_\mathrm{vir}$: i.e., the binning of particles is not cumulative. 
The variation of the ratios $b/a$ and $c/a$ is plotted as a function of radius in \autoref{fig:triax}.
All remnants display a high degree of symmetry, with the minimum value of $b/a\simeq 0.8$, and the minimum value of $c/a\simeq0.7$.
With a higher kick velocity, there is a slight tendency to have a higher degree of symmetry at radii $R\lesssim 10\,\mathrm{kpc}$, and a slightly lesser degree of symmetry at radii beyond this distance, compared to lower kick velocity remnants. 
The difference is however minimal between all merger remnants. 


\begin{figure}
    \centering
    \includegraphics[width=0.5\textwidth]{triaxiality}
    \caption{
        Triaxiality ratios $b/a$ and $c/a$ as a function of radius for all merger remnants in the analysis.
        Lines are colour coded by the kick velocity.
        Note that the $\vk=0\,\kmps$ case is equivalent to the pre-merger remnant. 
    }
    \label{fig:triax}
\end{figure}

\section{Core-Se\'rsic fit parameter estimates}\label{sec:app_fit}

\begin{figure*}
    \centering
    \includegraphics[width=\textwidth]{all-kick}
    \caption{
        Violin plots of the six parameters in the core-Se\'sic model (from top left to bottom right) as a function of the kick velocity: the core radius normalised to the pre-merger core radius $\rb/r_{\mathrm{b},0}$, the density at the core radius $\log_{10}\Sigma_\mathrm{b}$, the core slope $\gamma$, the effective radius $\Reff$, the profile transition index $\alpha$, and the Se\'rsic index $n$.
    }
    \label{fig:csparams}
\end{figure*}

\begin{figure*}
    \centering
    \includegraphics[width=\textwidth]{corner_latent_900}
    \caption{
        Corner plot of the latent parameters in the core-Se\'rsic model for the $\vk=960\,\kmps$ instance. \drafting{UPDATE TO 900}
        Contours indicate, from yellow to dark blue, the 25\%, 50\%, 75\%, and 99\% highest density intervals (HDIs), with sample draws beyond the 99\% HDI shown as individual points. 
        All distributions are unimodal with little cross-correlation between the latent variables.
    }
\end{figure*}


%%%%%%%%%%%%%%%%%%%%%%%%%%%%%%%%%%%%%%%%%%%%%%%%%%


% Don't change these lines
\bsp	% typesetting comment
\label{lastpage}
\end{document}

% End of mnras_template.tex
