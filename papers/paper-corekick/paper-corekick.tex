% mnras_template.tex 
%
% LaTeX template for creating an MNRAS paper
%
% v3.0 released 14 May 2015
% (version numbers match those of mnras.cls)
%
% Copyright (C) Royal Astronomical Society 2015
% Authors:
% Keith T. Smith (Royal Astronomical Society)

% Change log
%
% v3.0 May 2015
%    Renamed to match the new package name
%    Version number matches mnras.cls
%    A few minor tweaks to wording
% v1.0 September 2013
%    Beta testing only - never publicly released
%    First version: a simple (ish) template for creating an MNRAS paper

%%%%%%%%%%%%%%%%%%%%%%%%%%%%%%%%%%%%%%%%%%%%%%%%%%
% Basic setup. Most papers should leave these options alone.
\documentclass[fleqn,usenatbib]{mnras}

% MNRAS is set in Times font. If you don't have this installed (most LaTeX
% installations will be fine) or prefer the old Computer Modern fonts, comment
% out the following line
\usepackage{newtxtext,newtxmath}
% Depending on your LaTeX fonts installation, you might get better results with one of these:
%\usepackage{mathptmx}
%\usepackage{txfonts}

% Use vector fonts, so it zooms properly in on-screen viewing software
% Don't change these lines unless you know what you are doing
\usepackage[T1]{fontenc}

% Allow "Thomas van Noord" and "Simon de Laguarde" and alike to be sorted by "N" and "L" etc. in the bibliography.
% Write the name in the bibliography as "\VAN{Noord}{Van}{van} Noord, Thomas"
\DeclareRobustCommand{\VAN}[3]{#2}
\let\VANthebibliography\thebibliography
\def\thebibliography{\DeclareRobustCommand{\VAN}[3]{##3}\VANthebibliography}


%%%%% AUTHORS - PLACE YOUR OWN PACKAGES HERE %%%%%

% Only include extra packages if you really need them. Common packages are:
\usepackage{graphicx}	% Including figure files
\usepackage{amsmath}	% Advanced maths commands
%\usepackage{amssymb}	% Extra maths symbols
\usepackage{bm}         % Bold math fonts

%%%%%%%%%%%%%%%%%%%%%%%%%%%%%%%%%%%%%%%%%%%%%%%%%%

%%%%% AUTHORS - PLACE YOUR OWN COMMANDS HERE %%%%%
% drafting macros
\usepackage{xcolor}
\newcommand{\drafting}[1]{
{\leavevmode\color[RGB]{224, 77, 24}#1}
}

% Please keep new commands to a minimum, and use \newcommand not \def to avoid
% overwriting existing commands. Example:
%\newcommand{\pcm}{\,cm$^{-2}$}	% per cm-squared
\newcommand{\ketju}{\textsc{Ketju}}                           % code name 
\newcommand{\mstar}{\textsc{mstar}}                           % code name
\newcommand{\gadget}{\textsc{gadget-4}}                       % code name

\newcommand{\Msun}{\ensuremath{\mathrm{M}_{\sun}}}            % Solar Mass
\newcommand{\kmps}{\ensuremath{\mathrm{km}\,\mathrm{s}^{-1}} }% km/s formatting
\newcommand{\Reff}{\ensuremath{R_\mathrm{e}}}                 % effective radius
\newcommand{\rb}{\ensuremath{r_\mathrm{b}}}                   % core radius
\newcommand{\vk}{\ensuremath{v_\mathrm{kick}}}                % kick velocity

\newcommand{\dd}[1]{\ensuremath{\mathrm{d}#1}}                % integral dx
\newcommand{\dv}[2]{\ensuremath{\frac{\dd{#1}}{\dd{#2}}}}     % derivative
\newcommand{\vb}[1]{\ensuremath{\bm{#1}}}                     % vectors

\graphicspath{{figures/}}

%%%%%%%%%%%%%%%%%%%%%%%%%%%%%%%%%%%%%%%%%%%%%%%%%%

%%%%%%%%%%%%%%%%%%% TITLE PAGE %%%%%%%%%%%%%%%%%%%

% Title of the paper, and the short title which is used in the headers.
% Keep the title short and informative.
\title[Cores from kicks]{Cores from kicks: constraints for observations of gravitational recoil in elliptical galaxies}

% The list of authors, and the short list which is used in the headers.
% If you need two or more lines of authors, add an extra line using \newauthor
\author[A. Rawlings et al.]{
Alexander Rawlings,$^{1}$\thanks{E-mail: alexander.rawlings@helsinki.fi}
et al.
\vspace*{0.1cm}\\%
% List of institutions
$^{1}$
Department of Physics,
Gustaf H\"allstr\"omin katu 2, FI-00014, University of Helsinki, Finland
\\%
$^{2}$
Max-Planck-Institut f\"ur Astrophysik, Karl-Schwarzchild-Str 1, D-85748 Garching, Germany
}

% These dates will be filled out by the publisher
\date{Accepted XXX. Received YYY; in original form ZZZ}

% Enter the current year, for the copyright statements etc.
\pubyear{2023}

% Don't change these lines
\begin{document}
\label{firstpage}
\pagerange{\pageref{firstpage}--\pageref{lastpage}}
\maketitle

% Abstract of the paper
\begin{abstract}
\drafting{Investigate the relationship between core size and kick velocity}
\end{abstract}

% Select between one and six entries from the list of approved keywords.
% Don't make up new ones.
\begin{keywords}
black hole physics -- galaxies: kinematics and dynamics -- methods: numerical -- software: simulations
\end{keywords}

%%%%%%%%%%%%%%%%%%%%%%%%%%%%%%%%%%%%%%%%%%%%%%%%%%

%%%%%%%%%%%%%%%%% BODY OF PAPER %%%%%%%%%%%%%%%%%%


% INTRODUCTION
\section{Introduction}
\drafting{Some general intro here}

% NUMERICAL
\section{Numerical Simulations}
\label{sec:num_sims}

\subsection{Simulation Code}
\drafting{Taken from eccentricity paper, leave here as a guide for a similar description about Ketju}
We construct a number of idealised galaxy merger simulations, which we evolve with our new version of the \ketju{} code \citep{mannerkoski2023,rantala2017} coupled with \gadget{}.
The dynamics of
SMBHs, and stars in a small region around them, are integrated with an algorithmically regularised integrator \citep{rantala2020}, whereas the dynamics of the remaining particles
is computed with the \gadget{} \citep{springel2021} fast multiple method (FMM) with second order multipoles. Together with hierarchical time integration this allows for
symmetric interactions and manifest momentum conservation. \ketju{} also includes post-Newtonian (PN) correction terms up to order 3.5 between each pair of SMBHs \citep{blanchet2014}. 

\subsection{Initial Conditions}
\drafting{Rantala 2018 ICs}
We model the merger of two gas-poor elliptical galaxies using collisionless merger simulations.
Our galaxy initial conditions are chosen to match the model IC-3 presented in \citet{rantala2019}. 
The galaxy is thus represented as a multicomponent sphere, including a stellar component of total mass $M_\star\sim 1.38\times10^{11}\,\Msun$ embedded within a DM \drafting{ensure abbreviation stated} component of total mass $M_\mathrm{DM}=2.5\times10^{13}\,\Msun$, and at the centre a SMBH with mass $M_\bullet=2.93\times10^{9}\,\Msun$.
The stellar and DM components each follow a \citet{dehnen1993} profile with shape parameter $\gamma=1$, and scale radius $a_\star=3.9\,\mathrm{kpc}$ and $a_\mathrm{DM}=245\,\mathrm{kpc}$, respectively\footnote{A Dehnen profile with $\gamma=1$ is commonly referred to as a Hernquist profile.}.
The density profile is given by:
\begin{equation}\label{eq:dehnen}
    \rho_\star(r) = \frac{(3-\gamma)M_\star}{4\pi} \frac{a}{r^\gamma (r+a)^{(4-\gamma)}}.
\end{equation}
The mass of a stellar particle is set to $m_\star=5\times10^4\,\Msun$, and the mass of a DM particle is set to $m_\mathrm{DM}=5\times10^6\,\Msun$.

\subsection{Merger Simulations}
We merge two independently Monte Carlo sampled galaxy ICs, creating an equal mass galaxy merger with a near-radial orbit. 
The merger orbit has an initial separation of $D=30\,\mathrm{kpc}$, a first pericentre distance of $r_\mathrm{peri}=2\,\mathrm{kpc}$, and an initial eccentricity of $e_0=0.97$.
The merger configuration results in a rapid coalescence of the BH binary, however we note that the merger timescale we observe is driven by stochasticity in the binary eccentricity \citep{nasim2020,rawlings2023}.
We then select the snapshot just prior ($\sim8\,\mathrm{Myr}$) to the GW-driven SMBH merger, and generate 31 `child' simulations, where each child has a unique gravitational recoil kick velocity $\vk$ prescribed along the $x$-axis\footnote{As the $x$-axis is in the global coordinate frame, the direction of the kick is essentially random with respect to the angular momentum and inertia tensors. Additionally, the merger remnant displays almost no triaxiality, as seen in the $\vk=0\,\kmps$ case in \autoref{sec:app_triax}. We check for this by directing the recoil kick along the global $y$-axis, and final identical evolution to the $x$-axis case.}, ranging from $0\,\kmps$ to $1800\,\kmps$ (inclusive), in $60\,\kmps$ increments.
The upper limit on $\vk$ is chosen to match the escape velocity of the centre of the merger remnant, $v_\mathrm{esc}$.
We additionally run one simulation with a recoil kick above $v_\mathrm{esc}$, with $\vk=2000\,\kmps$.
Each child simulation is run until the median remnant SMBH velocity relative to the galaxy CoM \drafting{check abbreviation} velocity is less than $10\,\kmps$ for at least $0.1\,\mathrm{Gyr}$.
The time to this point, from the time of SMBH binary coalescence, we term the `infall time'.
Those simulations where the maximum displacement of the kicked SMBH exceeds $30\,\mathrm{kpc}$ are not included in the analysis, as we expect that observationally a SMBH further than this distance from the centre of its host galaxy would not be identified with that galaxy.
In practice, this limits our sample of analysed simulations to those with $\vk\leq 900\,\kmps$, thus totalling sixteen simulations in our analysis.
Interactions between stellar particles are softened with a softening length of $\varepsilon = 2.5\,\mathrm{pc}$, and the \ketju{} region radius is set to $r_{\rm ketju}=3\varepsilon = 7.5\,\mathrm{pc}$.


% RESULTS
\section{Results}\label{sec:results}

\drafting{
    Main points to hammer home:
    \begin{enumerate}
        \item Core radius, break density, sersic index have a dependence on the kick velocity, whereas other parameters (e.g. effective radius) do not.
        \item Increasing kick velocity induces a decrease in the overall $\langle h_4 \rangle$ profile
        \item Explanation of h4 trends through orbit decompostion.
    \end{enumerate}
}

\subsection{Infall time}
\drafting{
    \textbf{Atte's section}. Would be good to have:
    \begin{enumerate}
        \item Plot of infall time (i.e., time for stability criterion to be reached) as a function of kick velocity
        \item Number of oscillations before settling?
        \item Plot of maximum displacement of kicked BH as a function of kick velocity. How does this compare to core size? Which simulations have BH kicked beyond the core?
    \end{enumerate}
}
\begin{figure}
    \centering
    \includegraphics[width=0.4\textwidth]{infall_time}
    \caption{Plot of infall time (time to stability criterion) as a function of kick velocity.}
    \label{fig:infall}
\end{figure}
\begin{figure}
    \centering
    \includegraphics[width=0.4\textwidth]{max_disp}
    \caption{Plot of the maximum BH displacement as a function of kick velocity. Where do the displacement (red) and core size (green) lines intersect?}
    \label{fig:max_disp}
\end{figure}


\subsection{Density profiles}
We fit the six-parameter core-S\'ersic profile to the projection of each simulated merger remnants:
\begin{equation}\label{eq:cs}
    \Sigma(R) = \Sigma' \left[1 + \left(\frac{\rb}{R}\right)^\alpha \right]^{\gamma/\alpha} \exp\left[-b \left(\frac{R^\alpha + \rb^\alpha}{\Reff^\alpha}\right)^{(1/\alpha n)}\right]
\end{equation}
where
\begin{equation}
    \Sigma' = \Sigma_\mathrm{b} 2^{-\gamma/\alpha} \exp\left[b\left(2^{1/a} \frac{\rb}{\Reff}\right)^{1/n}\right].
\end{equation}
Here $\rb$ is the core radius, $\Sigma_\mathrm{b}$ is the density at the core radius, $\gamma$ is the core slope, $\Reff$ is the effective (half-light) radius, $\alpha$ is the profile transition index, and $n$ is the S\'ersic index.
We refer to the collective vector of these parameters as $\vb{\theta}$.
We view the simulated merger remnant from fifteen different angles, and use a Bayesian hierarchical model (HM) to fit the model parameters.
As each of the projections are unordered and exchangeable (i.e., there is no distinguishing label for the projections), and each projection is of the same merger remnant, a HM naturally lends itself.
The idea behind the HM is straightforward: for each projection, we infer the six parameters of $\vb{\theta}$, and assume that these values are draws from a common, global distribution of all possible vectors $\vb{\theta}$.
Hence, we are effectively inferring the global hyperparameters $\vb{\theta}^\mathrm{hyp}$ that describe the latent parameters $\vb{\theta}$ that we are interested in.
This idea is demonstrated in the directed acyclic graph (DAG) for the model, shown in \autoref{fig:dag}.

We use weakly-constrained priors for each of the hyperparameters in $\vb{\theta}^\mathrm{hyp}$, where these priors are not necessarily Gaussian (given that distributions of positive-constrained quantities, like radius, should not be described by a distribution with support on the full real space).
The distributions\footnote{Note that the normal distributions $\mathcal{N}$ are parameterised using the standard deviation $\sigma$, as opposed to the variance $\sigma^2$ as is common in many statistics references.} of the hyperparameters are given in \autoref{tab:hyper}.
We use the Hamiltonian Monte Carlo code \textsc{Stan} \citep{standevelopmentteam2018} with four chains each of 2000 posterior sample draws (excluding an additional 2000 burn-in draws) to fit the model parameters, ensuring that:
\begin{enumerate}
    \item the number of diverging draws is less than 5\%,
    \item the effective draws per transition exceeds 0.001,
    \item the sampler transitions of HMC potential energy is in excess of 0.3, and
    \item the value $\hat{R}$ (comparison of between-chain and within-chain estimates) is less than 1.05.
\end{enumerate}
Example corner plots of the latent parameters can be found in \autoref{sec:app_fit}.
\drafting{Do we need specifics of MC parameters and exit conditions?}

In agreement with previous studies \citep{nasim2021b}, we find an increase in the core radius $\rb$ with kick velocity.
\drafting{Need to check if others discuss this:} We find that the core radius saturates to a value of $\sim2$ times that of the merger remnant with no kick velocity by $\vk = 600\,\kmps$, corresponding to $0.3 v_\mathrm{esc}$. 
As shown in \autoref{fig:csparams}, the density at the core radius $\Sigma_\mathrm{b}$ decreases monotonically with kick velocity, consistent with the picture of the kicked SMBH removing more mass the more number of oscillations it is able to undergo before settling to Brownian motion limits. 
Interestingly, we observe a similar decrease in S\'ersic index $n$ with increasing kick velocity, although no correlation is seen between $n$ and $\Sigma_\mathrm{b}$ in the latent parameter corner plots (see for example \autoref{fig:csparams}).

\subsection{Predicted core size distribution}
Following \citet{nasim2021b}, we fit a power law of the form
\begin{equation}\label{eq:vkrb}
    \frac{\rb}{r_{\mathrm{b},0}} = K \left( \frac{\vk}{v_\mathrm{esc}} \right)^\beta + 1
\end{equation}
to our simulations, which is thus valid for $0\leq\vk/\kmps\leq900$, or equivalently $0\leq\vk/v_\mathrm{esc}\leq 0.5$.
We estimate the parameters $K$ and $\beta$ using HMC, and find median values of $K=3.70$ and $\beta=0.562$.

Using \autoref{eq:vkrb}, we push through the distribution of kick velocities from \citet{zlochower2015} for dry mergers assuming random azimuthal spin alignment (their Fig. 15, right panel).
The dimensionless spin parameter $\alpha_\bullet$ for the dry mergers follows a beta distribution, namely:
\begin{equation}
    P_\mathrm{Zlochower}(\alpha_\bullet) \propto (1-\alpha_\bullet)^{4.66884-1} \alpha_\bullet^{10.5868-1}.
\end{equation}
The range of kick velocities predicted by the model varies from $\vk=0\,\kmps$ to $\vk\sim4000\,\kmps$ \drafting{check this}.
Using transformation sampling allows us to obtain a distribution of core radii predicted by a given kick velocity model, as shown in \autoref{fig:rb_pdf}.
We find that the mode of the forward-folded core radius distribution is $1.64\,\mathrm{kpc}$ ($3.12\,r_{\mathrm{b},0}$).
Thus, assuming the \citet{zlochower2015} model to be a reasonable description of SMBH recoil velocities, most massive elliptical galaxies should have a non-negligible core \drafting{add references}. 


\begin{table}
    \caption{Hyperparameter distributions for core-S\'ersic model. $\mathcal{N}$ indicates a normal distribution, $\mathrm{Exp}(\lambda)$ indicates an exponential distribution.}
    \label{tab:hyper}
    \begin{tabular}{llc}
        \hline
        Hyperparameter & Distribution & Truncation \\
        \hline
        $\mu_{\log_{10}\Sigma_\mathrm{b}}$ & $\mathcal{N}(10, 1)$ & None \\
        $\sigma_{\log_{10}\Sigma_\mathrm{b}}$ & $\mathcal{N}(0, 0.05)$ & $x>0$ \\
        $\lambda_\gamma$ & $\mathrm{Exp}(10)$ & None \\
        $\mu_n$ & $\mathcal{N}(4,2)$ & $0 < x \leq 15$ \\
        $\sigma_n$ & $\mathcal{N}(0,2)$ & $x>0$ \\
        $\sigma_\alpha$ & $\mathcal{N}(0, 20)$ & $x>0$ \\
        $\sigma_{r_\mathrm{b}}$ & $\mathcal{N}(0, 0.2)$ & $x>0$ \\
        $\sigma_{R_\mathrm{e}}$ & $\mathcal{N}(0, 20)$ & $x>0$ \\
        $\mu_{\tau_{\log_{10}\Sigma}}$ & $\mathcal{N}(0, 1)$ & $x>0$ \\
        $\sigma_{\tau_{\log_{10}\Sigma}}$ & $\mathcal{N}(0, 0.2)$ & $x>0$ \\
        \hline
    \end{tabular}
\end{table}


\begin{figure}
    \centering
    \includegraphics[width=0.4\textwidth]{dag}
    \caption{Directed acyclic graph of the core-S\'ersic model of projected mass density within the Bayesian hierarchical framework. Single-line circle nodes represent fit parameters, double-line circles represent measured quantities (the data), and diamond nodes represent deterministic quantities. The particular distribution connecting nodes is written below the corresponding black square, with a subscript `T' indicating a truncated distribution, and the subscript `likelihood' indicating the likelihood function. The $R$ box indicates variables fit for each radial bin, and the projection box indicates variables specific to each projection realisation. Note that we additionally use $\hat{\Sigma}$ to distinguish the calculated value of surface density from the measured value $\Sigma$. The various distributions are normal ($\mathcal{N}$), exponential ($\mathrm{Exp}$), and Rayleigh ($\mathrm{Ray}$).}
    \label{fig:dag}
\end{figure}

\begin{figure}
    \centering
    \includegraphics[width=0.4\textwidth]{density}
    \caption{
        Surface density profiles with 25\% Bayesian HDI for select representative runs.
        Increasing the kick velocity induces a shallower density profile in the inner regions.
        }
    \label{fig:density}
\end{figure}

\begin{figure}
    \centering
    \includegraphics[width=0.4\textwidth]{rb-kick}
    \caption{
        Bayesian estimate of the merger remnant core size $r_b$, scaled to the core size of the pre-merger remnant $r_{b,0}$, as a function of kick velocity $\vk$.
        The core size distributions are shown as box plots, with the median core size indicated by the central mark.
        Larger kick velocities are correlated with larger core sizes. Additionally, a greater spread in the distribution of core sizes over different viewing projections of the merger remnant is associated with larger kick velocities.}
    \label{fig:vkrb}
\end{figure}

\begin{figure}
    \centering
    \includegraphics[width=0.4\textwidth]{rb_pdf}
    \caption{Probability density function of transformation sampled core radius. The SMBH kick velocity is Monte Carlo sampled from the \citet{zlochower2015} relation assuming randomly-aligned azimuthal spins, and pushed through the fitted model in \autoref{eq:vkrb}. The predicted kick velocities range from $\vk=0\,\kmps$ to $\vk\sim4000\,\kmps$. \drafting{check this}}
    \label{fig:rb_pdf}
\end{figure}


\subsection{Integral field unit kinematics}
We create mock integral field unit (IFU) observations using \drafting{ reference to Cappellari maybe?}. 
Prior to performing the analysis, we reorientate each merger remnant so that the $z$-axis coincides with the minor axis of the reduced inertia tensor.

\drafting{Describe voronoi tesselation}.
For each Voronoi bin, we follow \citet{vandermarel1993} and decompose the line of sight (LOS) velocity into a series of Gauss-Hermite functions, described by the mean radial velocity $V$, the velocity dispersion $\sigma$, asymmetric deviations $h_3$, and symmetric deviations $h_4$.
The line profile is thus described by:
\begin{equation}
    \mathfrak{L} = \frac{1}{\sqrt{2\pi}\sigma} e^{-w^2/2} \left\{ 1 + \sum_{j=3}^4 h_j H_j(w) \right\},
\end{equation}
where $w \equiv (v_\mathrm{LOS} - V)/\sigma$, and the Hermite polynomials $H_3$ and $H_4$ are defined:
\begin{align}
    H_3 &= \frac{1}{\sqrt{3}} (2w^3 - 3w) \nonumber \\
    H_4 &= \frac{1}{\sqrt{24}} (4w^4 - 12w^2 + 3).
\end{align}
A recent comprehensive study of LOS velocity distribution fitting applied to elliptical galaxies can be found in \citet{mehrgan2023}.

We observe a trend of the Voronoi bins with $h_4<0$ extending to larger radii for remnants with larger recoil kicks than for those remnants with smaller recoil kicks.
To quantify this recoil velocity dependence, we calculate a radially-varying mass-weighted $h_4$ parameter analogous to the observational spin parameter $\lambda$:
\begin{equation}
    \langle h_4(R') \rangle = \frac{\langle R h_4 \rangle}{\langle R \rangle} = \frac{\sum_i^N M_i h_{4,i} R_i}{\sum_i^N M_i R_i} \;\forall R_i \leq R',
\end{equation}
where the $\langle\rangle$ brackets indicate a mass-weighted average (thus equivalent to a flux weighted average in the case of constant mass-to-light ratio).
We observe an overall shift to more negative values of $h_4$ for kick velocities up until $\vk\sim400\,\kmps$ \drafting{check this, currently just read off plot}, above which the radial $h_4$ profile is not sensitive to increasing kick velocity. 
Additionally, all $h_4$ profiles are radially decreasing, obtaining a local minimum at $R\simeq 2\,\mathrm{kpc}$ ($R/\Reff\simeq 0.2$), and a modestly increasing for $R \gtrsim 2\,\mathrm{kpc}$.

\drafting{Jens: maybe you have some thoughts\dots Why is this 
\begin{enumerate}
    \item saturating beyond some point,
    \item displaying a dip at 0.2 Re??
\end{enumerate}
}


\begin{figure*}
    \centering
    \includegraphics[width=\textwidth]{IFU_v0000}
    \caption{
        Mock integral field unit kinematic maps of the $\vk=0\,\kmps$ remnant, plotted out to $0.5\,\Reff$. 
        Note the small magnitude of the unordered rotation $V$, and the centrally-concentrated velocity dispersion $\sigma$.
        The majority of the merger remnant within this aperture has $h_4>0$, with a small concentration of $h_4<0$ in an annulus about the central regions.
    }
    \label{fig:IFU0}
\end{figure*}

\begin{figure*}
    \centering
    \includegraphics[width=\textwidth]{IFU_v0900}
    \caption{
        Mock integral field unit kinematic maps of the $\vk=900\,\kmps$ remnant, again plotted out to $0.5\,\Reff$, with the same colour scaling as \autoref{fig:IFU0}. 
        Contrasting to the $\vk=0\,\kmps$ case in \autoref{fig:IFU0}, this IFU map has a much greater extent of $h_4<0$ that also reaches more extreme values of $h_4$. 
    }
    \label{fig:IFU900}
\end{figure*}

\begin{figure}
    \centering
    \includegraphics[width=0.4\textwidth]{h4}
    \caption{
        Plot of the flux-weighted value of $h_4$ as a function of mean bin distance $R$for each merger remnant simulated.
        The lines are coloured by the kick velocity. 
        Immediately apparent is the decrease in $\langle h_4 \rangle$ with increasing $\vk$.
        All merger remnants have a global minimum in $\langle h_4 \rangle$ at $R\simeq 2.0$, corresponding to $R/\Reff\simeq0.2$.
    }
    \label{fig:h4}
\end{figure}


\subsection{Orbit analysis}

\begin{figure*}
    \centering
    \includegraphics[width=\textwidth]{orbits}
    \caption{
        Orbit analysis of all sixteen merger remnants.
        Stellar particles are assigned to one of seven different orbital families, and binned into ten logarithmically-spaced radial shells such that $0.2 \leq R/\mathrm{kpc} \leq 30.0$.
        At radii greater than $1\,\mathrm{kpc}$, there is very little to distinguish between the different kick velocities for boxlet, $z$-tube, Keplerian, irregular, and unclassified orbits.
        Conversely, for radii $1 \leq R/\mathrm{kpc} \leq 10$, there is a clear gradient for $\pi$-box and $x$-tube families.
        For higher kick velocities, there is a tendency to a higher fraction of $\pi$-box orbits compared to lower kick velocities.
        An inverse trend is visible for $x$-tube orbits. 
    }
    \label{fig:orbits}
\end{figure*}

To understand the kinematic maps, we perform an orbit analysis on the sixteen merger remnants in our sample following \citet{frigo2021}.
\drafting{Discuss how the orbit analysis is done, in particular the different orbital families}.
First, the merger remnant is rotated so that the $z$-axis coincides with the minor axis of the reduced inertia tensor (and hence has the same orientation as the IFU maps in \autoref{fig:IFU0} and \autoref{fig:IFU900}).
The potential of the merger remnant is fit using a \drafting{self consistent field (SCF) potential, describe more?} method, providing an analytical potential in which the orbits of individual stellar particles can be integrated.
The potential of the merger remnant is checked for stability, and each stellar particle within $30\,\mathrm{kpc}$ of the centre is integrated for fifty orbits to determine which, if any, orbital resonances exist. 

As seen in \autoref{fig:orbits}, we observe a clear trend of higher kick velocity remnants having a higher fraction of $\pi$-box and boxlet orbits, and a lower fraction of $x$-tube orbits, and to some extent Keplerian orbits, than their lower kick velocity counterparts.
The other two orbital families are generally consistent between all merger remnants. 
As discussed in \citet{frigo2021}, negative values of $h_4$ indicate regions with weaker tails of the LOS velocity distribution due to the superposition of radial orbits with different orientations.
This explanation supports our observation of an increase in $\pi$-box orbits (which have no net angular momentum are typically very radial orbits) with kick velocity, and a corresponding decrease in the value of $h_4$ (comparing the IFU maps in \autoref{fig:IFU0} to \autoref{fig:IFU900}).
Equivalently, $x$-tube orbits (which rotate about the major axis of the inertia tensor) appear as a peaked LOS velocity distribution in the projection we use, leading to extended tails, and hence a more positive value of $h_4$ than for a weak tail distribution. 
All together, we can infer that as a SMBH ploughs through the surrounding stellar environment, regular $x$-tube orbits are disrupted to non-resonant $\pi$-box orbits.
With a higher kick velocity, the SMBH is able to cumulatively disturb orbits at a larger radial extent, converting a greater fraction of $x$-tube orbits to $\pi$-box orbits, resulting in the increased expanses of negative $h_4$ regions in the kinematic maps.
This also explains the overall downward shift in the $\langle h_4 \rangle$ profiles in \autoref{fig:h4} for higher kick velocities. 
A peculiar feature of \autoref{fig:h4} is the consistency in the $\langle h_4 \rangle$ profile shape.
This can be explained by the derivative of the orbital fraction distribution being consistent across all kick velocities for $x$-tube and boxlet orbits out to $\sim3\,\mathrm{kpc}$, as shown in \autoref{fig:orbits2}. 
\drafting{Thorsten: I would really appreciate your thoughts on this section, and its relation to the kinematic maps. }


% DISCUSSION
\section{Discussion}
\label{sec:discussion}



% CONCLUSION
\section{Conclusions}
\label{sec:conclusions}

% FINAL BITS
\section*{acknowledgments}
A.R. acknowledges the support by the University of Helsinki Research Foundation.
A.R., acknowledge the support
by the European Research Council via ERC Consolidator Grant KETJU (no. 818930) and the support of the Academy of Finland grant 339127.

The numerical simulations used computational resources provided by
the CSC -- IT centre for Science, Finland.

\section*{Author contributions}
We list here the roles and contributions of the authors according to the Contributor Roles Taxonomy (\href{https://credit.niso.org}{CRediT}). 
\textbf{AR}:

\section*{Software}
\ketju{} \citep{mannerkoski2023,rantala2017},
\gadget{} \citep{springel2021},
NumPy \citep{harris2020},
SciPy \citep{virtanen2020},
Matplotlib \citep{hunter2007},
pygad \citep{rottgers2020}.


%%%%%%%%%%%%%%%%%%%%%%%%%%%%%%%%%%%%%%%%%%%%%%%%%%
\section*{Data Availability}
The data underlying this article will be shared on reasonable request to the corresponding author.




%%%%%%%%%%%%%%%%%%%% REFERENCES %%%%%%%%%%%%%%%%%%

% The best way to enter references is to use BibTeX:

\bibliographystyle{mnras}
\bibliography{ref} % if your bibtex file is called example.bib


% Alternatively you could enter them by hand, like this:
% This method is tedious and prone to error if you have lots of references
%\begin{thebibliography}{99}
%\bibitem[\protect\citeauthoryear{Author}{2012}]{Author2012}
%Author A.~N., 2013, Journal of Improbable Astronomy, 1, 1
%\bibitem[\protect\citeauthoryear{Others}{2013}]{Others2013}
%Others S., 2012, Journal of Interesting Stuff, 17, 198
%\end{thebibliography}

%%%%%%%%%%%%%%%%%%%%%%%%%%%%%%%%%%%%%%%%%%%%%%%%%%

%%%%%%%%%%%%%%%%% APPENDICES %%%%%%%%%%%%%%%%%%%%%

\appendix
\section{Triaxiality}\label{sec:app_triax}
We present the triaxiality of all sixteen merger remnants in the analysis at the time when the SMBH has settled.
We determine the ratios $b/a$ and $c/a$, where $a$, $b$, and $c$ are the eigenvalues of the reduced inertia tensor $I_\mathrm{red}$ and $c\leq b\leq a$. 
The tensor $I_\mathrm{red}$ is determined by binning the stellar component of the remnant into 20 radial shells from $10^{-3} R_\mathrm{vir}$ to $10^{-1} R_\mathrm{vir}$: i.e., the binning of particles is not cumulative. 
The variation of the ratios $b/a$ and $c/a$ is plotted as a function of radius in \autoref{fig:triax}.
All remnants display a high degree of symmetry, with the minimum value of $b/a\simeq 0.8$, and the minimum value of $c/a\simeq0.7$.
With a higher kick velocity, there is a slight tendency to have a higher degree of symmetry at radii $R\lesssim 10\,\mathrm{kpc}$, and a slightly lesser degree of symmetry at radii beyond this distance, compared to lower kick velocity remnants. 
The difference is however minimal between all merger remnants. 


\begin{figure}
    \centering
    \includegraphics[width=0.5\textwidth]{triaxiality}
    \caption{
        Triaxiality ratios $b/a$ and $c/a$ as a function of radius for all merger remnants in the analysis.
        Lines are colour coded by the kick velocity.
        Note that the $\vk=0\,\kmps$ case is equivalent to the pre-merger remnant. 
    }
    \label{fig:triax}
\end{figure}

\section{Core-S\'ersic fit parameter estimates}\label{sec:app_fit}

\begin{figure*}
    \centering
    \includegraphics[width=\textwidth]{all-kick}
    \caption{
        Box plots of the six parameters in the core-S\'ersic model (from top left to bottom right) as a function of the kick velocity: the core radius normalised to the pre-merger core radius $\rb/r_{\mathrm{b},0}$, the density at the core radius $\log_{10}\Sigma_\mathrm{b}$, the core slope $\gamma$, the effective radius $\Reff$, the profile transition index $\alpha$, and the S\'ersic index $n$.
    }
    \label{fig:csparams}
\end{figure*}

\begin{figure*}
    \centering
    \includegraphics[width=\textwidth]{corner_latent_900}
    \caption{
        Corner plot of the latent parameters in the core-S\'ersic model for the $\vk=900\,\kmps$ instance.
        Contours indicate, from yellow to dark blue, the 25\%, 50\%, 75\%, and 99\% highest density intervals (HDIs), with sample draws beyond the 99\% HDI shown as individual points. 
        All distributions are unimodal with little cross-correlation between the latent variables.
    }
\end{figure*}


\section{Additional orbital plots}

\begin{figure*}
    \centering
    \includegraphics[width=\textwidth]{orbits2}
    \caption{
        Same as \autoref{fig:orbits}, but each merger remnant is shown in its own panel, with the different orbital families depicted by colour, to better demonstrate the radial fraction of orbits for a given merger.
    }
    \label{fig:orbits2}
\end{figure*}

%%%%%%%%%%%%%%%%%%%%%%%%%%%%%%%%%%%%%%%%%%%%%%%%%%


% Don't change these lines
\bsp	% typesetting comment
\label{lastpage}
\end{document}

% End of mnras_template.tex
