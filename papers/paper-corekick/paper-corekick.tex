% mnras_template.tex 
%
% LaTeX template for creating an MNRAS paper
%
% v3.0 released 14 May 2015
% (version numbers match those of mnras.cls)
%
% Copyright (C) Royal Astronomical Society 2015
% Authors:
% Keith T. Smith (Royal Astronomical Society)

% Change log
%
% v3.0 May 2015
%    Renamed to match the new package name
%    Version number matches mnras.cls
%    A few minor tweaks to wording
% v1.0 September 2013
%    Beta testing only - never publicly released
%    First version: a simple (ish) template for creating an MNRAS paper

%%%%%%%%%%%%%%%%%%%%%%%%%%%%%%%%%%%%%%%%%%%%%%%%%%
% Basic setup. Most papers should leave these options alone.
\documentclass[fleqn,usenatbib]{mnras}

% MNRAS is set in Times font. If you don't have this installed (most LaTeX
% installations will be fine) or prefer the old Computer Modern fonts, comment
% out the following line
\usepackage{newtxtext,newtxmath}
% Depending on your LaTeX fonts installation, you might get better results with one of these:
%\usepackage{mathptmx}
%\usepackage{txfonts}

% Use vector fonts, so it zooms properly in on-screen viewing software
% Don't change these lines unless you know what you are doing
\usepackage[T1]{fontenc}

% Allow "Thomas van Noord" and "Simon de Laguarde" and alike to be sorted by "N" and "L" etc. in the bibliography.
% Write the name in the bibliography as "\VAN{Noord}{Van}{van} Noord, Thomas"
\DeclareRobustCommand{\VAN}[3]{#2}
\let\VANthebibliography\thebibliography
\def\thebibliography{\DeclareRobustCommand{\VAN}[3]{##3}\VANthebibliography}


%%%%% AUTHORS - PLACE YOUR OWN PACKAGES HERE %%%%%

% Only include extra packages if you really need them. Common packages are:
\usepackage{graphicx}	% Including figure files
\usepackage{amsmath}	% Advanced maths commands
%\usepackage{amssymb}	% Extra maths symbols
\usepackage{bm}         % Bold math fonts

%%%%%%%%%%%%%%%%%%%%%%%%%%%%%%%%%%%%%%%%%%%%%%%%%%

%%%%% AUTHORS - PLACE YOUR OWN COMMANDS HERE %%%%%
% drafting macros
\usepackage{xcolor}
\newcommand{\drafting}[1]{
{\leavevmode\color[RGB]{224, 77, 24}#1}
}

% Please keep new commands to a minimum, and use \newcommand not \def to avoid
% overwriting existing commands. Example:
%\newcommand{\pcm}{\,cm$^{-2}$}	% per cm-squared
\newcommand{\ketju}{\textsc{Ketju}}                           % code name 
\newcommand{\mstar}{\textsc{mstar}}                           % code name
\newcommand{\gadget}{\textsc{gadget-4}}                       % code name

\newcommand{\Msun}{\ensuremath{\mathrm{M}_{\sun}}}            % Solar Mass
\newcommand{\kmps}{\ensuremath{\mathrm{km}\,\mathrm{s}^{-1}} }% km/s formatting
\newcommand{\Reff}{\ensuremath{R_\mathrm{e}}}                 % effective radius
\newcommand{\rb}{\ensuremath{r_\mathrm{b}}}                   % core radius
\newcommand{\vk}{\ensuremath{v_\mathrm{kick}}}                % kick velocity

\newcommand{\dd}[1]{\ensuremath{\mathrm{d}#1}}                % integral dx
\newcommand{\dv}[2]{\ensuremath{\frac{\dd{#1}}{\dd{#2}}}}     % derivative
\newcommand{\dnv}[3]{\ensuremath{\frac{\mathrm{d}^#1#2}{\dd{#3}^#1}}}  % nth derivative
\newcommand{\vb}[1]{\ensuremath{\bm{#1}}}                     % vectors

\graphicspath{{figures/}}

%%%%%%%%%%%%%%%%%%%%%%%%%%%%%%%%%%%%%%%%%%%%%%%%%%

%%%%%%%%%%%%%%%%%%% TITLE PAGE %%%%%%%%%%%%%%%%%%%

% Title of the paper, and the short title which is used in the headers.
% Keep the title short and informative.
\title[Cores from kicks]{Cores from kicks: constraints for observations of gravitational recoil in elliptical galaxies}

% The list of authors, and the short list which is used in the headers.
% If you need two or more lines of authors, add an extra line using \newauthor
\author[A. Rawlings et al.]{
Alexander Rawlings,$^{1}$\thanks{E-mail: alexander.rawlings@helsinki.fi}
et al.
\vspace*{0.1cm}\\%
% List of institutions
$^{1}$
Department of Physics,
Gustaf H\"allstr\"omin katu 2, FI-00014, University of Helsinki, Finland
\\%
%$^{2}$
%Max-Planck-Institut f\"ur Astrophysik, Karl-Schwarzchild-Str 1, D-85748 Garching, Germany
}

% These dates will be filled out by the publisher
\date{Accepted XXX. Received YYY; in original form ZZZ}

% Enter the current year, for the copyright statements etc.
\pubyear{2023}

% Don't change these lines
\begin{document}
\label{firstpage}
\pagerange{\pageref{firstpage}--\pageref{lastpage}}
\maketitle

% Abstract of the paper
\begin{abstract}
\drafting{Investigate the relationship between core size and kick velocity}
\end{abstract}

% Select between one and six entries from the list of approved keywords.
% Don't make up new ones.
\begin{keywords}
black hole physics -- galaxies: kinematics and dynamics -- methods: numerical -- software: simulations
\end{keywords}

%%%%%%%%%%%%%%%%%%%%%%%%%%%%%%%%%%%%%%%%%%%%%%%%%%

%%%%%%%%%%%%%%%%% BODY OF PAPER %%%%%%%%%%%%%%%%%%


% INTRODUCTION
\section{Introduction}
There is much evidence for the presence of supermassive black holes (SMBHs) at the centres of massive galaxies, with the shadow of one such SMBH being directly imaged \citep{eht2019}.
The dynamical interactions of two or more galaxies, during a galaxy merger, are thus also expected to induce an interaction of the residing SMBHs.
The interaction of two SMBHs during the merger of two galaxies is well understood to occur in a three stage process \citep{begelman1980}.
First, as the galaxies begin to dynamically interact and the SMBHs traverse through a dense stellar field, the overdensity of stars in the wakes of the SMBHs provides a restoring force which brings the SMBHs to the centre of the two interacting galaxies \citep{chandrasekhar1943}, generally torquing the SMBHs to highly radial orbits in the process. 
Second, after the SMBHs have formed a bound binary, iterative interactions with the surrounding stars ejects stars with high velocity \citep[e.g.][]{hills1980,hills1983,quinlan1996}, thus removing orbital energy and angular momentum from the SMBH binary orbit.
The removal of stars from the central regions by this slingshot mechanism produces a `core' \citep[e.g.][]{lauer1983,lauer1985,kormendy1984}, or depletion, in the luminosity profile of the remnant galaxy \citep[e.g.][]{begelman1980,hills1983,quinlan1996,rantala2018}, and can extend up to some kiloparsecs \citep{postman2012}.
Finally, at very small, sub-parsec, separations the SMBH binary loses its remaining orbital energy and angular momentum through gravitational wave (GW) emission, driving the SMBH binary to coalescence \citep{peters1963,peters1964}.

The final GW emission from the coalescing SMBH binary occurs anisotropically, and carries with it linear momentum from the system \citep[e.g.][]{gonzalez2007}.
This results in the remnant SMBH recoiling with some velocity, termed the kick velocity. 
The kick velocity is dependent on the mass ratio between the SMBHs prior to coalescence, and the magnitude and direction of angular momentum (spin) vector of each SMBH. 
As reported by \citet{campanelli2007}, symmetries in the masses or spins of the SMBHs suppresses the resulting recoil kick imparted to the coalesced SMBH.
In particular, it is found that asymmetry in the spins of the SMBHs has a larger impact on the magnitude of the recoil kick than asymmetry in the masses \citep{campanelli2007}, with antialigned spins in the orbital plane producing the largest recoil kick velocities \citep{gonzalez2007b,tichy2007}.
Numerical relativity studies indicate that while the majority of recoil kicks are of the order of a few hundred kilometres per second, in special configurations the SMBH may be imparted a kick velocity in excess of a $2000\,\kmps$, even up to $4000\,\kmps$ \citep{campanelli2007,gonzalez2007b,tichy2007}.
Taking the central escape velocity of a typical early type galaxy (ETG) as $\sim2000\,\kmps$, there may be a non-negligible fraction of ETGs with a SMBH that has escaped the galaxy \citep{madau2004}, though not so many so as to introduce considerable scatter into the observed relation between SMBH mass and stellar velocity dispersion \citep{volonteri2007}.
In cases where the kick velocity is less than the escape velocity of the inner regions of the host galaxy, the radial oscillations of the remnant SMBH as it returns to the centre of the merger remnant is expected to influence the surrounding stellar environment. \drafting{What sort of influences? References?}

Observations by \citet{komossa2008} point to a recoiling SMBH \drafting{need to read\dots}

In this work, we first aim to determine how the velocity with which the SMBH is kicked affects the surrounding stellar environment.
We then aim to determine if there is an observational signature of the SMBH recoil kick that may be used to infer the occurrence of a kicked SMBH. 
\drafting{Previous work by \dots indicates \dots}

\drafting{This paper is divided as follows:}

% NUMERICAL
\section{Numerical Simulations}
\label{sec:num_sims}

\subsection{Simulation Code}
\drafting{Taken from eccentricity paper, edited a little bit but need to ensure it is different enough.}
To investigate the effect of the recoiling SMBH on the host galaxy, and potential observational signatures of this, we run a number of numerical simulations of a galaxy merger setting using our new version of the \ketju{} code \citep{mannerkoski2023,rantala2017} coupled with \gadget{}.
\ketju{} integrates the dynamics of SMBHs, and stars in a small region thrice the stellar softening length around them, to high accuracy using the algorithmically regularised integrator \textsc{mstar} \citep{rantala2020}.
The dynamics of stellar particles beyond this small region of high integration accuracy, and all dark matter (DM) particles, is computed with the \gadget{} \citep{springel2021} fast multiple method (FMM) with second order multipoles. 
Additionally, we use hierarchical time integration, which allows for symmetric interactions and manifest momentum conservation. 
\ketju{} also includes post-Newtonian (PN) correction terms up to order 3.5 between each pair of SMBHs \citep{blanchet2014} and the fitting formula of \citet{zlochower2015} for recoil kick velocity, making \ketju{} a particularly well-suited code for investigating the consequence of SMBH recoil self-consistently in a galaxy merger environment. 

\subsection{Initial Conditions}
\drafting{Rantala 2018 ICs}
We model the merger of two gas-free elliptical galaxies using gas-free merger simulations.
Our galaxy initial conditions are chosen similarly to the model IC-3 presented in \citet{rantala2019}, however with different stellar density slope. 
The galaxy is thus represented as an isotropic multicomponent sphere, and consists of a stellar component of total mass $M_\star\sim 1.38\times10^{11}\,\Msun$ embedded within a DM component of total mass $M_\mathrm{DM}=2.5\times10^{13}\,\Msun$.
At the centre a SMBH with mass $M_\bullet=2.93\times10^{9}\,\Msun$ is placed with zero velocity.
The stellar and DM components each follow a \citet{dehnen1993} profile with shape parameter $\gamma=1$, and scale radius $a_\star=3.9\,\mathrm{kpc}$ and $a_\mathrm{DM}=245\,\mathrm{kpc}$, respectively\footnote{A Dehnen profile with $\gamma=1$ is commonly referred to as a Hernquist profile.}.
The density profile $\rho_i$ for a given component $i$ is given by:
\begin{equation}\label{eq:dehnen}
    \rho_i(r) = \frac{(3-\gamma)M}{4\pi} \frac{a}{r^\gamma (r+a)^{(4-\gamma)}}.
\end{equation}
We generate the ICs using the distribution function method following \citet{hilz2012}, where for each component (stellar and DM) the distribution function $f_i$ is computed using Eddington's formula \citep{binney2008} for each density profile $\rho_i$:
\begin{equation}
    f_i(\mathcal{E}) = \frac{1}{2\sqrt{2}\pi^2} \int_{\Phi_\mathrm{T}=0}^{\Phi_\mathrm{T}=\mathcal{E}} \dnv{2}{\rho_i}{\Phi_\mathrm{T}} \frac{\dd{\Phi_\mathrm{T}}}{\sqrt{\mathcal{E}-\Phi_\mathrm{T}}}.
\end{equation}
Here $\mathcal{E}$ is the relative energy, and $\Phi_\mathrm{T}$ is the total gravitational potential. 
The distribution functions are then sampled with discrete particles, where the mass of a stellar particle is set to $m_\star=5\times10^4\,\Msun$, and the mass of a DM particle is set to $m_\mathrm{DM}=5\times10^6\,\Msun$.
The radial velocity profiles of both stellar and DM particles is ergodic. 

As we are interested in the evolution of the galaxy merger \textit{remnant} following SMBH coalescence, the initial conditions such that that at the time of SMBH coalescence (\autoref{ssec:mergers}) the galaxy remnant agrees with observational data.
Specifically, we ensure that our remnant agrees with the half-light -- stellar mass data presented in \citet{sahu2020}, and lies within the $1\sigma$ predictive interval of the SMBH mass -- stellar velocity dispersion data given in \citet{vandenbosch2016}.

\subsection{Merger Simulations}\label{ssec:mergers}
We merge two independently Monte Carlo sampled galaxy ICs, creating an equal mass galaxy merger with a near-radial orbit. 
The merger orbit has an initial separation of $D=30\,\mathrm{kpc}$, a first pericentre distance of $r_\mathrm{peri}=2\,\mathrm{kpc}$, and an initial eccentricity of $e_0=0.97$.
The merger configuration results in a rapid coalescence of the BH binary, however we note that the merger timescale we observe is driven by stochasticity in the binary eccentricity \citep{nasim2020,rawlings2023}.
We then select the snapshot just prior ($\sim8\,\mathrm{Myr}$) to the GW-driven SMBH merger, and generate 31 `child' simulations, where each child has a unique gravitational recoil kick velocity $\vk$ prescribed along the $x$-axis\footnote{As the $x$-axis is in the global coordinate frame, the direction of the kick is essentially random with respect to the angular momentum and inertia tensors. To test the effect of differing kick directions on the merger remnant, we test directing the recoil kick along the global $y$-axis, and find identical evolution to the $x$-axis case.}, ranging from $0\,\kmps$ to $1800\,\kmps$ (inclusive), in $60\,\kmps$ increments.
The upper limit on $\vk$ is chosen to match the escape velocity of the centre of the merger remnant, $v_\mathrm{esc}$.
We additionally run one simulation with a recoil kick above $v_\mathrm{esc}$, with $\vk=2000\,\kmps$.
As discussed in \drafting{section}, higher recoil velocities typically result in larger excursions of the remnant SMBH from the isophotal centre.
To provide a fair comparison between simulations, we choose to analyse each child simulation at a fixed time after the recoil kick, namely $t\simeq 0.075\,\mathrm{Gyr}$.
This value is chosen so as to minimise the effects of relaxation in the central regions of the galaxy remnant.
We estimate the relaxation time $t_\mathrm{relax}$ at a radius $r$ as \citep{binney2008}:
\begin{equation}\label{eq:relax}
    t_\mathrm{relax} \simeq 2.1 \frac{\sigma r^2}{G \bar{m} \ln \Lambda},
\end{equation}
where $\sigma$ is the particle velocity dispersion within $r$, $\bar{m}$ is the mean particle mass (hence the mean of the stellar and DM particle masses), and $\Lambda$ is the argument of the Coulomb logarithm, given by:
\begin{equation}
    \Lambda = \frac{r \langle v^2 \rangle}{2 G \bar{m}},
\end{equation}
where $\langle v^2 \rangle$ is the mean squared particle velocity within $r$. 
The relaxation time increases quadratically with increasing $r$, and we find for a radius of $r=0.1\,\mathrm{kpc}$ the relaxation time to be $t_\mathrm{relax} \sim 0.1 \, \mathrm{Gyr}$, thus motivating our choice of $0.075\,\mathrm{Gyr}$ for the analysis time. 
We  note that \autoref{eq:relax} does not take into account gravitational softening, and thus provides a conservative lower bound for the relaxation time.
This is critical as the \ketju{} region does not include gravitational softening between stellar particles and the SMBH, so is a not fully-softened simulation.

For all simulations, interactions between stellar particles are softened with a softening length of $\varepsilon = 2.5\,\mathrm{pc}$, DM with a softening length of $\varepsilon = 300\,\mathrm{pc}$, and the \ketju{} region radius is set to $r_{\rm ketju}=3\varepsilon = 7.5\,\mathrm{pc}$.


% RESULTS
\section{Results}\label{sec:results}

\drafting{
    Main points to hammer home:
    \begin{enumerate}
        \item Core radius, break density, sersic index have a dependence on the kick velocity, whereas other parameters (e.g. effective radius) do not.
        \item Increasing kick velocity induces a decrease in the overall $\langle h_4 \rangle$ profile
        \item Explanation of h4 trends through orbit decompostion.
    \end{enumerate}
}

\subsection{Infall time}
\drafting{
    \textbf{Atte's section}. Would be good to have:
    \begin{enumerate}
        \item Plot of infall time (i.e., time for stability criterion to be reached) as a function of kick velocity
        \item Number of oscillations before settling?
        \item Plot of maximum displacement of kicked BH as a function of kick velocity. How does this compare to core size? Which simulations have BH kicked beyond the core?
    \end{enumerate}
}
\drafting{
    Include this text here, this is how we identify `settled' systems, but we're not really interested in when the SMBH has settled for the rest of the paper (refer to Nasim where the majority of the GW-induced scouring occurs during the first excursion of the SMBH).
    Each child simulation is run until the median remnant SMBH velocity relative to the galaxy CoM \drafting{check abbreviation} velocity is less than $10\,\kmps$ for at least $0.1\,\mathrm{Gyr}$.
    The time to this point, from the time of SMBH binary coalescence, we term the `infall time'.
    Those simulations where the maximum displacement of the kicked SMBH exceeds $30\,\mathrm{kpc}$ are not included in the analysis, as we expect that observationally a SMBH further than this distance from the centre of its host galaxy would not be identified with that galaxy.
}
\begin{figure}
    \centering
    \includegraphics[width=0.4\textwidth]{infall_time}
    \caption{Plot of infall time (time to stability criterion) as a function of kick velocity.}
    \label{fig:infall}
\end{figure}
\begin{figure}
    \centering
    \includegraphics[width=0.4\textwidth]{max_disp}
    \caption{Plot of the maximum BH displacement as a function of kick velocity. Where do the displacement (red) and core size (green) lines intersect?}
    \label{fig:max_disp}
\end{figure}


\subsection{Mass density profiles in three dimensions}
Before considering the projected, 2D stellar mass density profiles of the merger remnants, we investigate the 3D stellar mass distribution.
This alleviates the influence of projection effects on inferring if a core has formed during the BH binary merger process, and gives insight into the dependence of how matter is spatially distributed as a function of the recoiling SMBH kick velocity.
We observe a clear dichotomy in the 3D spatial distribution of stellar mass, with mergers that had an SMBH with a kick velocity $\vk<540\,\kmps$ having more mass in the central $2\,\mathrm{kpc}$ than those mergers with $\vk>540\,\kmps$. 
All merger remnants have a consistent mass density distribution at $r>2\,\mathrm{kpc}$. 
The reasons for the dichotomy are discussed further in \drafting{section}, however we continue our analysis using the projected stellar density (as an observer would see) with the knowledge that there is \textit{a} dependence of the removal of central stellar mass on the SMBH kick velocity. 

\drafting{INCLUDE FIGURE HERE}



\subsection{Projected mass density profiles}
To facilitate comparisons with observations \drafting{some references}, we fit the six-parameter core-S\'ersic profile \citep{graham2003} to the projection of each simulated merger remnants:
\begin{equation}\label{eq:cs}
    \Sigma(R) = \Sigma' \left[1 + \left(\frac{\rb}{R}\right)^\alpha \right]^{\gamma/\alpha} \exp\left[-b \left(\frac{R^\alpha + \rb^\alpha}{\Reff^\alpha}\right)^{(1/\alpha n)}\right]
\end{equation}
where
\begin{equation}
    \Sigma' = \Sigma_\mathrm{b} 2^{-\gamma/\alpha} \exp\left[b\left(2^{1/a} \frac{\rb}{\Reff}\right)^{1/n}\right].
\end{equation}
Here $\rb$ is the core radius, $\Sigma_\mathrm{b}$ is the density at the core radius, $\gamma$ is the core slope, $\Reff$ is the effective (half-light) radius\footnote{As we assume a constant mass-to-light ratio, the half-light radius is equivalent to the projected half-mass radius.}, $n$ is the S\'ersic index, $b$ is estimated as $b\simeq 2n - 1/3 + 0.009876/n$ \citep{prugniel1997}, and $\alpha$ is the profile transition index.
We refer to the collective vector of these parameters as $\vb{\theta}$.
We view the simulated merger remnant from fifteen random angles, and use a Bayesian hierarchical model (HM) to fit the model parameters.
As each of the projections are unordered and exchangeable (i.e., there is no distinguishing label for the projections), and each projection is of the same merger remnant, a HM naturally lends itself.
The idea behind the HM is as such: for each projection, we infer the six parameters of $\vb{\theta}$, and assume that these values are draws from a common, global distribution of all possible $\vb{\theta}$ vectors.
Hence, we are effectively inferring the global hyperparameters $\vb{\theta}^\mathrm{hyp}$ that describe the latent parameters $\vb{\theta}$ that we are interested in.
This idea is demonstrated in the directed acyclic graph (DAG) for the model, shown in \autoref{fig:dag}.

We use weakly-constrained prior distributions for each of the hyperparameters in $\vb{\theta}^\mathrm{hyp}$, where these prior distributions are chosen to reflect reasonable values that the parameters may take (e.g. distributions of a radial quantity are chosen so as to be positively-constrained).
The distributions\footnote{Note that the normal distributions $\mathcal{N}$ are parameterised using the standard deviation $\sigma$, as opposed to the variance $\sigma^2$ as is common in many statistics references.} of the hyperparameters are given in \autoref{tab:hyper}.

We fit the model with \textsc{Stan} using the No U-turn sampler (NUTS, a variant of Hamiltonian Monte Carlo (HMC) with four chains each of 4000 iterations, of which the first 2000 are discarded as warmup.
This resulted in 2000 draws from the posterior distribution of $\vb{\theta}$, whereby we ensure the diagnostic value $\hat{R}$ (comparison of between-chain and within-chain estimates) is less than 1.05.
We perform additional prior sensitivity tests using the power scaling method described in \citet{kallioinen21}, ensuring that the cumulative Jensen-Shannon (CJS) divergence is less than 0.05 for the latent parameters $\vb{\theta}$ of the model. 
This recommended threshold indicates that the sampled posterior distributions of the latent parameters are not sensitive to the particular prior distributions used.

In agreement with previous studies \citep{nasim2021b}, we find an increase in the core radius $\rb$ with kick velocity.
\drafting{Need to check if others discuss this:} We find that the core radius saturates to a value of $\sim2$ times that of the merger remnant with no kick velocity by $\vk = 600\,\kmps$, corresponding to $0.3 v_\mathrm{esc}$. 
As shown in \autoref{fig:csparams}, the density at the core radius $\Sigma_\mathrm{b}$ decreases monotonically with kick velocity, consistent with the picture of the kicked SMBH removing more mass the more number of oscillations it is able to undergo before settling to Brownian motion limits. 
Interestingly, we observe a similar decrease in S\'ersic index $n$ with increasing kick velocity, although no correlation is seen between $n$ and $\Sigma_\mathrm{b}$ in the latent parameter corner plots (see for example \autoref{fig:csparams}).

\subsection{Predicted core size distribution}
Following \citet{nasim2021b}, we fit a power law of the form
\begin{equation}\label{eq:vkrb}
    \frac{\rb}{r_{\mathrm{b},0}} = K \left( \frac{\vk}{v_\mathrm{esc}} \right)^\beta + 1
\end{equation}
to our simulations, which is thus valid for $0\leq\vk/\kmps\leq900$, or equivalently $0\leq\vk/v_\mathrm{esc}\leq 0.5$.
We estimate the parameters $K$ and $\beta$ using HMC, and find median values of $K=3.70$ and $\beta=0.562$.

Using \autoref{eq:vkrb}, we push through the distribution of kick velocities from \citet{zlochower2015} for dry mergers assuming random azimuthal spin alignment (their Fig. 15, right panel).
The dimensionless spin parameter $\alpha_\bullet$ for the dry mergers follows a beta distribution, namely:
\begin{equation}
    P_\mathrm{Zlochower}(\alpha_\bullet) \propto (1-\alpha_\bullet)^{4.66884-1} \alpha_\bullet^{10.5868-1}.
\end{equation}
The range of kick velocities predicted by the model varies from $\vk=0\,\kmps$ to $\vk\sim4000\,\kmps$ \drafting{check this}.
Using transformation sampling allows us to obtain a distribution of core radii predicted by a given kick velocity model, as shown in \autoref{fig:rb_pdf}.
We find that the mode of the forward-folded core radius distribution is $1.64\,\mathrm{kpc}$ ($3.12\,r_{\mathrm{b},0}$).
Thus, assuming the \citet{zlochower2015} model to be a reasonable description of SMBH recoil velocities, most massive elliptical galaxies should have a non-negligible core \drafting{add references}. 


\begin{table}
    \caption{Hyperparameter distributions for core-S\'ersic model. $\mathcal{N}$ indicates a normal distribution, $\mathrm{Exp}(\lambda)$ indicates an exponential distribution.}
    \label{tab:hyper}
    \begin{tabular}{llc}
        \hline
        Hyperparameter & Distribution & Truncation \\
        \hline
        $\mu_{\log_{10}\Sigma_\mathrm{b}}$ & $\mathcal{N}(10, 2)$ & None \\
        $\sigma_{\log_{10}\Sigma_\mathrm{b}}$ & $\mathcal{N}(0, 1)$ & $x>0$ \\
        $\lambda_\gamma$ & $\mathrm{Exp}(10)$ & None \\
        $\mu_n$ & $\mathcal{N}(8,4)$ & $0 < x \leq 15$ \\
        $\sigma_n$ & $\mathcal{N}(0,4)$ & $x>0$ \\
        $\sigma_\alpha$ & $\mathrm{Gamma}(2, 0.2)$ & None \\
        $\sigma_{r_\mathrm{b}}$ & $\mathcal{N}(0, 1)$ & $x>0$ \\
        $\sigma_{R_\mathrm{e}}$ & $\mathcal{N}(0, 12)$ & $x>0$ \\
        $\mu_{\tau_{\log_{10}\Sigma}}$ & $\mathcal{N}(0, 1)$ & $x>0$ \\
        $\sigma_{\tau_{\log_{10}\Sigma}}$ & $\mathcal{N}(0, 0.2)$ & $x>0$ \\
        \hline
    \end{tabular}
\end{table}


\begin{figure}
    \centering
    \includegraphics[width=0.4\textwidth]{dag}
    \caption{Directed acyclic graph of the core-S\'ersic model of projected mass density within the Bayesian hierarchical framework. Single-line circle nodes represent fit parameters, double-line circles represent measured quantities (the data), and diamond nodes represent deterministic quantities. The particular distribution connecting nodes is written below the corresponding black square, with a subscript `$\mathrm{T}[l,u]$' indicating a distribution $f(\lambda)$ truncated to $l<\lambda<u$, and the subscript `likelihood' indicating the likelihood function. The $R$ box indicates variables fit for each radial bin, and the projection box indicates variables specific to each projection realisation. Note that we additionally use $\hat{\Sigma}$ to distinguish the calculated value of surface density from the measured value $\Sigma$. The various distributions are normal ($\mathcal{N}$), exponential ($\mathrm{Exp}$), and Rayleigh ($\mathrm{Ray}$).}
    \label{fig:dag}
\end{figure}

\begin{figure}
    \centering
    \includegraphics[width=0.4\textwidth]{density}
    \caption{
        Surface density profiles with 25\% Bayesian HDI for select representative runs.
        Increasing the kick velocity induces a shallower density profile in the inner regions.
        }
    \label{fig:density}
\end{figure}

\begin{figure}
    \centering
    \includegraphics[width=0.4\textwidth]{rb-kick}
    \caption{
        Bayesian estimate of the merger remnant core size $r_b$, scaled to the core size of the pre-merger remnant $r_{b,0}$, as a function of kick velocity $\vk$.
        The core size distributions are shown as box plots, with the median core size indicated by the central mark.
        Larger kick velocities are correlated with larger core sizes. Additionally, a greater spread in the distribution of core sizes over different viewing projections of the merger remnant is associated with larger kick velocities.}
    \label{fig:vkrb}
\end{figure}

\begin{figure}
    \centering
    \includegraphics[width=0.4\textwidth]{rb_pdf}
    \caption{Probability density function of transformation sampled core radius. The SMBH kick velocity is Monte Carlo sampled from the \citet{zlochower2015} relation assuming randomly-aligned azimuthal spins, and pushed through the fitted model in \autoref{eq:vkrb}. The predicted kick velocities range from $\vk=0\,\kmps$ to $\vk\sim4000\,\kmps$. \drafting{check this}}
    \label{fig:rb_pdf}
\end{figure}


\subsection{Integral field unit kinematics}
We create mock integral field unit (IFU) observations using two different projections for the line of sight (LOS) velocity distribution: one where the LOS is along the BH kick axis (`parallel'), and another where the LOS is along an axis orthogonal to the BH kick axis (`orthogonal').
Testing these two special projections, we investigate the limiting cases of how the galaxy kinematics behave when all or none of the BH motion is directed along the LOS to the observer.

We create our mock IFU observations following \citet{naab2014}.
Our observations are centred on the stellar centre of mass using the shrinking sphere method, and encompass a circular aperture of $0.25R_{1/2}$, where $R_{1/2}$ is the three-dimensional half mass radius. 
For each particle within this aperture, we generate 25 pseudo-particles with identical LOS velocities as the original particle, but are spatially displaced from the original particle with a Gaussian distribution in the projected $x$ and $y$ direction with a standard deviation of $0.3\,\mathrm{kpc}$, to mimic seeing effects.
The pseudo-particles are then binned onto a spatial grid, where the grid is assumed to have a resolution of $0.2''$, similar to the observations of \citet{neureiter2023}, which then corresponds to a physical resolution\footnote{https://cosmocalc.icrar.org} of $\sim0.04\,\mathrm{kpc}$ at redshift $z=0.01$. 
Following the method in \citet{cappellari2003}, we then group adjacent bins into larger Voronoi bins to achieve a particle number of $\sim 50000$ particles per Voronoi bin.

For each Voronoi bin, we follow \citet{vandermarel1993} and decompose the LOS velocity into a series of Gauss-Hermite functions, described by the mean radial velocity $V$, the velocity dispersion $\sigma$, asymmetric deviations $h_3$, and symmetric deviations $h_4$.
The line profile is thus described by:
\begin{equation}
    \mathfrak{L} = \frac{1}{\sqrt{2\pi}\sigma} e^{-w^2/2} \left\{ 1 + \sum_{j=3}^4 h_j H_j(w) \right\},
\end{equation}
where $w \equiv (v_\mathrm{LOS} - V)/\sigma$, and the Hermite polynomials $H_3$ and $H_4$ are defined:
\begin{align}
    H_3 &= \frac{1}{\sqrt{3}} (2w^3 - 3w) \nonumber \\
    H_4 &= \frac{1}{\sqrt{24}} (4w^4 - 12w^2 + 3).
\end{align}
A recent comprehensive study of LOS velocity distribution fitting applied to elliptical galaxies can be found in \citet{mehrgan2023}.

We observe a trend of the Voronoi bins with $h_4<0$ extending to larger radii for remnants with larger recoil kicks than for those remnants with smaller recoil kicks.
To quantify this recoil velocity dependence, we calculate the cumulative $h_4$ parameter:
\begin{equation}
    \langle h_4 \rangle = \sum_i^N h_4(R_i)
\end{equation}
We observe an overall shift to more negative values of $h_4$ for kick velocities up until $\vk\sim400\,\kmps$ \drafting{check this, currently just read off plot}, above which the radial $h_4$ profile is not sensitive to increasing kick velocity. 
Additionally, all $h_4$ profiles are radially decreasing, obtaining a local minimum at $R\simeq 2\,\mathrm{kpc}$ ($R/\Reff\simeq 0.2$), and a modestly increasing for $R \gtrsim 2\,\mathrm{kpc}$.

\drafting{Jens: maybe you have some thoughts\dots Why is this 
\begin{enumerate}
    \item saturating beyond some point,
    \item displaying a dip at 0.2 Re??
\end{enumerate}
}


\begin{figure*}
    \centering
    \includegraphics[width=\textwidth]{IFU_v0000}
    \caption{
        Mock integral field unit kinematic maps of the $\vk=0\,\kmps$ remnant, plotted out to $0.5\,\Reff$. 
        Note the small magnitude of the unordered rotation $V$, and the centrally-concentrated velocity dispersion $\sigma$.
        The majority of the merger remnant within this aperture has $h_4>0$, with a small concentration of $h_4<0$ in an annulus about the central regions.
    }
    \label{fig:IFU0}
\end{figure*}

\begin{figure*}
    \centering
    \includegraphics[width=\textwidth]{IFU_v0900}
    \caption{
        Mock integral field unit kinematic maps of the $\vk=900\,\kmps$ remnant, again plotted out to $0.5\,\Reff$, with the same colour scaling as \autoref{fig:IFU0}. 
        Contrasting to the $\vk=0\,\kmps$ case in \autoref{fig:IFU0}, this IFU map has a much greater extent of $h_4<0$ that also reaches more extreme values of $h_4$. 
    }
    \label{fig:IFU900}
\end{figure*}

\begin{figure}
    \centering
    \includegraphics[width=0.4\textwidth]{h4}
    \caption{
        Plot of the flux-weighted value of $h_4$ as a function of mean bin distance $R$for each merger remnant simulated.
        The lines are coloured by the kick velocity. 
        Immediately apparent is the decrease in $\langle h_4 \rangle$ with increasing $\vk$.
        All merger remnants have a global minimum in $\langle h_4 \rangle$ at $R\simeq 2.0$, corresponding to $R/\Reff\simeq0.2$.
    }
    \label{fig:h4}
\end{figure}


\subsection{Orbit analysis}

\begin{figure*}
    \centering
    \includegraphics[width=\textwidth]{orbits}
    \caption{
        Orbit analysis of all sixteen merger remnants.
        Stellar particles are assigned to one of seven different orbital families, and binned into ten logarithmically-spaced radial shells such that $0.2 \leq R/\mathrm{kpc} \leq 30.0$.
        At radii greater than $1\,\mathrm{kpc}$, there is very little to distinguish between the different kick velocities for boxlet, $z$-tube, Keplerian, irregular, and unclassified orbits.
        Conversely, for radii $1 \leq R/\mathrm{kpc} \leq 10$, there is a clear gradient for $\pi$-box and $x$-tube families.
        For higher kick velocities, there is a tendency to a higher fraction of $\pi$-box orbits compared to lower kick velocities.
        An inverse trend is visible for $x$-tube orbits. 
    }
    \label{fig:orbits}
\end{figure*}

To understand the kinematic maps, we perform an orbit analysis of the merger remnants in our sample following \citet{frigo2021}, and is briefly described here.

\drafting{Discuss how the orbit analysis is done, in particular the different orbital families}.
First, the merger remnant is rotated so that the $z$-axis coincides with the minor axis of the reduced inertia tensor as measured using the top 50\% most bound stellar particles to the stellar centre of mass.
We cannot centre on the recoiling SMBH for this analysis, as for high kick velocities the SMBH is significantly displaced from the centre.
The stellar potential of the merger remnant is fit using a self consistent field (SCF) potential \citep{hernquist1992}, using the \citet{hernquist1990} profile as a zeroth-order basis and limiting the expansion to $n_\mathrm{max}=18$ and $l_\mathrm{max}=7$ following \citet{frigo2021}.
The potential from the SMBH is then added as a point mass potential to the SCF potential.
Collectively, this represents the potential in an analytical form in which the orbits of individual stellar particles can be integrated.
The potential of the merger remnant is checked for stability by comparing the potential from the SCF method to the potential computed from the particle data, and ensuring the ratio of the two is $\sim1$ at all radii.
Each stellar particle within $30\,\mathrm{kpc}$ of the centre is integrated for fifty orbits to determine which, if any, orbital resonances exist. 
The orbital resonances define the different families of orbits, as given in \citet{frigo2021}.

\begin{table}
    \caption{Orbital families}
    \label{tab:orbits}
    \begin{tabular*}{0.48\textwidth}{p{0.1\textwidth}p{0.35\textwidth}}
        \hline
        Family & Description \\
        \hline
        $x$-tube & Rotate about the major axis of the galaxy \\
        $z$-tube & Rotate about the minor axis of the galaxy \\
        $\pi$-box & Non-resonant motion with no net angular momentum (radial motion) \\
        boxlet & Resonant motion with angular momentum \\
        Keplerian & Typical orbit in a point-mass dominated spherically-symmetric potential \\
        irregular & No integrals of motion \\
        unclassified & Orbits unable to be classified \\
        \hline
    \end{tabular*}
\end{table}

As seen in \autoref{fig:orbits}, we observe a clear trend of higher kick velocity remnants having a higher fraction of $\pi$-box and boxlet orbits, and a lower fraction of $x$-tube orbits, and to some extent Keplerian orbits, than their lower kick velocity counterparts.
The other two orbital families are generally consistent between all merger remnants. 
As discussed in \citet{frigo2021}, negative values of $h_4$ indicate regions with weaker tails of the LOS velocity distribution due to the superposition of radial orbits with different orientations.
This explanation supports our observation of an increase in $\pi$-box orbits (which have no net angular momentum are typically very radial orbits) with kick velocity, and a corresponding decrease in the value of $h_4$ (comparing the IFU maps in \autoref{fig:IFU0} to \autoref{fig:IFU900}).
Equivalently, $x$-tube orbits (which rotate about the major axis of the inertia tensor) appear as a peaked LOS velocity distribution in the projection we use, leading to extended tails, and hence a more positive value of $h_4$ than for a weak tail distribution. 
All together, we can infer that as a SMBH ploughs through the surrounding stellar environment, regular $x$-tube orbits are disrupted to non-resonant $\pi$-box orbits.
With a higher kick velocity, the SMBH is able to cumulatively disturb orbits at a larger radial extent, converting a greater fraction of $x$-tube orbits to $\pi$-box orbits, resulting in the increased expanses of negative $h_4$ regions in the kinematic maps.
This also explains the overall downward shift in the $\langle h_4 \rangle$ profiles in \autoref{fig:h4} for higher kick velocities. 
A peculiar feature of \autoref{fig:h4} is the consistency in the $\langle h_4 \rangle$ profile shape.
This can be explained by the derivative of the orbital fraction distribution being consistent across all kick velocities for $x$-tube and boxlet orbits out to $\sim3\,\mathrm{kpc}$, as shown in \autoref{fig:orbits2}. 
\drafting{Thorsten: I would really appreciate your thoughts on this section, and its relation to the kinematic maps. }


% DISCUSSION
\section{Discussion}
\label{sec:discussion}



% CONCLUSION
\section{Conclusions}
\label{sec:conclusions}

% FINAL BITS
\section*{acknowledgments}
A.R. acknowledges the support by the University of Helsinki Research Foundation.
A.R., acknowledge the support
by the European Research Council via ERC Consolidator Grant KETJU (no. 818930) and the support of the Academy of Finland grant 339127.

The numerical simulations used computational resources provided by
the CSC -- IT centre for Science, Finland.

\section*{Author contributions}
We list here the roles and contributions of the authors according to the Contributor Roles Taxonomy (\href{https://credit.niso.org}{CRediT}). 
\textbf{AR}:

\section*{Software}
\ketju{} \citep{mannerkoski2023,rantala2017},
\gadget{} \citep{springel2021},
NumPy \citep{harris2020},
SciPy \citep{virtanen2020},
Matplotlib \citep{hunter2007},
pygad \citep{rottgers2020},
\textsc{Stan} \citep{standevelopmentteam2018},
CmdStanPy \citep{standevelopmentteam2018},
Arviz \citep{kumar2019}.


%%%%%%%%%%%%%%%%%%%%%%%%%%%%%%%%%%%%%%%%%%%%%%%%%%
\section*{Data Availability}
The data underlying this article will be shared on reasonable request to the corresponding author.




%%%%%%%%%%%%%%%%%%%% REFERENCES %%%%%%%%%%%%%%%%%%

% The best way to enter references is to use BibTeX:

\bibliographystyle{mnras}
\bibliography{ref} % if your bibtex file is called example.bib


% Alternatively you could enter them by hand, like this:
% This method is tedious and prone to error if you have lots of references
%\begin{thebibliography}{99}
%\bibitem[\protect\citeauthoryear{Author}{2012}]{Author2012}
%Author A.~N., 2013, Journal of Improbable Astronomy, 1, 1
%\bibitem[\protect\citeauthoryear{Others}{2013}]{Others2013}
%Others S., 2012, Journal of Interesting Stuff, 17, 198
%\end{thebibliography}

%%%%%%%%%%%%%%%%%%%%%%%%%%%%%%%%%%%%%%%%%%%%%%%%%%

%%%%%%%%%%%%%%%%% APPENDICES %%%%%%%%%%%%%%%%%%%%%

\appendix
\section{Triaxiality}\label{sec:app_triax}
We present the triaxiality of all sixteen merger remnants in the analysis at the time when the SMBH has settled.
We determine the ratios $b/a$ and $c/a$, where $a$, $b$, and $c$ are the eigenvalues of the reduced inertia tensor $I_\mathrm{red}$ and $c\leq b\leq a$. 
The tensor $I_\mathrm{red}$ is determined by binning the stellar component of the remnant into 20 radial shells from $10^{-3} R_\mathrm{vir}$ to $10^{-1} R_\mathrm{vir}$: i.e., the binning of particles is not cumulative. 
The variation of the ratios $b/a$ and $c/a$ is plotted as a function of radius in \autoref{fig:triax}.
All remnants display a high degree of symmetry, with the minimum value of $b/a\simeq 0.8$, and the minimum value of $c/a\simeq0.7$.
With a higher kick velocity, there is a slight tendency to have a higher degree of symmetry at radii $R\lesssim 10\,\mathrm{kpc}$, and a slightly lesser degree of symmetry at radii beyond this distance, compared to lower kick velocity remnants. 
The difference is however minimal between all merger remnants. 


\begin{figure}
    \centering
    \includegraphics[width=0.5\textwidth]{triaxiality}
    \caption{
        Triaxiality ratios $b/a$ and $c/a$ as a function of radius for all merger remnants in the analysis.
        Lines are colour coded by the kick velocity.
        Note that the $\vk=0\,\kmps$ case is equivalent to the pre-merger remnant. 
    }
    \label{fig:triax}
\end{figure}

\section{Core-S\'ersic fit parameter estimates}\label{sec:app_fit}

\begin{figure*}
    \centering
    \includegraphics[width=\textwidth]{all-kick}
    \caption{
        Box plots of the six parameters in the core-S\'ersic model (from top left to bottom right) as a function of the kick velocity: the core radius normalised to the pre-merger core radius $\rb/r_{\mathrm{b},0}$, the density at the core radius $\log_{10}\Sigma_\mathrm{b}$, the core slope $\gamma$, the effective radius $\Reff$, the profile transition index $\alpha$, and the S\'ersic index $n$.
    }
    \label{fig:csparams}
\end{figure*}

\begin{figure*}
    \centering
    \includegraphics[width=\textwidth]{corner_latent_900}
    \caption{
        Corner plot of the latent parameters in the core-S\'ersic model for the $\vk=900\,\kmps$ instance.
        Contours indicate, from yellow to dark blue, the 25\%, 50\%, 75\%, and 99\% highest density intervals (HDIs), with sample draws beyond the 99\% HDI shown as individual points. 
        All distributions are unimodal with little cross-correlation between the latent variables.
    }
\end{figure*}


\section{Additional orbital plots}

\begin{figure*}
    \centering
    \includegraphics[width=\textwidth]{orbits2}
    \caption{
        Same as \autoref{fig:orbits}, but each merger remnant is shown in its own panel, with the different orbital families depicted by colour, to better demonstrate the radial fraction of orbits for a given merger.
    }
    \label{fig:orbits2}
\end{figure*}

%%%%%%%%%%%%%%%%%%%%%%%%%%%%%%%%%%%%%%%%%%%%%%%%%%


% Don't change these lines
\bsp	% typesetting comment
\label{lastpage}
\end{document}

% End of mnras_template.tex
