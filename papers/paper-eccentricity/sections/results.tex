\section{Results}

\begin{itemize}
    \item Show that brownian motion affects impact parameter of the final passage before BHs bound --> eccentricity set
\end{itemize}

\drafting{Nasim result takes the mean of the eccentricity over some time period, and then determines the standard deviation across the means of different realisations of a merger system. We probably need to show this too for `consistency', however we could also introduce a simple hierarchical model here that allows us to do something similar more robustly.}

\subsection{Eccentricity scatter in low-e orbits}
\begin{itemize}
    \item Show that impact parameters not sensitive
\end{itemize}
\begin{figure*}
    \includegraphics{nasim_scatter_e0-900.pdf}
    \caption{\drafting{Show a plot like this where the scatter in eccentricity is determined as the standard deviation across simulations}}
    \label{fig:nasim_e90}
\end{figure*}

\subsection{Eccentricity scatter in high-e orbits}
\begin{itemize}
    \item show that impact parameters sensitive
\end{itemize}
\begin{figure*}
    \includegraphics{nasim_scatter_e0-990.pdf}
    \caption{\drafting{Show a plot like this where the scatter in eccentricity is determined as the standard deviation across simulations}}
    \label{fig:nasim_e99}
\end{figure*}
We show the dependence of eccentricity standard deviation on mass resolution for the $e_0=0.99$ orbit in \autoref{fig:nasim_e99}. 
We find that the scaling of $\sigma_e$ is better described by a relation proportional to $1/N_\star$, as opposed to the $1/\sqrt{N_\star}$ relation described in \citet{nasim20}.
We attribute this dramatic decrease in $\sigma_e$ to an inflated value of $\sigma_e$ at low mass resolution: there is increased scatter in eccentricity at low mass resolutions for very radial initial orbits of the SMBH binary. 



\begin{figure*}
    \includegraphics{orbit_pars}
    \caption{\drafting{Orbital parameters of the different merger systems, colour coded according to the initial orbital eccentricity}}
    \label{fig:orbitpars}
\end{figure*}

\begin{figure*}
    \includegraphics{hyperparameter}
    \caption{\drafting{Hyperparameters of a hierarchical model of the binary hardening from a time when the binary is hard. To show convergence, we could compare the the distributions of $\sigma_{e_h}$ directly between resolutions}}
    \label{fig:hyperparameter}
\end{figure*}

\begin{figure}
    \includegraphics{deflect_e-0900}
    \caption{\drafting{Theta-e plane for $e_0=0.90$ orbit}}
    \label{fig:deflect_e90}
\end{figure}

\begin{figure}
    \includegraphics{deflect_e-0990}
    \caption{\drafting{Theta-e plane for $e_0=0.99$ orbit}}
    \label{fig:deflect_e99}
\end{figure}


\subsection{Explaining the eccentricity scatter with a toy model}

We can gain insight into the large binary eccentricity scatter observed in the
runs with high galactic merger orbit eccentricities by considering a simplified
model that captures the essential features of how the SMBH binary becomes bound
and settles on an orbit with a given eccentricity.
The components we include in this model are the two SMBHs,
a smooth background potential, and dynamical friction.
For simplicity, we consider only symmetrical 1:1 mergers, and take the BH orbits
to lie in the $xz$-plane.
In this case the relative motion of the BHs can be treated as a one-body problem,
with the equation of motion
\begin{equation} \label{eq:toy_model_eom}
\ddot{\vb{x}} = -\frac{2 G M_\bullet}{|\vb{x}|^3} \vb{x} + \vb{a}_\mathrm{bg} + \vb{a}_\mathrm{DF},
\end{equation}
where $\vb{x}$ is the separation vector of the BHs, $M_\bullet$ the mass of a single BH,
$\vb{a}_\mathrm{bg}$ the acceleration due to the background potential,
and $\vb{a}_\mathrm{DF}$ the acceleration due to dynamical friction.

The background galaxy is modelled as a with constant density spheroidal potential,
causing the acceleration \drafting{(e.g.\ B\&T)}
\begin{equation}
\vb{a}_\mathrm{bg} = - 2 \pi G \rho \operatorname{diag}(A_x,A_y,A_z) \vb{x},
\end{equation}
where $\rho$ is the stellar density,
and the $A$ coefficients are related to the eccentricity $e_\mathrm{s}$ of the spheroid as
\begin{equation}
A_x = A_y  = 2 \frac{1-e_\mathrm{s}^2}{e_\mathrm{s}^2} 
    \left[\frac{1}{2 e_\mathrm{s}} \ln\left(\frac{1+e_\mathrm{s}}{1-e_\mathrm{s}}\right) - 1 \right]
\end{equation}
\begin{equation}
A_z = \frac{1-e_\mathrm{s}^2}{e_\mathrm{s}^2} 
        \left[ \frac{1}{1-e_\mathrm{s}^2} 
        - \frac{1}{2 e_\mathrm{s}} \ln\left(\frac{1+e_\mathrm{s}}{1-e_\mathrm{s}}\right)\right].
\end{equation}
With eccentricity values of $e_\mathrm{s} \approx 0.8\textnormal{--}0.9$,
this spheroidal potential matches fairly well with the central potential in the galaxy during the
BH binary binding phase, apart from the contributions from the BHs themselves and the groups of stars bound to them.
\drafting{(Figure showing this?)}

For dynamical friction, we use the \drafting{Chandrasekhar ref} formula assuming
a Maxwellian distribution with a constant velocity dispersion $\sigma$.
This formula gives the acceleration of a single BH due to dynamical friction as
\begin{equation}\label{eq:chandra_df}
\vb{a}_{1,\mathrm{DF}} = - 4 \pi G^2 M_\bullet \rho \ln{\Lambda} 
                        (\operatorname{erf}(X) - 2 \pi^{-1/2} X e^{-X^2}) \frac{\vb{v}_{1}}{|\vb{v}_{1}|^3},
\end{equation}
where $\vb{v}_1 = \dot{\vb{x}}/2$ is the velocity of a single BH,
$X = |\vb{v}_1|/\sqrt{2}\sigma$,
and the Coulomb logarithm is taken to have the value $\ln{\Lambda} = 10$.
To account for differences between the constant density background of stars
assumed here and the more complex system that is present in the simulations
we additionally multiply the acceleration given by Eq. \eqref{eq:chandra_df}
with a scaling and cut-off function
\begin{equation}
f(a) = \frac{\alpha}{1 + \exp[(a_\mathrm{c}-a)/d_\mathrm{c}]},
\end{equation}
which serves to remove the effect of dynamical friction when the BH binary has become
hard in addition to allowing the strength of dynamical friction to be scaled to
achieve similar behaviour as observed in the simulations.
Here $a$ is the semi-major axis of a bound binary, and for unbound binaries we set $f=\alpha$.
We set the scaling factor $\alpha=0.5$, and use cut-off scales $a_\mathrm{c}=2 a_\mathrm{h}$
and $d_\mathrm{c} = 0.5 a_\mathrm{h}$,
with $a_\mathrm{h}$ being the semimajor-axis of a hard BH binary.
The total dynamical friction term is then
\begin{equation}
\vb{a}_{\mathrm{DF}} = 2 f(a) \vb{a}_{1,\mathrm{DF}}.
\end{equation}

After specifying the initial position and velocity, as well as the other parameters of the model,
the equation of motion can be solved using standard numerical integration techniques.
We use the error controlled 8th order Runge--Kutta method DOP853 included in the SciPy library \citep{scipy}
for this purpose.
To mimic the nearly linearly plunging orbit in the high-eccentricity merger orbit simulations,
we specify the initial conditions confined to the $xz$-plane as 
\begin{gather}
\vb{x}_0 = (r_0, b)\\
\dot{\vb{x}}_0 = (-v_0, 0).
\end{gather}
We fix $r_0 = 100\,\mathrm{pc}$,
and use model parameters
$\rho = 400 \,\Msun\,\mathrm{pc}^{-3}$,
$\sigma=200\,\mathrm{km\,s^{-1}}$,
$M_\bullet=10^8\,\Msun$,
and
$e_\mathrm{s}=0.85$,
which match fairly well with the central properties of the merging galaxy when
the SMBHs are becoming bound in the $e=0.99$ merger orbit simulations.
\drafting{(Check the params)}


\drafting{
Figures to do:
\begin{itemize}
\item $b$-$e$ curve at a few different $v_0$ values, shows minima in $e$.
      Also plot data points from high-e simulations, we can hopefully match the curve shape at least roughly.
      Possibly also deflection angle plot.
\item Plot of the minimum possible eccentricity at different initial velocities, and the corresponding $b$, deflection angles.
\end{itemize}
}

\drafting{
Findings
\begin{itemize}
\item Final $e$ varies with impact parameter for given $v_0$, two minima on small scales at $b$s that are likely to occur in simulations. 
\item location and depth of minima varies with $v_0$, for some values possible to span the full range of $e$ for different impact params.
\item Minimum $e$ for deflection angles $\sim 60^\circ, 120^\circ$.
\end{itemize}
}

\drafting{
Interpretation:
\begin{itemize}
\item Realistic merger orbits generally result in linearly plunging orbits just before binary formation,
        with fairly small impact parameters.
\item Often the relative velocities are in the range that leads to deflections around $90^\circ$
      (some reason for this from energy considerations, since $b_{90}=GM_\mathrm{tot}/v^2$?),
      small changes in impact param around that point hit the eccentricity minima in this model.
\item Impact param differences seeded by the Brownian motion in sims,
      but few pc differences small enough that you can't measure them from large scale merger orbits anyway.
\item Conclusion: the scatter is physically relevant, not pure numerics.
\end{itemize}
}
