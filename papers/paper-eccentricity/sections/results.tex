\section{Results}

\subsection{Eccentricity scatter in a toy model}
\begin{itemize}
    \item Show that brownian motion affects impact parameter of the final passage before BHs bound --> eccentricity set
\end{itemize}

\drafting{Nasim result takes the mean of the eccentricity over some time period, and then determines the standard deviation across the means of different realisations of a merger system. We probably need to show this too for `consistency', however we could also introduce a simple hierarchical model here that allows us to do something similar more robustly.}

\subsection{Eccentricity scatter in low-e orbits}
\begin{itemize}
    \item Show that impact parameters not sensitive
\end{itemize}

\subsection{Eccentricity scatter in high-e orbits}
\begin{itemize}
    \item show that impact parameters sensitive
\end{itemize}


\begin{figure*}
    \includegraphics{orbit_pars}
    \caption{\drafting{Orbital parameters of the different merger systems, colour coded according to the initial orbital eccentricity}}
    \label{fig:orbitpars}
\end{figure*}

\begin{figure*}
    \includegraphics{nasim_style_plot}
    \caption{\drafting{Show a plot like this where the scatter in eccentricity is determined as the standard deviation across simulations}}
    \label{fig:nasim}
\end{figure*}

\begin{figure*}
    \includegraphics{hyperparameter}
    \caption{\drafting{Hyperparameters of a simple hierarchical model of the binary angular momentum and energy at time when binary is hard. To show convergence, we could compare the the distributions of $\sigma_{e_h}$ directly between resolutions}}
    \label{fig:hyperparameter}
\end{figure*}