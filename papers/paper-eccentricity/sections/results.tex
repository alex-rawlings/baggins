\section{Results}

\begin{itemize}
    \item Show that brownian motion affects impact parameter of the final passage before BHs bound --> eccentricity set
\end{itemize}

\begin{figure}
    \includegraphics{convergence}
    \caption{\drafting{Convergence of eccentricity scatter as a function of mass resolution for $e_0=0.90$ and $e_0=0.99$. }}
    \label{fig:convergence}
\end{figure}

\subsection{Eccentricity scatter in low-e orbits}
\begin{itemize}
    \item Show that impact parameters not sensitive
\end{itemize}
We show the dependence of eccentricity standard deviation on mass resolution for the $e_0=0.90$ orbit in \autoref{fig:convergence} with purple markers. 
Taking the full set of ten simulations for this set (circle markers), immediately apparent is the lack of convergence of $\sigma_e$, with $5\times10^3$ and $1\times10^4$ mass resolutions demonstrating an increased scatter compared to the lowest mass resolution of $2.5\times10^3$.
The peculiar values of $\sigma_e$ arises due to the assumed Gaussian distribution of the eccentricity between simulations.
Inspecting the orbital parameters for all simulations that form the $e_0=0.90$ suite, it is found that for resolutions of $5\times10^3$ and $1\times10^4$ the eccentricity clusters around a value of $e_\mathrm{h}\simeq0.90$, except for one simulation at each resolution which has a value of $e_\mathrm{h}<0.4$.
This one outlier for each of the two resolutions artificially inflates the measured eccentricity variance, and by applying a $2.5\sigma$-cut in eccentricity we recover the $1/\sqrt{M_\bullet/m_\star}\propto 1/\sqrt{N_\star}$ scaling of $\sigma_e$ as in \citet{nasim20} (square markers), apart from the highest mass resolution set of $M_\bullet/m_\star=4\times10^4$.
However, the eccentricity values beyond $2.5\sigma$ from the mean are each a valid measurement from the simulation, and highlight the presence of stochasticity in eccentricity for these idealised simulation set-ups. 
The cause of these low eccentricity values is discussed in \autoref{ssec:toy}.


\subsection{Eccentricity scatter in high-e orbits}
\begin{itemize}
    \item show that impact parameters sensitive
\end{itemize}
We show the dependence of eccentricity standard deviation on mass resolution for the $e_0=0.99$ orbit in \autoref{fig:convergence} with blue markers, where the marker symbols are consistent with those for the $e_0=0.90$ suite. 
We find that the scaling of $\sigma_e$ is steeper than the $1/\sqrt{M_\bullet/m_\star}\propto1/\sqrt{N_\star}$ relation described in \citet{nasim20}.
We attribute this dramatic decrease in $\sigma_e$ to an inflated value of $\sigma_e$ at low mass resolution: there is increased scatter in eccentricity at low mass resolutions for very radial initial orbits of the SMBH binary. 
This observation is highlighted in \autoref{fig:nasim_e_comp}.
As the majority of major galaxy mergers in the real Universe are expected to be radial \citep[e.g.][]{khochfar06}, understanding if the origin of the scatter in eccentricity is a numerical or a physical effect is of paramount importance, and will have direct implications for the predictions of SMBH merger rates by ongoing and future GW detection missions. 

\begin{figure}
    \includegraphics{eccentricities.pdf}
    \caption{\drafting{Inverse modelling of eccentricity scatter as a function of initial eccentricity for $M_\bullet/m_\star=2\times10^4$. Combine as a panel into a the convergence figure?}}
    \label{fig:nasim_e_comp}
\end{figure}

\begin{figure}
    \includegraphics{deflect_e-0900}
    \caption{\drafting{Theta-e plane for $e_0=0.90$ orbit. Update this confetti colour scheme. This figure and the one for $e_0=0.99$ will be combined into a single plot overlaid on the tow model results}}
    \label{fig:deflect_e90}
\end{figure}

\begin{figure}
    \includegraphics{deflect_e-0990}
    \caption{\drafting{Theta-e plane for $e_0=0.99$ orbit}}
    \label{fig:deflect_e99}
\end{figure}


\subsection{Explaining the eccentricity scatter with a toy model}\label{ssec:toy}

We can gain insight into the large binary eccentricity scatter observed in the
runs with high galactic merger orbit eccentricities by considering a simplified
model that captures the essential features of how the SMBH binary becomes bound
and settles on an orbit with a given eccentricity.
The components we include in this model are the two SMBHs,
a smooth background potential, and dynamical friction.
For simplicity, we consider only symmetrical 1:1 mergers, and take the BH orbits
to lie in the $xz$-plane.
In this case the relative motion of the BHs can be treated as a one-body problem,
with the equation of motion
\begin{equation} \label{eq:toy_model_eom}
\ddot{\vb{x}} = -\frac{2 G M_\bullet}{|\vb{x}|^3} \vb{x} + \vb{a}_\mathrm{bg} + \vb{a}_\mathrm{DF},
\end{equation}
where $\vb{x}$ is the separation vector of the BHs, $M_\bullet$ the mass of a single BH,
$\vb{a}_\mathrm{bg}$ the acceleration due to the background potential,
and $\vb{a}_\mathrm{DF}$ the acceleration due to dynamical friction.

The background galaxy is modelled as a with constant density spheroidal potential,
causing the acceleration \drafting{(e.g.\ B\&T)}
\begin{equation}
\vb{a}_\mathrm{bg} = - 2 \pi G \rho \operatorname{diag}(A_x,A_y,A_z) \vb{x},
\end{equation}
where $\rho$ is the stellar density,
and the $A$ coefficients are related to the eccentricity $e_\mathrm{s}$ of the spheroid as
\begin{equation}
A_x = A_y  = 2 \frac{1-e_\mathrm{s}^2}{e_\mathrm{s}^2} 
    \left[\frac{1}{2 e_\mathrm{s}} \ln\left(\frac{1+e_\mathrm{s}}{1-e_\mathrm{s}}\right) - 1 \right]
\end{equation}
\begin{equation}
A_z = \frac{1-e_\mathrm{s}^2}{e_\mathrm{s}^2} 
        \left[ \frac{1}{1-e_\mathrm{s}^2} 
        - \frac{1}{2 e_\mathrm{s}} \ln\left(\frac{1+e_\mathrm{s}}{1-e_\mathrm{s}}\right)\right].
\end{equation}
With eccentricity values of $e_\mathrm{s} \approx 0.8\textnormal{--}0.9$,
this spheroidal potential matches fairly well with the central potential in the galaxy during the
BH binary binding phase, apart from the contributions from the BHs themselves and the groups of stars bound to them.
\drafting{(Figure showing this?)}

For dynamical friction, we use the \drafting{Chandrasekhar ref} formula assuming
a Maxwellian distribution with a constant velocity dispersion $\sigma$.
This formula gives the acceleration of a single BH due to dynamical friction as
\begin{equation}\label{eq:chandra_df}
\vb{a}_{1,\mathrm{DF}} = - 4 \pi G^2 M_\bullet \rho \ln{\Lambda} 
                        (\operatorname{erf}(X) - 2 \pi^{-1/2} X e^{-X^2}) \frac{\vb{v}_{1}}{|\vb{v}_{1}|^3},
\end{equation}
where $\vb{v}_1 = \dot{\vb{x}}/2$ is the velocity of a single BH,
$X = |\vb{v}_1|/\sqrt{2}\sigma$,
and the Coulomb logarithm is taken to have the value $\ln{\Lambda} = 10$.
To account for differences between the constant density background of stars
assumed here and the more complex system that is present in the simulations
we additionally multiply the acceleration given by Eq. \eqref{eq:chandra_df}
with a scaling and cut-off function
\begin{equation}
f(a) = \frac{\alpha}{1 + \exp[(a_\mathrm{c}-a)/d_\mathrm{c}]},
\end{equation}
which serves to remove the effect of dynamical friction when the BH binary has become
hard in addition to allowing the strength of dynamical friction to be scaled to
achieve similar behaviour as observed in the simulations.
Here $a$ is the semi-major axis of a bound binary, and for unbound binaries we set $f=\alpha$.
We set the scaling factor $\alpha=0.5$, and use cut-off scales $a_\mathrm{c}=2 a_\mathrm{h}$
and $d_\mathrm{c} = 0.5 a_\mathrm{h}$,
with $a_\mathrm{h}$ being the semimajor-axis of a hard BH binary.
The total dynamical friction term is then
\begin{equation}
\vb{a}_{\mathrm{DF}} = 2 f(a) \vb{a}_{1,\mathrm{DF}}.
\end{equation}

After specifying the initial position and velocity, as well as the other parameters of the model,
the equation of motion can be solved using standard numerical integration techniques.
We use the error controlled 8th order Runge--Kutta method DOP853 included in the SciPy library \citep{scipy}
for this purpose.
To mimic the nearly linearly plunging orbit in the high-eccentricity merger orbit simulations,
we specify the initial conditions confined to the $xz$-plane as 
\begin{gather}
\vb{x}_0 = (r_0, b)\\
\dot{\vb{x}}_0 = (-v_0, 0).
\end{gather}
We fix $r_0 = 100\,\mathrm{pc}$,
and use model parameters
$\rho = 400 \,\Msun\,\mathrm{pc}^{-3}$,
$\sigma=200\,\mathrm{km\,s^{-1}}$,
$M_\bullet=10^8\,\Msun$,
and
$e_\mathrm{s}=0.85$,
which match fairly well with the central properties of the merging galaxy when
the SMBHs are becoming bound in the $e_0=0.99$ merger orbit simulations.
\drafting{(Check the params)}

\drafting{This right place for this section?}
To investigate the effect of the deflection angle, and by extension the impact parameter, on the final SMBH binary eccentricity, we determine the deflection angle at each pericentre passage before the SMBHs become bound \citep{binney08}:
\begin{equation}
    \theta_\mathrm{defl} = 2 \arctan \left(\frac{GM}{L\sqrt{2E}} \right)
\end{equation}
where $M=M_{\bullet,1}+M_{\bullet,2}$, and $L$ is the magnitude of the SMBH binary angular momentum vector and $E$ the binary orbital energy at the time of the pericentre passage.
We set as the first `hard-scattering' angle the first angle that exceeds an orbit-dependent threshold: $\theta_{\mathrm{defl,min}}=90$ for $e_0=0.99$ and $\theta_{\mathrm{defl,min}}=45$ for $e_0=0.90$ \drafting{need to insert degrees sign, but command not working?}.
To quantify the eccentricity of the resulting SMBH binary, we determine when the binary becomes hard (\autoref{eq:ahard}), and calculate the median and interquartile range (IQR) over a maximum of ten orbits about the time when $t=t(a=a_\mathrm{h})$.
These points are shown in \drafting{FIGURE}, where the IQR is generally smaller than the symbol marker used in the plot. 

\begin{figure}
    \includegraphics{orbit}
    \caption{\drafting{Orbit of a SMBH in one of the $e_0=0.99$ mergers, which by symmetry of the equal mass system is the same as the second BH orbit reflected. Line coloured according to time. Point A shows a time just before the first hard scattering, and point B a time just after. The circle marker indicates the final position of the BH before the BHs become bound.}}
    \label{fig:orbit}
\end{figure}

\drafting{
Figures to do:
\begin{itemize}
\item $b$-$e$ curve at a few different $v_0$ values, shows minima in $e$.
      Also plot data points from high-e simulations, we can hopefully match the curve shape at least roughly.
      Possibly also deflection angle plot.
\item Plot of the minimum possible eccentricity at different initial velocities, and the corresponding $b$, deflection angles.
\end{itemize}
}

\drafting{
Findings
\begin{itemize}
\item Final $e$ varies with impact parameter for given $v_0$, two minima on small scales at $b$s that are likely to occur in simulations. 
\item location and depth of minima varies with $v_0$, for some values possible to span the full range of $e$ for different impact params.
\item Minimum $e$ for deflection angles $\sim 60^\circ, 120^\circ$.
\end{itemize}
}

\drafting{
Interpretation:
\begin{itemize}
\item Realistic merger orbits generally result in linearly plunging orbits just before binary formation,
        with fairly small impact parameters.
\item Often the relative velocities are in the range that leads to deflections around $90^\circ$
      (some reason for this from energy considerations, since $b_{90}=GM_\mathrm{tot}/v^2$?),
      small changes in impact param around that point hit the eccentricity minima in this model.
\item Impact param differences seeded by the Brownian motion in sims,
      but few pc differences small enough that you can't measure them from large scale merger orbits anyway.
\item Conclusion: the scatter is physically relevant, not pure numerics.
\end{itemize}
}
