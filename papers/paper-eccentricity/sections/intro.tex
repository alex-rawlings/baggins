\section{Introduction}

Supermassive black holes (SMBHs) are believed to reside at the centres of almost all massive galaxies. 
As galaxies dynamically interact, so too are their supermassive black holes expected to interact in a three-phase merger process.
The first phase, dynamical friction, acts to bring the SMBHs from kpc-scale separations to pc-scale separations, after which the SMBHs form a bound binary system, which can be characterised by an orbit with a semimajor axis $a$ and eccentricity $e$.
Following the binding of the SMBHs, the SMBH binary separation is reduced through sequential slingshot encounters with the surrounding stellar distribution, and undergoes two types of interactions: hierarchical interactions and chaotic interactions.
Hierarchical interactions occur when a perturbing third body is far from from the SMBH binary, $a_\mathrm{bin} < a_\mathrm{perturber}$, and the energy of the system does not change, however the angular momentum does.
Hierarchical interactions drive changes in the SMBH binary eccentricity. 
Conversely, chaotic interactions between the SMBH binary and a perturber remove orbital energy from the system when the semimajor axis of the perturber is comparable to the semimajor axis of the SMBH binary, $a_\mathrm{bin}\simeq a_\mathrm{perturber}$.
The angular momentum of the system also varies during the chaotic interactions.
Chaotic interactions drive the SMBH binary to sub-pc separation: a regime where gravitational wave (GW) emission becomes the dominant mechanism by which the SMBH binary can lose its remaining orbital energy and angular momentum, resulting in the coalescence of the two SMBHs into a single SMBH.

It is well known that the efficiency by which GW emission can drive a SMBH binary to coalescence is highly sensitive to the eccentricity of the binary system \citep{peters1964}, with the merger timescale of the SMBH binary being particularly sensitive. 
Additionally, the complex nature of SMBH coalescence in a galaxy merger setting necessitates the use of numerical techniques to understand, and to provide quantitative predictions for observational programmes such as pulsar timing arrays (PTAs) and the upcoming Laser Interferometer Space Antenna (LISA) mission.
The requirement to understand the faithfulness to which SMBH binary eccentricity is captured in simulations is thus a critical piece of the puzzle in constraining the SMBH merger rate that the GW observational community aims to detect. 

Previous work by \citet{nasim2020} have argued that scatter in SMBH binary eccentricity observed in simulations is an artefact of poor phase space discretisation in simulations, and that in the real Universe where SMBH masses are far greater than stellar masses, $M_\bullet \gg m_\star$, SMBH binary eccentricity is a reliably predictable quantity.
In this work, we find that SMBH binary eccentricity is reliably predictable only for particular initial conditions of the galaxy merger system, and does not in general hold.
Indeed, we find that for realistic galaxy merger orbits where the initial orbital eccentricity of the system is $e_0>0.99$, stochasticity in the final SMBH binary eccentricity is present. 
This paper is organised \drafting{as follows: X, X, X.}
\drafting{Add relevant references.}