\section{Numerical Methods}

\begin{itemize}
    \item describe toy model
    \item describe galaxy ICs: 0.5 Dehnen spheres
    \item describe merger orbit: low and high e
\end{itemize}

\subsection{Galaxy ICs}\label{ssec:gics}
We construct our galaxy models to match those of \citet{gualandris22}.
Each galaxy is represented as a stellar-only Dehnen sphere \citep{dehnen93} with shape parameter $\gamma=0.5$ and scale radius $a=186\,\mathrm{pc}$, where the Dehnen profile is given by:
\begin{equation}\label{eq:dehnen}
    \rho_\star(r) = \frac{(3-\gamma)M_\star}{4\pi} \frac{a}{r^\gamma (r+a)^{(4-\gamma)}}.
\end{equation}
The total stellar mass is $M_\star=10^{10}\,\Msun$, and at the centre of the model galaxy a SMBH of mass $M_\bullet=10^8\,\Msun$ is placed with zero initial velocity.
We test \drafting{X} different mass resolutions by sampling the stellar phase space distribution with varying number of stellar particles: \drafting{list resolutions here}.

\subsection{Merger ICs}
We construct isolated merger initial conditions by placing two galaxies on an elliptical orbit with varying values of eccentricity, at a fixed initial separation of $D=3.72\,\mathrm{kpc}$ and semimajor axis $a_0=2.79\,\mathrm{kpc}$.
We follow the same orbital configurations as \citet{gualandris22}, with three additional orbit. 
The chosen eccentricities $e$ are 0.5, 0.7, 0.9, 0.95, 0.97, and 0.99.
We ensure that the radial and tangential velocities of the initial merger setup are consistent with the values reported in \citet{gualandris22}.
An example of each orbital configuration is shown in \autoref{fig:orbits}.
For each orbital configuration, we run ten realisations, to account for Brownian motion effects of the SMBH binary caused by the discretised phase space. 
Each orbital geometry is simulated using our fiducial particle resolution of $M_\bullet/m_\star = 10^4$.
Additionally, we run the $e=0.9$ and $e=0.99$ orbits at varying mass resolutions, using the galaxy ICs at the mass resolutions described in \autoref{ssec:gics}.

\subsection{Assessing Eccentricity Uncertainty}
In our analysis, we wish to determine the population spread in eccentricity from a finite set of observations.
We assess the uncertainty in eccentricity using two methods: an inverse modelling approach and a forward modelling approach.

The first method follows \citet{nasim20}, and is an inverse problem. 
For each simulation in a given set, we determine the mean eccentricity $e_\mathrm{h}$ over five orbital periods centred on the orbit within which the SMBH binary has become hard.
The binary hardening radius $a_\mathrm{h}$ is defined as \citep[e.g.][]{merritt13}
\begin{equation*}
    a_\mathrm{h} = \frac{q}{1+q} \frac{r_\mathrm{h}}{4},
\end{equation*}
where $r_\mathrm{h}=r(m<2M_{\bullet,1})$ is the influence radius and $q = \{q \in \mathbb{R} | 0<q \leq 1\}$ is the mass ratio between the SMBHs.
Under the assumption that each calculated mean eccentricity is a sample of some common distribution, we may (assuming that the distribution is Gaussian) obtain an estimate of the standard deviation of this distribution by taking the standard deviation of the collection of sample eccentricity means. 

The second method is a forward modelling method, where we use Bayesian analysis to construct a simple hierarchical model. 
We construct the hierarchical model using \textsc{Stan}, which is an implementation of Hamiltonian Monte Carlo.
Our likelihood function is constructed using the analytical hardening relations \citep{quinlan96,sesana06}:
\begin{equation}\label{eq:quinlan_H}
    \dv{}{t}\left( \frac{1}{a} \right) = H \frac{G\rho}{\sigma},
\end{equation}
and
\begin{equation}\label{eq:quinlan_K}
    \dv{e}{t} = -K a^{-1} \dv{a}{t}.
\end{equation}
In these equations, $H$ and $K$ are dimensionless constants describing the evolution rate of the semimajor axis and eccentricity of the SMBH binary, respectively, and $\rho$ and $\sigma$ are the stellar density and velocity dispersion within the binary influence radius, respectively.
We wish to determine the uncertainty on the free parameters $H' = HG\rho/\sigma$ and $K$, in addition to the initial conditions for the differential equation system, $a_\mathrm{h}$ and $e_\mathrm{h}$, at a time $t_\mathrm{h}$. 
Let us denote the set of parameters $[H', K, a_\mathrm{h}, e_\mathrm{h}]$ as $\theta$.
To determine the uncertainty in the parameter set $\theta$, we solve the differential equation system analytically for a time period after the SMBH binary becomes hard where the effect of GW emission can be ignored: here taken to be $t_\mathrm{end} = t(a=10\,\mathrm{pc})$. 
Defining a time translation $t' = t - t_\mathrm{h}$, \autoref{eq:quinlan_H} becomes:
\begin{equation}\label{eq:quinlan_H2}
    \frac{1}{a(t')} \simeq H't' + \frac{1}{a_\mathrm{h}},
\end{equation}
and \autoref{eq:quinlan_K} becomes:
\begin{equation}\label{eq:quinlan_K2}
    e(t') \simeq K \ln\left[ \frac{a_\mathrm{h}}{a(t')} \right] + e_\mathrm{h}.
\end{equation}
The parallel to the first method now becomes apparent: in a given set of simulations, each simulation will have its own value of $\theta$, however each realisation of $\theta$ is itself a draw from a common population distribution that we wish to understand.
If we assume that $e_\mathrm{h}$ is normally distributed with some mean $\mu_{e_\mathrm{h}}$ and standard deviation $\sigma_{e_\mathrm{h}}$, then the distribution of $\sigma_{e_\mathrm{h}}$ is precisely the uncertainty in the standard deviation of the binary eccentricity at the time the binary becomes hard.
The primary benefit to using this second, forward modelling approach, is that we obtain an uncertainty estimate on $\sigma_{e_\mathrm{h}}$, which is not obtainable when using the first, inverse modelling approach. 

A probabilistic graphical model is shown in \drafting{FIGURE}.




\begin{figure*}
    \includegraphics{orbits}
    \caption{\drafting{Orbit of a SMBH in one of the e=0.99 mergers, which by symmetry of the equal mass system is the same as the second BH orbit reflected. Line coloured according to time. Point A shows a time just before the first hard scattering, and point B a time just after. The circle marker indicates the final position of the BH before the BHs become bound.}}
    \label{fig:orbit}
\end{figure*}