\section{Numerical Methods}

\begin{itemize}
    \item describe toy model
    \item describe galaxy ICs: 0.5 Dehnen spheres
    \item describe merger orbit: low and high e
\end{itemize}

\subsection{Galaxy ICs}\label{ssec:gics}
We construct our galaxy models to match those of \citet{gualandris22}.
Each galaxy is represented as a stellar-only Dehnen sphere \citep{dehnen93} with shape parameter $\gamma=0.5$ and scale radius $a=186\,\mathrm{pc}$, where the Dehnen profile is given by:
\begin{equation}\label{eq:dehnen}
    \rho_\star(r) = \frac{(3-\gamma)M_\star}{4\pi} \frac{a}{r^\gamma (r+a)^{(4-\gamma)}}.
\end{equation}
The total stellar mass is $M_\star=10^{10}\,\Msun$, and at the centre of the model galaxy a SMBH of mass $M_\bullet=10^8\,\Msun$ is placed with zero initial velocity.
We test \drafting{X} different mass resolutions by sampling the stellar phase space distribution with varying number of stellar particles: \drafting{list resolutions here}.

\subsection{Merger ICs}
We construct isolated merger initial conditions by placing two galaxies on an elliptical orbit with varying values of eccentricity, at a fixed initial separation of $D=3.72\,\mathrm{kpc}$ and semimajor axis $a_0=2.79\,\mathrm{kpc}$.
We follow the same orbital configurations as \citet{gualandris22}, with three additional orbit. 
The chosen eccentricities $e$ are 0.5, 0.7, 0.9, 0.95, 0.97, and 0.99.
We ensure that the radial and tangential velocities of the initial merger setup are consistent with the values reported in \citet{gualandris22}.
For each orbital configuration, we run ten realisations, to account for Brownian motion effects of the SMBH binary caused by the discretised phase space. 
Each orbital geometry is simulated using our fiducial particle resolution of $M_\bullet/m_\star = 10^4$.
Additionally, we run the $e_0=0.9$ and $e_0=0.99$ orbits at varying mass resolutions, using the galaxy ICs at the mass resolutions described in \autoref{ssec:gics}.

\subsection{Assessing Eccentricity Uncertainty}
In our analysis, we wish to determine the population spread in eccentricity from a finite set of observations.
We assess the uncertainty in eccentricity using two methods: an inverse modelling approach and a forward modelling approach.

The first method follows \citet{nasim20}, and is an inverse problem. 
For each simulation in a given set, we determine the mean eccentricity $e_\mathrm{h}$ over five orbital periods centred on the orbit within which the SMBH binary has become hard.
The binary hardening radius $a_\mathrm{h}$ is defined as \citep[e.g.][]{merritt13}
\begin{equation}\label{eq:ahard}
    a_\mathrm{h} = \frac{q}{1+q} \frac{r_\mathrm{h}}{4},
\end{equation}
where $r_\mathrm{h}=r(m<2M_{\bullet,1})$ is the influence radius and $q = \{q \in \mathbb{R} | 0<q \leq 1\}$ is the mass ratio between the SMBHs.
Under the assumption that each calculated mean eccentricity is a sample of some common distribution, we may (assuming that the distribution is Gaussian) obtain an estimate of the standard deviation of this distribution by taking the standard deviation of the collection of sample eccentricity means. 
