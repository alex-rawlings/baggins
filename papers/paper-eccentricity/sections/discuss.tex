\section{Discussion}

\drafting{
    An interesting point to discuss is the origin of our eccentricity scatter. Early tests of the Gualandris setup seems to indicate that low eccentricity mergers do not demonstrate orbital flips, and are more converged. How to distinguish if the flip is the cause of the eccentricity scatter, or just correlated with eccentricity scatter?
}

We present in \autoref{sec:results} evidence for a tight correlation between the first hard-scattering angle $\theta_\mathrm{defl}$ and the hard SMBH binary eccentricity. 
With the inclusion of the toy model, which provides a resolution-free model of the SMBH scattering process, we may draw two key interpretations of the results.

First, the numerical simulations performed with \ketju{} are deterministic for a given set of initial conditions, given the same computing architecture and numerical parameters (e.g. the number of computing cores used).
The scatter we observe in $\theta_\mathrm{defl}$ is thus a manifestation of Brownian motion induced by the phase space discretisation and other numerical errors, such as floating point round-off errors, denoted $\sigma_\mathrm{sim}$. 
Thus, the numerical simulations present an instance where there is zero uncertainty in the initial conditions of the system, however there is error in the numerical evolution of the system.
We determine the positional displacement between corresponding SMBHs across the simulations at a given mass resolution for the $e_0=0.90$ and $e_0=0.99$ orbits by moving the SMBH binary system to the centre of mass frame, and setting the SMBH positions to coincide with the origin at the beginning of the simulation. 
Computing the standard deviation in the positional displacements for corresponding SMBHs $\sigma_\mathrm{displ}$ between simulation $i$ and simulation $j\neq i$ for all times $t<t_\mathrm{bound}$, we find the SMBH positions vary of the order $\sim10\,\mathrm{pc}$ at the time of the hard scattering, which translates to an uncertainty in the deflection angle. 

Second, in the real Universe there is no numerical error affecting the system evolution.
There is however uncertainty in the initial conditions of the system, whereby the chance of two galaxy merger events having the exact same initial configuration is practically zero.
We denote this uncertainty as $\sigma_\mathrm{true}$.
For the mergers of massive galaxies of the types modelled in this work, an uncertainty in the initial state of $\sigma_\mathrm{true} \sim \sigma_\mathrm{sim}$ is not unexpected, given the influence radius of a $10^8\,\Msun$ SMBH in a stellar background with velocity dispersion of the order $200\,\mathrm{km}\,\mathrm{s}^{-1}$ is $\sim 10\,\mathrm{pc}$ \drafting{From MBW, find better reference}.

Thus, any process that perturbs the impact parameter associated with the hard scattering angle will result in an eccentricity sampled from the $\theta_\mathrm{defl}$--$e_\mathrm{h}$ curve, subject to some small scatter in $e$ resulting from the hierarchical interactions of distant perturbers, or by exchanges in angular momentum with the stellar cluster immediately surrounding the SMBH. 
Critically however, the expected value of eccentricity is a direct consequence of the hard scattering angle, and is not an independent random process, and can be reliably predicted.
We caution that there is not necessarily a one-to-one mapping of the initial merger eccentricity $e_0$ and the hard binary eccentricity $e_\mathrm{h}$, unless the $\theta_\mathrm{defl}$--$e_\mathrm{h}$ curve is uniform.
As such a uniform function is not associated with the radial ($e_0>0.99$) mergers expected to occur in reality, we expect that for a given initial merger orbit $e_0$ a range of possible $e_\mathrm{h}$ would be observed.
The converse also holds: inference about merging SMBH binary eccentricity does not necessarily provide a mapping back to the initial merger eccentricity unless the history of the system is known, which is unable to be achieved in a practical sense in reality.

