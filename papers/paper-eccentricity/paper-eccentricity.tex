%% Based on file 'sample631.tex'
%%
%% Modified 2021 March
%%
%% This is a sample manuscript marked up using the
%% AASTeX v6.31 LaTeX 2e macros.
%%
%% AASTeX is now based on Alexey Vikhlinin's emulateapj.cls 
%% (Copyright 2000-2015).  See the classfile for details.

%% AASTeX requires revtex4-1.cls and other external packages such as
%% latexsym, graphicx, amssymb, longtable, and epsf.  Note that as of 
%% Oct 2020, APS now uses revtex4.2e for its journals but remember that 
%% AASTeX v6+ still uses v4.1. All of these external packages should 
%% already be present in the modern TeX distributions but not always.
%% For example, revtex4.1 seems to be missing in the linux version of
%% TexLive 2020. One should be able to get all packages from www.ctan.org.
%% In particular, revtex v4.1 can be found at 
%% https://www.ctan.org/pkg/revtex4-1.

%% The first piece of markup in an AASTeX v6.x document is the \documentclass
%% command. LaTeX will ignore any data that comes before this command. The 
%% documentclass can take an optional argument to modify the output style.
%% The command below calls the preprint style which will produce a tightly 
%% typeset, one-column, single-spaced document.  It is the default and thus
%% does not need to be explicitly stated.
%%
%% using aastex version 6.3
\documentclass[twocolumn]{aastex631}

%% The default is a single spaced, 10 point font, single spaced article.
%% There are 5 other style options available via an optional argument. They
%% can be invoked like this:
%%
%% \documentclass[arguments]{aastex631}
%% 
%% where the layout options are:
%%
%%  twocolumn   : two text columns, 10 point font, single spaced article.
%%                This is the most compact and represent the final published
%%                derived PDF copy of the accepted manuscript from the publisher
%%  manuscript  : one text column, 12 point font, double spaced article.
%%  preprint    : one text column, 12 point font, single spaced article.  
%%  preprint2   : two text columns, 12 point font, single spaced article.
%%  modern      : a stylish, single text column, 12 point font, article with
%% 		  wider left and right margins. This uses the Daniel
%% 		  Foreman-Mackey and David Hogg design.
%%  RNAAS       : Supresses an abstract. Originally for RNAAS manuscripts 
%%                but now that abstracts are required this is obsolete for
%%                AAS Journals. Authors might need it for other reasons. DO NOT
%%                use \begin{abstract} and \end{abstract} with this style.
%%
%% Note that you can submit to the AAS Journals in any of these 6 styles.
%%
%% There are other optional arguments one can invoke to allow other stylistic
%% actions. The available options are:
%%
%%   astrosymb    : Loads Astrosymb font and define \astrocommands. 
%%   tighten      : Makes baselineskip slightly smaller, only works with 
%%                  the twocolumn substyle.
%%   times        : uses times font instead of the default
%%   linenumbers  : turn on lineno package.
%%   trackchanges : required to see the revision mark up and print its output
%%   longauthor   : Do not use the more compressed footnote style (default) for 
%%                  the author/collaboration/affiliations. Instead print all
%%                  affiliation information after each name. Creates a much 
%%                  longer author list but may be desirable for short 
%%                  author papers.
%% twocolappendix : make 2 column appendix.
%%   anonymous    : Do not show the authors, affiliations and acknowledgments 
%%                  for dual anonymous review.
%%
%% these can be used in any combination, e.g.
%%
%% \documentclass[twocolumn,linenumbers,trackchanges]{aastex631}
%%
%% AASTeX v6.* now includes \hyperref support. While we have built in specific
%% defaults into the classfile you can manually override them with the
%% \hypersetup command. For example,
%%
%% \hypersetup{linkcolor=red,citecolor=green,filecolor=cyan,urlcolor=magenta}
%%
%% will change the color of the internal links to red, the links to the
%% bibliography to green, the file links to cyan, and the external links to
%% magenta. Additional information on \hyperref options can be found here:
%% https://www.tug.org/applications/hyperref/manual.html#x1-40003
%%
%% Note that in v6.3 "bookmarks" has been changed to "true" in hyperref
%% to improve the accessibility of the compiled pdf file.
%%
%% If you want to create your own macros, you can do so
%% using \newcommand. Your macros should appear before
%% the \begin{document} command.
%%
\usepackage{amsmath}
\usepackage{amssymb}
\usepackage{bm} % italic bold math symbols

\newcommand{\vdag}{(v)^\dagger}
\newcommand\aastex{AAS\TeX}
\newcommand\latex{La\TeX}

\newcommand{\ketju}{\textsc{Ketju}}
\newcommand{\mstar}{\textsc{MSTAR}}
\newcommand{\gadget}{\textsc{Gadget-4}}

\newcommand{\Msun}{\ensuremath{\mathrm{M}_{\sun}}}
\newcommand{\Rvir}{\ensuremath{R_\mathrm{vir}}}

\newcommand{\dd}[1]{\ensuremath{\mathrm{d}#1}}
\newcommand{\dv}[2]{\ensuremath{\frac{\dd{#1}}{\dd{#2}}}}
\newcommand{\vb}[1]{\ensuremath{\bm{#1}}} %vectors

% drafting macros
\newcommand{\drafting}[1]{
{\leavevmode\color[RGB]{224, 77, 24}#1}
}

\newcommand{\phj}[1]{
{\leavevmode\color[RGB]{0,0,255}#1}
}

%% Reintroduced the \received and \accepted commands from AASTeX v5.2
%\received{March 1, 2021}
%\revised{April 1, 2021}
%\accepted{\today}

%% Command to document which AAS Journal the manuscript was submitted to.
%% Adds "Submitted to " the argument.
%\submitjournal{PSJ}

%% For manuscript that include authors in collaborations, AASTeX v6.31
%% builds on the \collaboration command to allow greater freedom to 
%% keep the traditional author+affiliation information but only show
%% subsets. The \collaboration command now must appear AFTER the group
%% of authors in the collaboration and it takes TWO arguments. The last
%% is still the collaboration identifier. The text given in this
%% argument is what will be shown in the manuscript. The first argument
%% is the number of author above the \collaboration command to show with
%% the collaboration text. If there are authors that are not part of any
%% collaboration the \nocollaboration command is used. This command takes
%% one argument which is also the number of authors above to show. A
%% dashed line is shown to indicate no collaboration. This example manuscript
%% shows how these commands work to display specific set of authors 
%% on the front page.
%%
%% For manuscript without any need to use \collaboration the 
%% \AuthorCollaborationLimit command from v6.2 can still be used to 
%% show a subset of authors.
%
%\AuthorCollaborationLimit=2
%
%% will only show Schwarz & Muench on the front page of the manuscript
%% (assuming the \collaboration and \nocollaboration commands are
%% commented out).
%%
%% Note that all of the author will be shown in the published article.
%% This feature is meant to be used prior to acceptance to make the
%% front end of a long author article more manageable. Please do not use
%% this functionality for manuscripts with less than 20 authors. Conversely,
%% please do use this when the number of authors exceeds 40.
%%
%% Use \allauthors at the manuscript end to show the full author list.
%% This command should only be used with \AuthorCollaborationLimit is used.

%% The following command can be used to set the latex table counters.  It
%% is needed in this document because it uses a mix of latex tabular and
%% AASTeX deluxetables.  In general it should not be needed.
%\setcounter{table}{1}

%%%%%%%%%%%%%%%%%%%%%%%%%%%%%%%%%%%%%%%%%%%%%%%%%%%%%%%%%%%%%%%%%%%%%%%%%%%%%%%%
%%
%% The following section outlines numerous optional output that
%% can be displayed in the front matter or as running meta-data.
%%
%% If you wish, you may supply running head information, although
%% this information may be modified by the editorial offices.
\shorttitle{Reviving Stochasticity}
\shortauthors{A. Rawlings et al.}
%%
%% You can add a light gray and diagonal water-mark to the first page 
%% with this command:
%% \watermark{text}
%% where "text", e.g. DRAFT, is the text to appear.  If the text is 
%% long you can control the water-mark size with:
%% \setwatermarkfontsize{dimension}
%% where dimension is any recognized LaTeX dimension, e.g. pt, in, etc.
%%
%%%%%%%%%%%%%%%%%%%%%%%%%%%%%%%%%%%%%%%%%%%%%%%%%%%%%%%%%%%%%%%%%%%%%%%%%%%%%%%%
\graphicspath{{./}{figures/}}
%% This is the end of the preamble.  Indicate the beginning of the
%% manuscript itself with \begin{document}.

\begin{document}

\title{Reviving stochasticity: SMBH-binary eccentricity is inherently random}

%% LaTeX will automatically break titles if they run longer than
%% one line. However, you may use \\ to force a line break if
%% you desire. In v6.31 you can include a footnote in the title.

%% A significant change from earlier AASTEX versions is in the structure for 
%% calling author and affiliations. The change was necessary to implement 
%% auto-indexing of affiliations which prior was a manual process that could 
%% easily be tedious in large author manuscripts.
%%
%% The \author command is the same as before except it now takes an optional
%% argument which is the 16 digit ORCID. The syntax is:
%% \author[xxxx-xxxx-xxxx-xxxx]{Author Name}
%%
%% This will hyperlink the author name to the author's ORCID page. Note that
%% during compilation, LaTeX will do some limited checking of the format of
%% the ID to make sure it is valid. If the "orcid-ID.png" image file is 
%% present or in the LaTeX pathway, the OrcID icon will appear next to
%% the authors name.
%%
%% Use \affiliation for affiliation information. The old \affil is now aliased
%% to \affiliation. AASTeX v6.31 will automatically index these in the header.
%% When a duplicate is found its index will be the same as its previous entry.
%%
%% Note that \altaffilmark and \altaffiltext have been removed and thus 
%% can not be used to document secondary affiliations. If they are used latex
%% will issue a specific error message and quit. Please use multiple 
%% \affiliation calls for to document more than one affiliation.
%%
%% The new \altaffiliation can be used to indicate some secondary information
%% such as fellowships. This command produces a non-numeric footnote that is
%% set away from the numeric \affiliation footnotes.  NOTE that if an
%% \altaffiliation command is used it must come BEFORE the \affiliation call,
%% right after the \author command, in order to place the footnotes in
%% the proper location.
%%
%% Use \email to set provide email addresses. Each \email will appear on its
%% own line so you can put multiple email address in one \email call. A new
%% \correspondingauthor command is available in V6.31 to identify the
%% corresponding author of the manuscript. It is the author's responsibility
%% to make sure this name is also in the author list.
%%
%% While authors can be grouped inside the same \author and \affiliation
%% commands it is better to have a single author for each. This allows for
%% one to exploit all the new benefits and should make book-keeping easier.
%%
%% If done correctly the peer review system will be able to
%% automatically put the author and affiliation information from the manuscript
%% and save the corresponding author the trouble of entering it by hand.

%\correspondingauthor{August Muench}
%\email{greg.schwarz@aas.org, gus.muench@aas.org}

\author[0000-0003-1807-6321]{Alexander Rawlings}
\affiliation{Department of Physics,
Gustaf H\"allstr\"omin katu 2, FI-00014, University of Helsinki, Finland}

\author[0000-0001-5721-9335]{Matias Mannerkoski}
\affiliation{Department of Physics,
Gustaf H\"allstr\"omin katu 2, FI-00014, University of Helsinki, Finland}

\author[0000-0001-8741-8263]{Peter H. Johansson}
\affiliation{Department of Physics,
Gustaf H\"allstr\"omin katu 2, FI-00014, University of Helsinki, Finland}

\author{Others}

\correspondingauthor{Alexander Rawlings}
\email{alexander.rawlings@helsinki.fi}

%% Note that the \and command from previous versions of AASTeX is now
%% depreciated in this version as it is no longer necessary. AASTeX 
%% automatically takes care of all commas and "and"s between authors names.

%% AASTeX 6.31 has the new \collaboration and \nocollaboration commands to
%% provide the collaboration status of a group of authors. These commands 
%% can be used either before or after the list of corresponding authors. The
%% argument for \collaboration is the collaboration identifier. Authors are
%% encouraged to surround collaboration identifiers with ()s. The 
%% \nocollaboration command takes no argument and exists to indicate that
%% the nearby authors are not part of surrounding collaborations.

%% Mark off the abstract in the ``abstract'' environment. 
\begin{abstract}


\drafting{
    Key results:
    \begin{enumerate}
        \item Eccentricity of hard binary depends on deflection angle 
        \item Mapping $\theta_\mathrm{defl} \rightarrow e_\mathrm{h}$ relatively independent of resolution, however low res explores wider domain space 
        \item Explored eccentricity space can be full $[0,1]$ range for certain systems 
        \item Exact values of $e$ from simulations meaningless, need to understand the distribution of 
        \item Current resolutions appropriate for exploring statistics of eccentricity
    \end{enumerate}
    Limited to 250 words.
}

\end{abstract}

%% Keywords should appear after the \end{abstract} command. 
%% The AAS Journals now uses Unified Astronomy Thesaurus concepts:
%% https://astrothesaurus.org
%% You will be asked to selected these concepts during the submission process
%% but this old "keyword" functionality is maintained in case authors want
%% to include these concepts in their preprints.
\keywords{Edit keywords}

%% From the front matter, we move on to the body of the paper.
%% Sections are demarcated by \section and \subsection, respectively.
%% Observe the use of the LaTeX \label
%% command after the \subsection to give a symbolic KEY to the
%% subsection for cross-referencing in a \ref command.
%% You can use LaTeX's \ref and \label commands to keep track of
%% cross-references to sections, equations, tables, and figures.
%% That way, if you change the order of any elements, LaTeX will
%% automatically renumber them.
%%
%% We recommend that authors also use the natbib \citep
%% and \citet commands to identify citations.  The citations are
%% tied to the reference list via symbolic KEYs. The KEY corresponds
%% to the KEY in the \bibitem in the reference list below. 


%-------------------------- MAIN TEXT STARTS HERE --------------------------%
% INTRODUCTION
\section{Introduction}
Supermassive black holes (SMBHs) are believed to reside at the centres of almost all massive galaxies \citep[e.g.][]{kormendy2013}. 
As galaxies dynamically interact, so too are their SMBHs expected to interact in a three-phase merger process \citep{begelman1980}.
The first phase, dynamical friction \citep{chandrasekhar1943}, acts to bring the SMBHs from kiloparsec-scale separations to parsec-scale separations, after which the SMBHs form a bound binary system with a semimajor axis $a$ and eccentricity $e$.
Following the binding of the SMBHs, the SMBH binary separation is reduced through sequential slingshot encounters with the surrounding stellar distribution \citep{hills1980, hills1983, quinlan1996}.
Chaotic interactions drive the SMBH binary to sub-parsec separation: a regime where gravitational wave (GW) emission becomes the dominant mechanism by which the SMBH binary can lose its remaining orbital energy and angular momentum, resulting in the coalescence of the two SMBHs into a single SMBH.

It is well known that the efficiency by which GW emission can remove orbital energy and angular momentum from a binary system is highly sensitive to the binary eccentricity \citep{peters1964}, with the binary merger timescale being particularly sensitive. 
The complex nature of SMBH coalescence in a galaxy merger setting necessitates the use of numerical techniques to understand, and to provide quantitative predictions for observational programmes such as pulsar timing arrays \citep[PTAs, e.g.][]{babak2016,falxa2023} and the upcoming Laser Interferometer Space Antenna \citep[LISA, e.g.][]{amaro-seoane2007} mission.
Understanding how faithfully SMBH binary eccentricity is captured in simulations is thus a critical piece of the puzzle in constraining the SMBH merger rate that the observational GW community aims to detect. 

Previous work by \citet{nasim2020} have argued that scatter in SMBH binary eccentricity observed in gas-free merger simulations is an artefact of poor phase space discretisation, and that in the real Universe where SMBH masses are far greater than stellar masses, $M_\bullet \gg m_\star$, SMBH binary eccentricity is a reliably predictable quantity.
In this work, we find that SMBH binary eccentricity is reliably predictable only for particular initial conditions of the galaxy merger system, and does not in general hold.
Indeed, we find that for realistic galaxy merger orbits where the initial orbital eccentricity of the system is $e_0>0.90$, stochasticity in the final SMBH binary eccentricity is present. 


% NUMERICAL
\section{Numerical Methods}

\begin{figure*}
    \centering
    \includegraphics{eccentricities.pdf}
    \caption{Eccentricity $e$ (note the non-linear scale) as a function of shifted time $t'=t-t_\mathrm{bound}$ for select representative simulations from the $e_0 = 0.90$ set (blue lines) and $e_0 = 0.99$ set (orange lines). The downwards-pointing arrows indicate the mean time when the SMBH binaries becomes hard for the two simulation sets. In both the shown $e_0 = 0.90$ and $e_0 = 0.99$ sets, one simulation has a low eccentricity, displacing it to a value of $e$ greater than $2.5\sigma$ from the collective mean. In the $e_0=0.90$ set, two simulations have a SMBH binary merger, as indicated by the rapid circularisation of the eccentricity.}
    \label{fig:eccentricities}
\end{figure*}

We construct a number of idealised galaxy merger simulations, which we evolve with our new version of \ketju{} \citep{rantala2017}, implemented in \gadget{}.
For details about the implementation of \ketju{}, we point the reader to \drafting{Mannerkoski in prep}.

\subsection{Galaxy ICs}\label{ssec:gics}
We construct our galaxy models to match those of \citet{gualandris2022}.
Each galaxy is represented as a stellar-only Dehnen sphere \citep{dehnen1993} with shape parameter $\gamma=0.5$ and scale radius $a=186\,\mathrm{pc}$, where the Dehnen profile is given by:
\begin{equation}\label{eq:dehnen}
    \rho_\star(r) = \frac{(3-\gamma)M_\star}{4\pi} \frac{a}{r^\gamma (r+a)^{(4-\gamma)}}.
\end{equation}
The total stellar mass is $M_\star=10^{10}\,\Msun$, and at the centre of the model galaxy a SMBH of mass $M_\bullet=10^8\,\Msun$ is placed with zero initial velocity.
We test five different mass resolutions by sampling the stellar phase space distribution with varying number of stellar particles: $N_\star = \{2.5\times10^5, 5.0\times10^5, 1.0\times10^6, 2.0\times10^6, 4.0\times10^6\}$.

\subsection{Merger ICs}
We construct isolated merger initial conditions by placing two galaxies on an elliptical orbit with varying values of eccentricity, at a fixed initial separation of $D=3.72\,\mathrm{kpc}$ and semimajor axis $a_0=2.79\,\mathrm{kpc}$.
The chosen eccentricities $e_0$ are 0.90 and 0.99.
We ensure that the radial and tangential velocities of the initial merger setup are consistent with the values reported in \citet{gualandris2022}.
For each orbital configuration, we run ten realisations, to account for Brownian motion effects of the SMBH binary caused by the discretised phase space. 
Each orbital geometry is simulated using our fiducial particle resolution of $M_\bullet/m_\star = 10^4$.
Additionally, we run the $e_0=0.90$ and $e_0=0.99$ orbits at varying mass resolutions, using the galaxy ICs at the mass resolutions described in \autoref{ssec:gics}.

\subsection{Assessing Eccentricity Uncertainty}
In our analysis, we wish to determine the population spread in eccentricity from a finite set of observations.
We assess the uncertainty in eccentricity using an inverse modelling approach following \citet{nasim2020}. 
For each simulation in a given set, we determine the mean eccentricity $e_\mathrm{h}$ over five orbital periods centred on the orbit within which the SMBH binary has become hard.
The binary hardening radius $a_\mathrm{h}$ is defined as \citep[e.g.][]{merritt2006}
\begin{equation}\label{eq:ahard}
    a_\mathrm{h} = \frac{q}{(1+q)^2} \frac{r_\mathrm{m}}{4},
\end{equation}
where $r_\mathrm{m}=r(m<2M_{\bullet,1})$ is the influence radius and $q = \{q \in \mathbb{R} | 0<q \leq 1\}$ is the mass ratio between the SMBHs.
Under the assumption that each calculated mean eccentricity is a sample of some common distribution, we may (assuming that the distribution is Gaussian) obtain an estimate of the standard deviation of this distribution by taking the standard deviation of the collection of sample eccentricity means, denoted $\sigma_e$. 


% RESULTS
\section{Results}\label{sec:results}

\begin{figure}
    \includegraphics{convergence}
    \caption{Convergence of the eccentricity scatter $\sigma_e$ as a function of mass resolution $M_\bullet/m_\star$ and the total number of stellar particles $2N_\star$ for initial eccentricities $e_0=0.90$ and $e_0=0.99$. The expected $1/\sqrt{N_\star}$ scaling is recovered for the $e_0 = 0.90$ suite after applying a $2.5\sigma$-cut to the simulation data. For the $e_0=0.99$ suite, a steeper scaling than the $1/\sqrt{N_\star}$ is obtained after applying a similar $2.5\sigma$-cut to the data.}
    \label{fig:convergence}
\end{figure}


\subsection{Eccentricity scatter in low-e orbits}\label{ssec:lowe}

A representative set of realisations for the $e_0=0.90$ orbit is shown with blue lines in \autoref{fig:eccentricities}.
We define a shifted time $t'$ such that $t'(0) = t_\mathrm{bound}$. 
One simulation is seen to lie significantly beyond the mean eccentricity, with $e_\mathrm{h}\sim 0.2$.

We show the dependence of eccentricity standard deviation on mass resolution for the $e_0=0.90$ orbit in \autoref{fig:convergence} with circle and square markers. 
Taking for each mass resolution the full set of ten simulations (circle markers), immediately apparent is the lack of convergence of $\sigma_e$, with $5\times10^3$ and $1\times10^4$ mass resolutions demonstrating an increased scatter compared to the lowest mass resolution of $2.5\times10^3$.
The peculiar values of $\sigma_e$ arise due to the assumed Gaussian distribution of the eccentricity between simulations.
Inspecting the orbital parameters for all simulations that form the $e_0=0.90$ suite, it is found that for resolutions of $5\times10^3$ and $1\times10^4$ the eccentricity clusters around a value of $e_\mathrm{h}\simeq0.90$, except for one simulation at both $5\times10^3$ and $1\times10^4$ resolutions which has a value of $e_\mathrm{h}<0.4$.
This one outlier for each of the two resolutions artificially inflates the measured eccentricity variance, and by applying a $2.5\sigma$-cut in eccentricity (i.e., taking those lines within the blue region of the left panel of \autoref{fig:eccentricities}) we recover the $1/\sqrt{M_\bullet/m_\star}\propto 1/\sqrt{N_\star}$ scaling of $\sigma_e$ as in \citet{nasim2020} (square markers), apart from the highest mass resolution set of $M_\bullet/m_\star=4\times10^4$.
However, the eccentricity values beyond $2.5\sigma$ from the mean are each a valid measurement from the simulation, and highlight the presence of stochasticity in eccentricity for these idealised simulation set-ups. 
The cause of these low eccentricity values is discussed in \autoref{ssec:toy}.


\subsection{Eccentricity scatter in high-e orbits}\label{ssec:highe}

Similar to the $e_0=0.90$ case, we show a representative set of realisation for the $e_0=0.99$ orbit with orange lines in \autoref{fig:eccentricities}.
Again, one simulation is seen to have $e_\mathrm{h} \sim 0.1$, significantly offset from the mean value of $e_\mathrm{h}$ for this simulation set. 

The dependence of eccentricity standard deviation on mass resolution for the $e_0=0.99$ orbit is shown in \autoref{fig:convergence} with triangle and diamond markers. 
We find that the scaling of $\sigma_e$ is steeper than the $1/\sqrt{N_\star}$ relation described in \citet{nasim2020}.
We attribute this dramatic decrease in $\sigma_e$ to an inflated value of $\sigma_e$ at low mass resolution: there is increased scatter in eccentricity at low mass resolutions for very radial initial orbits of the SMBH binary. 
As the majority of major galaxy mergers in the real Universe are expected to be radial \citep[e.g.][]{khochfar2006}, understanding if the origin of the scatter in eccentricity is a numerical or a physical effect is of paramount importance, and will have direct implications for the predictions of SMBH merger rates by ongoing and future GW detection missions. 

\subsection{Binary binding process and the scatter in eccentricity}

To understand the observed scatter in eccentricity, we investigate the binary binding process in each simulation. 
Whilst the two SMBHs are unbound, they undergo a number of hyperbolic orbits influenced by the galactic merger potential.
At each pericentre passage, the two SMBHs scatter off each other with some deflection angle $\theta_\mathrm{defl}$, defined:
\begin{equation}\label{eq:theta}
    \theta_\mathrm{defl} = 2 \arctan \left(\frac{GM}{L\sqrt{2E}} \right)
\end{equation}
where $M=M_{\bullet,1}+M_{\bullet,2}$, and $L$ is the magnitude of the SMBH 
binary angular momentum vector and $E$ the binary orbital energy at the time of the pericentre passage \citep{binney2008}.
The deflection angle is a function of the impact parameter $b$ of the SMBH trajectory, however is able to be more robustly measured than $b$ from the simulation data.
We set as the first deflection angle the first 
angle $\theta_\mathrm{defl}$ that exceeds an orbit-dependent threshold:
$\theta_{\mathrm{defl,min}}=90\degr$ for the $e_0=0.99$ suite and $\theta_{\mathrm{defl,min}}=45\degr$ for the $e_0=0.90$ suite.
In general, this corresponds to the second or third actual pericentre passage of the SMBHs in the simulations.
An example of $\theta_\mathrm{defl}$ is shown for two realisations of the $e_0=0.90$ ($e_0=0.99$) orbit in the left (right) panel of \autoref{fig:orbit}, which occurs between the points A and B in the SMBH binary orbit in each inset panel. 
We observe evidence for a relationship between the deflection angle $\theta_\mathrm{defl}$ and the resulting eccentricity at the time the SMBH binary becomes hard, which we explain using a simple toy model.

\begin{figure*}
    \centering
    \includegraphics{orbit}
    \caption{Orbits from $t=0\,\mathrm{Myr}$ to a time shortly after $t_\mathrm{bound}$ of a single SMBH in two representative realisations of the $e_0=0.90$ mergers (left) and the $e_0=0.99$ mergers (right), which by symmetry of the equal mass system is the same as the second BH orbit reflected. Each line is coloured according to the shifted time $t'=t-t_\mathrm{bound}$ of the simulation, however the colours for $|t'|>2\,\mathrm{Myr}$ is constant for visual clarity of the deflection-phase of the evolution. In each inset panel, point A shows a time just before the first hard scattering $\theta_\mathrm{defl}$, and point B a time just after. The circle marker indicates the position of the SMBH just before the SMBHs form a bound binary.}
    \label{fig:orbit}
\end{figure*}


\begin{figure*}
    \centering
    \includegraphics{theta_e_sim_and_model}
    \caption{
    Scattering deflection angle $\theta_\mathrm{defl}$ and the resulting binary eccentricity
    in the $e_0=0.90$ (left) and $e_0=0.99$ (right) simulations,
    compared to reasonably well fitting model curves.
    The solid lines show the results from the toy model with $e_\mathrm{s}=0.905$,
    while the semitransparent lines show the results for  $e_\mathrm{s}=0.900$ and $e_\mathrm{s}=0.910$.
    The model curves have been shifted left by $12\degr$ and $6\degr$
    to better match the simulation data, with the shift also indicated by arrows.
    The marginal histograms show kernel density estimates to the simulation data.
    }
    \label{fig:theta_e}
\end{figure*}


\subsection{Reproducing the eccentricity behavior with a toy model}\label{ssec:toy}

\begin{figure*}
    \centering
    \includegraphics{sample_model_orbits}
    \caption{
    Top:
    Sample orbits computed with the toy model for different background potential eccentricities $e_\mathrm{s}$,
    with otherwise identical initial conditions.
    The dashed lines show the isopotential contours of the model.
    Only the early part of the evolution is shown for clarity.
    Bottom:
    The orbital eccentricity $e$ of the model orbits (note the non-linear scale).
    The vertical line marks the end of the period shown in the top panels. 
    }
    \label{fig:model_orbits}
\end{figure*}

The dependency of the binary eccentricity on the 
deflection angle $\theta_\mathrm{defl}$ during the binary formation
can be reproduced using a simple toy model,
which includes only the essential components relevant to this process.
Taking the orbit to lie in the $xy$-plane,
the relative motion of the two equal-mass SMBHs in this model is follows the 
equation of motion
\begin{equation} \label{eq:toy_model_eom}
\ddot{\vb{x}} = -\frac{2 G M_\bullet}{|\vb{x}|^3} \vb{x} + \vb{a}_\mathrm{bg} + \vb{a}_\mathrm{DF},
\end{equation}
where $\vb{x}=(x,y)$ is the separation vector of the SMBHs,
$M_\bullet = 10^8\,\Msun$ is the mass of a single SMBH,
$\vb{a}_\mathrm{bg}$ the acceleration due to the asymmetric background potential,
and $\vb{a}_\mathrm{DF}$ the acceleration due to dynamical friction.

The stellar background is modeled as a constant density spheroidal potential
\citep[e.g.][]{binney2008}
\begin{equation}
\Phi_\mathrm{bg}(\vb{x}) = \pi G \rho (A_x x^2 + A_y y^2),
\end{equation}
with the acceleration given by 
\begin{equation}
\vb{a}_\mathrm{bg}(\vb{x}) = - \nabla \Phi_\mathrm{bg}(\vb{x}).
\end{equation}
The $A$ coefficients are related to the eccentricity $e_\mathrm{s}$ of the spheroid as
\begin{align}
A_x &= 2 \frac{1-e_\mathrm{s}^2}{e_\mathrm{s}^2} 
    \left[\frac{1}{2 e_\mathrm{s}} \ln\left(\frac{1+e_\mathrm{s}}{1-e_\mathrm{s}}\right) - 1 \right]\\
A_y &= \frac{1-e_\mathrm{s}^2}{e_\mathrm{s}^2} 
        \left[ \frac{1}{1-e_\mathrm{s}^2} 
        - \frac{1}{2 e_\mathrm{s}} \ln\left(\frac{1+e_\mathrm{s}}{1-e_\mathrm{s}}\right)\right],
\end{align}
with the long axis aligned along the $x$-axis.
The stellar density is set to $\rho = 300 \,\Msun\,\mathrm{pc}^{-3}$,
which approximately matches the values seen in the simulations when SMBH binary is becoming bound.

Dynamical friction is modeled using the \citet{chandrasekhar1943} formula assuming
a Maxwellian distribution with a constant velocity dispersion
${\sigma_\mathrm{M}=200\,\mathrm{km\,s^{-1}}}$,
set based on the value measured from the simulations.
This formula gives the acceleration of a single BH due to dynamical friction as
\citep[e.g.][]{binney2008}
\begin{equation}\label{eq:chandra_df}
\begin{aligned}
\vb{a}_{1,\mathrm{DF}} = &-\frac{4 \pi G^2 M_\bullet \rho \ln{\Lambda}}{|\vb{v}_{1}|^3}\\
                        &\times\left(\operatorname{erf}(X) - \frac{2X}{\sqrt{\pi}} \exp(-X^2)\right)\vb{v}_{1}.
\end{aligned}
\end{equation}
Here $\vb{v}_1 = \dot{\vb{x}}/2$ is the velocity of a single BH, 
$X = |\vb{v}_1|/\sqrt{2}\sigma_\mathrm{M}$,
and the Coulomb logarithm is taken to have the value $\ln{\Lambda} = 5$,
which is appropriate for the size of the simulated systems.

The effect of dynamical friction weakens as the SMBH binary orbit shrinks.
To account for this, when the binary is bound 
we multiply the acceleration given by Eq.~\eqref{eq:chandra_df}
with the smooth cut-off function
\begin{equation}
f(a) = \frac{1}{1 + \exp[(a_\mathrm{c}-a)/d_\mathrm{c}]}.
\end{equation}
Here $a$ is the semi-major axis of a bound binary,
and the cut-off scales are $a_\mathrm{c}=2 a_\mathrm{h}$
and $d_\mathrm{c} = 0.5 a_\mathrm{h}$,
with $a_\mathrm{h}$ defined by equation \eqref{eq:ahard}.
The total dynamical friction term is then
\begin{equation}
\vb{a}_{\mathrm{DF}} = 2 f(a) \vb{a}_{1,\mathrm{DF}}.
\end{equation}

To mimic the nearly linearly plunging orbits before the scattering event seen in \autoref{fig:orbit},
we specify the initial conditions as 
\begin{align}
\vb{x}_0 &= (25\,\mathrm{pc}, b)\\
\dot{\vb{x}}_0 &= (-v_0, 0).
\end{align}
Based on the simulations, we set $v_0 = 450\,\mathrm{km\,s^{-1}}$ to match the
typical SMBH relative velocity at this separation in the $e_0=0.90$ case,
and $v_0 = 560\,\mathrm{km\,s^{-1}}$ to match the $e_0=0.99$ case.
The spheroid eccentricity parameter is not easily measurable from the simulations,
due to stellar components that remain bound to the SMBHs and obscure the background
potential relevant for the dynamics,
as well as due to the potential in the simulations being constantly evolving.
We therefore perform calculations with different values of $e_\mathrm{s}$,
finding a good fit to simulation results for $e_\mathrm{s}\approx 0.9$.

We compute the resulting eccentricities for a range of impact parameters up to $b=20\,\mathrm{pc}$
for the two initial velocities,
by solving the equation of motion \eqref{eq:toy_model_eom} until the binary has become hard using
the error controlled 8th order Runge--Kutta method DOP853 included in the SciPy library \citep{virtanen2020}.
For reference, a $90\degr$ deflection during the first pericenter passage
corresponds to initial impact parameters of $b\approx 5.8\,\mathrm{pc}$ and
$b\approx 3.3\,\mathrm{pc}$ for the $e_0=0.90$ and $e_0=0.99$ cases, respectively.

The resulting model curves are shown together with the simulation data in \autoref{fig:theta_e}.
To correctly match the data in the $e_0=0.90$ case, the model curve has been shifted to the left by $12\degr$,
while in the $e_0=0.99$ case a smaller shift of $6\degr$ is used.
These shifts are likely required due to the rotation of the stellar component tilting the background potential
relative to the SMBH trajectories in the simulations, with the effect being stronger in the $e_0=0.90$ case
due to the less radial merger orbit.
In the $e_0=0.99$ case we can also see that the simulation data has significant scatter around
the model curve in the $\theta_\mathrm{defl} \gtrsim 120\degr$ region,
although the model does appear to capture the mean behavior fairly well.
The behavior of the model in this region is also quite sensitive to the exact value of $e_\mathrm{s}$,
which might be related to the large scatter in the simulation data.

Overall we can however see that this simple model correctly captures the behavior of eccentricity seen in the simulations.
We can therefore use the model to understand the cause of the complicated dependence of the binary eccentricity
on the deflection angle $\theta_\mathrm{defl}$.
\autoref{fig:model_orbits} shows sets of orbits computed with the model for two different potential shapes,
with the parameters otherwise as in the model that matches the $e_0=0.90$ simulation results.
In the $e_\mathrm{s}=0.9$ case the shape of the orbit after the initial scattering depends strongly on the deflection angle,
as the asymmetric potential causes torque on the SMBHs that are not moving along the main axes of the potential.
On the other hand, in the more symmetric potential with $e_\mathrm{s}=0.2$,
the orbit shape is not very sensitive to the deflection angle due to the symmetry of the system,
and the binary always reaches extremely high eccentricities.


% DISCUSSION
\section{Discussion and Conclusions}

We show in \autoref{sec:results} that variation in eccentricity $e_\mathrm{h}$ for a given system configuration is tightly correlated with the variation in the deflection angle $\theta_\mathrm{defl}$, independent of the simulation mass resolution.
The inclusion of the toy model allows us to investigate the relationship between $\theta_\mathrm{defl}$ and $e_\mathrm{h}$ in the infinite-mass resolution limit, around which the quantities measured from the simulation demonstrate minimal scatter. 
As shown by the simulation points and the marginal kernel density estimates (KDEs) of \autoref{fig:theta_e}, the variation in eccentricity can span almost the full domain space of $[0,1]$.

We turn our discussion to the cause of variation in the impact parameter $b$ (and thus by \autoref{eq:theta}, $\theta_\mathrm{defl}$) first in simulations, and then in reality. 

In numerical simulations with a discretised phase space, the two primary causes of variation in $b$ are Brownian motion of the SMBH binary \citep{merritt2001}, and initial particle sampling variation in the merger IC, both of which have been shown to be more pronounced in lower mass resolution simulations compared to higher mass resolution simulations. 
By allowing the SMBH binary to wander further from its expected location in phase space, a wider range of impact parameters, and thus deflection angles, may be sampled.
This consequently maps to a different sampling of the eccentricity space, however due to the non-linear mapping (as shown in \autoref{fig:theta_e}), the eccentricity domain space sampled can vary between $\sigma_e \sim 0.01$ to $\sigma_e>0.1$, for the shown system configurations.
Different systems, with different galaxy density profiles or merger orbits, may have an even more pronounced value of $\sigma_e$.

Conversely, in the real universe, variation in the impact parameter of a similar magnitude to the presented simulations but as a consequence of different processes is expected due to the lack of system reproducibility.
As with many physical systems, random variations in the system state demand the use of statistics to ascertain a most probable configuration or outcome. 
Work by \citet{batcheldor2010} has suggested that the SMBH in M87 may be displaced from the galactic photo-centre by $\sim7\,\mathrm{pc}$ due to jet acceleration or gravitational recoil kicks. 
Assuming these processes are at play in other galaxies (particularly elliptical galaxies constructed through major mergers \citep{naab2006}), deviations of a few parsecs of the SMBH from the host galaxy photo-centre are expected and would influence the impact parameter two merging SMBHs would experience.

The results presented in this work are significant in constraining observational quantities that are correlated with binary eccentricity: most notably SMBH binary merger timescale. 
As the relevant scales of the impact parameter that lead to the deflection angle are typically of the order of a few parsecs, and are thus generally unresolvable in gravitationally-softened simulations, numerical simulations sample instead a non-deterministic distribution of potential impact parameters.
Building on from work by \citet{portegieszwart2018}, microscopic perturbations can grow to large scale, observable effects in a gravitational system over relatively short timescales, suggesting that even at physical mass resolution, chaotic dynamics still lead to a non-deterministic sampling of the impact parameter domain. 
From \autoref{fig:theta_e}, different initial orbital geometries also influence the deflection-angle domain space sampled, with not all merger geometries necessarily able to sample the full eccentricity domain (neglecting minor scatter in $e_\mathrm{h}$ due to sampling variation compared to the analytical prediction).
Importantly however, the sampling variation in $e_\mathrm{h}$ by the numerical simulations are largely independent of mass resolution, indicating that even at moderate mass resolutions of $M_\bullet/m_\star \sim 10^3$, the eccentricity distribution can be reliably sampled. 

Finally, we discuss the caveats of our conclusions.
\begin{enumerate}
\item The systems we test in this work are exceptionally idealistic: spherical stellar bulges without dark matter (DM) on a merger orbit with very small initial separation. 
We do not however expect the conclusions concerning the $\theta_\mathrm{defl}$--$e_\mathrm{h}$ mapping to be significantly impacted by expanding our models to more realistic gravitational systems, as DM acts to provide a dynamical friction force over separation scales much larger than that of the impact parameter. 
Qualitatively, the scatter in eccentricity presented in this work is consistent with the eccentricity scatter in simulations including DM. 
\item The mergers presented in this work are all of equal mass galaxies, leading to very dramatic deformations of the potential that deviate from spherical symmetry. 
In the instance of minor mergers where the galaxy merger remnant is not as triaxial as the major merger case \citep{khan2012}, we do not expect to see as sensitive dependence of eccentricity on deflection angle compared to the major merger case (refer to \autoref{fig:model_orbits}).
\end{enumerate}

In conclusion, we have shown the importance of the deflection angle $\theta_\mathrm{defl}$ on the resulting hard binary eccentricity $e_\mathrm{h}$. 
By using a resolution-free toy model of the SMBH scattering process, we demonstrate that uncertainty in SMBH binary eccentricity is not caused solely by discretisation effects in galaxy merger simulations.
The results presented here justify extending the investigation to more realistic galaxy merger scenarios, to quantify the expected range of hard binary eccentricities critical for merger timescale predictions for future gravitational wave observation missions. 

% FINAL BITS
\textit{Software:} \ketju{} \citep{rantala2017}, \gadget{} \citep{springel2022}, NumPy \citep{harris2020}, SciPy \citep{virtanen2020}, Matplotlib \citep{hunter2007}, pygad \citep{rottgers2020}. 


\section*{Acknowledgements}

P.H.J. acknowledge the support
by the European Research Council via ERC Consolidator Grant KETJU (no. 818930).

The numerical simulations used computational resources provided by
the CSC -- IT Center for Science, Finland.

We list here the roles and contributions of the authors according to the Contributor Roles Taxonomy (CRediT\footnote{https://credit.niso.org}). \textbf{AR}: Conceptualisation, Investigation, Formal analysis, Data curation, Writing -- original draft. \textbf{MM}: Conceptualisation, Formal analysis, Writing -- original draft. \textbf{PHJ}: Supervision, Writing -- review \& editing.

%% For this sample we use BibTeX plus aasjournals.bst to generate the
%% the bibliography. The sample631.bib file was populated from ADS. To
%% get the citations to show in the compiled file do the following:
%%
%% pdflatex sample631.tex
%% bibtext sample631
%% pdflatex sample631.tex
%% pdflatex sample631.tex

\bibliography{../ref}{}
\bibliographystyle{aasjournal}

%% This command is needed to show the entire author+affiliation list when
%% the collaboration and author truncation commands are used.  It has to
%% go at the end of the manuscript.
%\allauthors

%% Include this line if you are using the \added, \replaced, \deleted
%% commands to see a summary list of all changes at the end of the article.
%\listofchanges

\end{document}

% End of file `sample631.tex'.
