% mnras_template.tex 
%
% LaTeX template for creating an MNRAS paper
%
% v3.0 released 14 May 2015
% (version numbers match those of mnras.cls)
%
% Copyright (C) Royal Astronomical Society 2015
% Authors:
% Keith T. Smith (Royal Astronomical Society)

% Change log
%
% v3.0 May 2015
%    Renamed to match the new package name
%    Version number matches mnras.cls
%    A few minor tweaks to wording
% v1.0 September 2013
%    Beta testing only - never publicly released
%    First version: a simple (ish) template for creating an MNRAS paper

%%%%%%%%%%%%%%%%%%%%%%%%%%%%%%%%%%%%%%%%%%%%%%%%%%
% Basic setup. Most papers should leave these options alone.
\documentclass[fleqn,usenatbib]{mnras}

% MNRAS is set in Times font. If you don't have this installed (most LaTeX
% installations will be fine) or prefer the old Computer Modern fonts, comment
% out the following line
\usepackage{newtxtext,newtxmath}
% Depending on your LaTeX fonts installation, you might get better results with one of these:
%\usepackage{mathptmx}
%\usepackage{txfonts}

% Use vector fonts, so it zooms properly in on-screen viewing software
% Don't change these lines unless you know what you are doing
\usepackage[T1]{fontenc}

% Allow "Thomas van Noord" and "Simon de Laguarde" and alike to be sorted by "N" and "L" etc. in the bibliography.
% Write the name in the bibliography as "\VAN{Noord}{Van}{van} Noord, Thomas"
\DeclareRobustCommand{\VAN}[3]{#2}
\let\VANthebibliography\thebibliography
\def\thebibliography{\DeclareRobustCommand{\VAN}[3]{##3}\VANthebibliography}


%%%%% AUTHORS - PLACE YOUR OWN PACKAGES HERE %%%%%

% Only include extra packages if you really need them. Common packages are:
\usepackage{graphicx}	% Including figure files
\usepackage{amsmath}	% Advanced maths commands
\usepackage{amssymb}	% Extra maths symbols

%%%%%%%%%%%%%%%%%%%%%%%%%%%%%%%%%%%%%%%%%%%%%%%%%%

%%%%% AUTHORS - PLACE YOUR OWN COMMANDS HERE %%%%%

% Please keep new commands to a minimum, and use \newcommand not \def to avoid
% overwriting existing commands. Example:
%\newcommand{\pcm}{\,cm$^{-2}$}	% per cm-squared
\newcommand{\ketju}{\textsc{Ketju}}
\newcommand{\mstar}{\textsc{Mstar}}
\newcommand{\gadget}{\textsc{Gadget-3}}

\newcommand{\Msun}{\ensuremath{M_{\sun}}}
\newcommand{\Rvir}{\ensuremath{R_\mathrm{vir}}}

\newcommand{\dd}[1]{\ensuremath{\mathrm{d}#1}}
\newcommand{\dv}[2]{\ensuremath{\frac{\dd{#1}}{\dd{#2}}}}

\graphicspath{{figures/}}

% XXX: For highlighting draft notes, these should be removed when done
\usepackage{color}
\newcommand{\drafting}[1]{
{\leavevmode\color[RGB]{224, 77, 24}#1}
}


\newcommand{\phj}[1]{
{\leavevmode\color[RGB]{0,0,255}#1}
}

%%%%%%%%%%%%%%%%%%%%%%%%%%%%%%%%%%%%%%%%%%%%%%%%%%

%%%%%%%%%%%%%%%%%%% TITLE PAGE %%%%%%%%%%%%%%%%%%%

% Title of the paper, and the short title which is used in the headers.
% Keep the title short and informative.
\title[Reviving Stochasticity]{Reviving stochasticity: eccentricity convergence depends on black hole binary orbit}

% The list of authors, and the short list which is used in the headers.
% If you need two or more lines of authors, add an extra line using \newauthor
\author[A. Rawlings et al.]{
Alexander Rawlings,$^{1}$\thanks{E-mail: alexander.rawlings@helsinki.fi}
Others,$^{1}$
\\
% List of institutions
$^{1}$
Department of Physics,
Gustaf H\"allstr\"omin katu 2, FI-00014, University of Helsinki, Finland
}

% These dates will be filled out by the publisher
\date{Accepted XXX. Received YYY; in original form ZZZ}

% Enter the current year, for the copyright statements etc.
\pubyear{2022}

% Don't change these lines
\begin{document}
\label{firstpage}
\pagerange{\pageref{firstpage}--\pageref{lastpage}}
\maketitle

% Abstract of the paper
\begin{abstract}
We present some stuff about eccentricity convergence in Nbody sims
\end{abstract}

% Select between one and six entries from the list of approved keywords.
% Don't make up new ones.
\begin{keywords}
keyword1 -- keyword2 -- keyword3
\end{keywords}

%%%%%%%%%%%%%%%%%%%%%%%%%%%%%%%%%%%%%%%%%%%%%%%%%%

%%%%%%%%%%%%%%%%% BODY OF PAPER %%%%%%%%%%%%%%%%%%


\drafting{
Something bland but informative here
}

\drafting{
    Introduce core sersic profile here.
}
\begin{equation}\label{eq:coresersic}
    \Sigma(R) = \Sigma'\left[ 1+\left(\frac{r_\mathrm{b}}{R}\right)^\alpha \right]^{\gamma/\alpha} \exp\left[ -b_n \left(\frac{R^\alpha+r_\mathrm{b}^\alpha}{R_\mathrm{e}^\alpha}\right)^{1/(\alpha n)} \right],
\end{equation}
where
\begin{equation}\label{eq:coresersic2}
    \Sigma' = \Sigma_\mathrm{b} 2^{-\gamma/\alpha} \exp\left[ b_n \left( 2^{1/\alpha} r_\mathrm{b}/R_\mathrm{e} \right)\right]
\end{equation}

\section{Numerical Methods}

\begin{itemize}
    \item describe toy model
    \item describe galaxy ICs: 0.5 Dehnen spheres
    \item describe merger orbit: low and high e
\end{itemize}

\subsection{Galaxy ICs}\label{ssec:gics}
We construct our galaxy models to match those of \citet{gualandris22}.
Each galaxy is represented as a stellar-only Dehnen sphere \citep{dehnen93} with shape parameter $\gamma=0.5$ and scale radius $a=186\,\mathrm{pc}$, where the Dehnen profile is given by:
\begin{equation}\label{eq:dehnen}
    \rho_\star(r) = \frac{(3-\gamma)M_\star}{4\pi} \frac{a}{r^\gamma (r+a)^{(4-\gamma)}}.
\end{equation}
The total stellar mass is $M_\star=10^{10}\,\Msun$, and at the centre of the model galaxy a SMBH of mass $M_\bullet=10^8\,\Msun$ is placed with zero initial velocity.
We test \drafting{X} different mass resolutions by sampling the stellar phase space distribution with varying number of stellar particles: \drafting{list resolutions here}.

\subsection{Merger ICs}
We construct isolated merger initial conditions by placing two galaxies on an elliptical orbit with varying values of eccentricity, at a fixed initial separation of $D=3.72\,\mathrm{kpc}$ and semimajor axis $a_0=2.79\,\mathrm{kpc}$.
We follow the same orbital configurations as \citet{gualandris22}, with three additional orbit. 
The chosen eccentricities $e$ are 0.5, 0.7, 0.9, 0.95, 0.97, and 0.99.
We ensure that the radial and tangential velocities of the initial merger setup are consistent with the values reported in \citet{gualandris22}.
An example of each orbital configuration is shown in \autoref{fig:orbits}.
For each orbital configuration, we run ten realisations, to account for Brownian motion effects of the SMBH binary caused by the discretised phase space. 
Each orbital geometry is simulated using our fiducial particle resolution of $M_\bullet/m_\star = 10^4$.
Additionally, we run the $e=0.9$ and $e=0.99$ orbits at varying mass resolutions, using the galaxy ICs at the mass resolutions described in \autoref{ssec:gics}.

\subsection{Assessing Eccentricity Uncertainty}
In our analysis, we wish to determine the population spread in eccentricity from a finite set of observations.
We assess the uncertainty in eccentricity using two methods: an inverse modelling approach and a forward modelling approach.

The first method follows \citet{nasim20}, and is an inverse problem. 
For each simulation in a given set, we determine the mean eccentricity $e_\mathrm{h}$ over five orbital periods centred on the orbit within which the SMBH binary has become hard.
The binary hardening radius $a_\mathrm{h}$ is defined as \citep[e.g.][]{merritt13}
\begin{equation*}
    a_\mathrm{h} = \frac{q}{1+q} \frac{r_\mathrm{h}}{4},
\end{equation*}
where $r_\mathrm{h}=r(m<2M_{\bullet,1})$ is the influence radius and $q = \{q \in \mathbb{R} | 0<q \leq 1\}$ is the mass ratio between the SMBHs.
Under the assumption that each calculated mean eccentricity is a sample of some common distribution, we may (assuming that the distribution is Gaussian) obtain an estimate of the standard deviation of this distribution by taking the standard deviation of the collection of sample eccentricity means. 

The second method is a forward modelling method, where we use Bayesian analysis to construct a simple hierarchical model. 
We construct the hierarchical model using \textsc{Stan}, which is an implementation of Hamiltonian Monte Carlo.
Our likelihood function is constructed using the analytical hardening relations \citep{quinlan96,sesana06}:
\begin{equation}\label{eq:quinlan_H}
    \dv{}{t}\left( \frac{1}{a} \right) = H \frac{G\rho}{\sigma},
\end{equation}
and
\begin{equation}\label{eq:quinlan_K}
    \dv{e}{t} = -K a^{-1} \dv{a}{t}.
\end{equation}
In these equations, $H$ and $K$ are dimensionless constants describing the evolution rate of the semimajor axis and eccentricity of the SMBH binary, respectively, and $\rho$ and $\sigma$ are the stellar density and velocity dispersion within the binary influence radius, respectively.
We wish to determine the uncertainty on the free parameters $H' = HG\rho/\sigma$ and $K$, in addition to the initial conditions for the differential equation system, $a_\mathrm{h}$ and $e_\mathrm{h}$, at a time $t_\mathrm{h}$. 
Let us denote the set of parameters $[H', K, a_\mathrm{h}, e_\mathrm{h}]$ as $\theta$.
To determine the uncertainty in the parameter set $\theta$, we solve the differential equation system analytically for a time period after the SMBH binary becomes hard where the effect of GW emission can be ignored: here taken to be $t_\mathrm{end} = t(a=10\,\mathrm{pc})$. 
Defining a time translation $t' = t - t_\mathrm{h}$, \autoref{eq:quinlan_H} becomes:
\begin{equation}\label{eq:quinlan_H2}
    \frac{1}{a(t')} \simeq H't' + \frac{1}{a_\mathrm{h}},
\end{equation}
and \autoref{eq:quinlan_K} becomes:
\begin{equation}\label{eq:quinlan_K2}
    e(t') \simeq K \ln\left[ \frac{a_\mathrm{h}}{a(t')} \right] + e_\mathrm{h}.
\end{equation}
The parallel to the first method now becomes apparent: in a given set of simulations, each simulation will have its own value of $\theta$, however each realisation of $\theta$ is itself a draw from a common population distribution that we wish to understand.
If we assume that $e_\mathrm{h}$ is normally distributed with some mean $\mu_{e_\mathrm{h}}$ and standard deviation $\sigma_{e_\mathrm{h}}$, then the distribution of $\sigma_{e_\mathrm{h}}$ is precisely the uncertainty in the standard deviation of the binary eccentricity at the time the binary becomes hard.
The primary benefit to using this second, forward modelling approach, is that we obtain an uncertainty estimate on $\sigma_{e_\mathrm{h}}$, which is not obtainable when using the first, inverse modelling approach. 

A probabilistic graphical model is shown in \drafting{FIGURE}.




\begin{figure*}
    \includegraphics{orbits}
    \caption{\drafting{Orbit of a SMBH in one of the e=0.99 mergers, which by symmetry of the equal mass system is the same as the second BH orbit reflected. Line coloured according to time. Point A shows a time just before the first hard scattering, and point B a time just after. The circle marker indicates the final position of the BH before the BHs become bound.}}
    \label{fig:orbit}
\end{figure*}

\section{Results}

\subsection{Eccentricity scatter in a toy model}
\begin{itemize}
    \item Show that brownian motion affects impact parameter of the final passage before BHs bound --> eccentricity set
\end{itemize}

\drafting{Nasim result takes the mean of the eccentricity over some time period, and then determines the standard deviation across the means of different realisations of a merger system. We probably need to show this too for `consistency', however we could also introduce a simple hierarchical model here that allows us to do something similar more robustly.}

\subsection{Eccentricity scatter in low-e orbits}
\begin{itemize}
    \item Show that impact parameters not sensitive
\end{itemize}

\subsection{Eccentricity scatter in high-e orbits}
\begin{itemize}
    \item show that impact parameters sensitive
\end{itemize}


\begin{figure*}
    \includegraphics{orbit_pars}
    \caption{\drafting{Orbital parameters of the different merger systems, colour coded according to the initial orbital eccentricity}}
    \label{fig:orbitpars}
\end{figure*}

\begin{figure*}
    \includegraphics{nasim_style_plot}
    \caption{\drafting{Show a plot like this where the scatter in eccentricity is determined as the standard deviation across simulations}}
    \label{fig:nasim}
\end{figure*}

\begin{figure*}
    \includegraphics{hyperparameter}
    \caption{\drafting{Hyperparameters of a simple hierarchical model of the binary angular momentum and energy at time when binary is hard. To show convergence, we could compare the the distributions of $\sigma_{e_h}$ directly between resolutions}}
    \label{fig:hyperparameter}
\end{figure*}

\section{Discussion}

\drafting{
    An interesting point to discuss is the origin of our eccentricity scatter. Early tests of the Gualandris setup seems to indicate that low eccentricity mergers do not demonstrate orbital flips, and are more converged. How to distinguish if the flip is the cause of the eccentricity scatter, or just correlated with eccentricity scatter?
}

We present in \autoref{sec:results} evidence for a tight correlation between the first hard-scattering angle $\theta_\mathrm{defl}$ and the hard SMBH binary eccentricity. 
With the inclusion of the toy model, which provides a resolution-free model of the SMBH scattering process, we may draw two key interpretations of the results.

First, the numerical simulations performed with \ketju{} are deterministic for a given set of initial conditions, given the same computing architecture and numerical parameters (e.g. the number of computing cores used).
The scatter we observe in $\theta_\mathrm{defl}$ is thus a manifestation of Brownian motion induced by the phase space discretisation and other numerical errors, such as floating point round-off errors, denoted $\sigma_\mathrm{sim}$. 
Thus, the numerical simulations present an instance where there is zero uncertainty in the initial conditions of the system, however there is error in the numerical evolution of the system.
We determine the positional displacement between corresponding SMBHs across the simulations at a given mass resolution for the $e_0=0.90$ and $e_0-0.99$ orbits by moving the SMBH binary system to the centre of mass frame, and setting the SMBH positions to coincide with the origin at the time of the simulation. 
Computing the standard deviation in the positional displacements $\sigma_\mathrm{displ}$ between simulation $i$ and simulation $j\neq i$ for all times $t<t_\mathrm{bound}$, we find the SMBH positions vary of the order $\sim10\,\mathrm{pc}$ at the time of the hard scattering, which translates to an uncertainty in the deflection angle. 

Second, in the real Universe there is no numerical error affecting the system evolution.
There is however uncertainty in the initial conditions of the system, whereby the chance of two galaxy merger events having the exact same initial configuration is practically zero.
We denote this uncertainty as $\sigma_\mathrm{true}$.
For the mergers of massive galaxies of the types modelled in this work, an uncertainty in the initial state of $\sigma_\mathrm{true} \sim \sigma_\mathrm{sim}$ is not unexpected, given the influence radius of a $10^8\,\Msun$ SMBH in a stellar background with velocity dispersion of the order $200\,\mathrm{km}\,\mathrm{s}^{-1}$ is $\sim 10\,\mathrm{pc}$ \drafting{From MBW, find better reference}.
In particular, processes that perturb the impact parameter associated with the hard scattering of a SMBH pair will manifest as an effect in the resulting binary eccentricity.

\drafting{
Summarise and leave the reader with a good feeling
}


\textit{Software:} \ketju{} \citep{rantala17}, \gadget{} \citep{springel05}, \textsc{Stan} \citep{carpenter17}, NumPy \citep{numpy}, SciPy \citep{scipy}, Matplotlib \citep{matplotlib}, pygad \citep{rottgers20}, Arviz \citep{arviz}. 


\section*{Acknowledgements}

P.H.J. acknowledge the support
by the European Research Council via ERC Consolidator Grant KETJU (no. 818930).

The numerical simulations used computational resources provided by
the CSC -- IT Center for Science, Finland.

%%%%%%%%%%%%%%%%%%%%%%%%%%%%%%%%%%%%%%%%%%%%%%%%%%
\section*{Data Availability}

 
\drafting{The inclusion of a Data Availability Statement is a requirement for articles published in MNRAS. Data Availability Statements provide a standardised format for readers to understand the availability of data underlying the research results described in the article. The statement may refer to original data generated in the course of the study or to third-party data analysed in the article. The statement should describe and provide means of access, where possible, by linking to the data or providing the required accession numbers for the relevant databases or DOIs.}




%%%%%%%%%%%%%%%%%%%% REFERENCES %%%%%%%%%%%%%%%%%%

% The best way to enter references is to use BibTeX:

\bibliographystyle{mnras}
\bibliography{../ref} % if your bibtex file is called example.bib


% Alternatively you could enter them by hand, like this:
% This method is tedious and prone to error if you have lots of references
%\begin{thebibliography}{99}
%\bibitem[\protect\citeauthoryear{Author}{2012}]{Author2012}
%Author A.~N., 2013, Journal of Improbable Astronomy, 1, 1
%\bibitem[\protect\citeauthoryear{Others}{2013}]{Others2013}
%Others S., 2012, Journal of Interesting Stuff, 17, 198
%\end{thebibliography}

%%%%%%%%%%%%%%%%%%%%%%%%%%%%%%%%%%%%%%%%%%%%%%%%%%

%%%%%%%%%%%%%%%%% APPENDICES %%%%%%%%%%%%%%%%%%%%%

\appendix

%\section{Prior Predictive Checks for Projected Mass Density}
Here we show the prior predictive check for the projected mass density of the H-1.00 system, as well the corner plots of the latent parameter hyperdistributions. 

\begin{figure*}
    \includegraphics{{graham_density-H-H-3.0-0.001-hierarchy_gqs_prior}.pdf}
    \caption{\drafting{Prior predictive distributions of latent parameters. Note the mode of the distributions roughly coincide with what one would expect from the literature.}}
    \label{fig:prior_latent}
\end{figure*}

\begin{figure*}
    \includegraphics{{graham_density-H-H-3.0-0.001-hierarchy_prior_pred_log10_proj_density_mean}.pdf}
    \caption{\drafting{Prior predictive estimate for the core-S\'ersic fit. Darker shaded regions correspond to greater Highest Density Intervals (HDIs). The observed data is overlaid. From the prior predictive estimate, it is seen that the model and the assumed parameter hyperdistributions are able to describe the observed data, in that the observed data lies within the range allowed by the prior predictive estimate.}}
    \label{fig:priorpred}
\end{figure*}

\begin{figure*}
    \includegraphics{{graham_density-H-H-3.0-0.001-hierarchy_pair_0}.pdf}
    \caption{\drafting{Corner plot of hierarchical model hyperparameters. Each of the hyperdistributions are modelled as Gamma distributions with shape parameters $\alpha$ and $\beta$. For example, the hyperdistribution on the core radius $r_\mathrm{b}$ is modelled as $r_\mathrm{b} \sim \text{Gamma}(r_{\mathrm{b},\alpha}, r_{\mathrm{b},\beta})$. Critically, the hyperdistributions are unimodal with highest-density regions located some distance from the distribution edges, indicating the HMC sampling has converged well. The contours correspond to the 50\%, 90\%, 95\%, and 99\% HDIs. }}
    \label{fig:corner_params_1}
\end{figure*}

\begin{figure*}
    \includegraphics{{graham_density-H-H-3.0-0.001-hierarchy_pair_1}.pdf}
    \caption{\drafting{Same as \autoref{fig:corner_params_1}}}
    \label{fig:corner_params_2}
\end{figure*}

\begin{figure*}
    \includegraphics{{graham_density-H-H-3.0-0.001-hierarchy_pair_2}.pdf}
    \caption{\drafting{Same as \autoref{fig:corner_params_1}}}
    \label{fig:corner_params_3}
\end{figure*}

%%%%%%%%%%%%%%%%%%%%%%%%%%%%%%%%%%%%%%%%%%%%%%%%%%


% Don't change these lines
\bsp	% typesetting comment
\label{lastpage}
\end{document}

% End of mnras_template.tex
