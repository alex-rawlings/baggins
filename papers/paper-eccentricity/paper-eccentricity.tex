%% Based on file 'sample631.tex'
%%
%% Modified 2021 March
%%
%% This is a sample manuscript marked up using the
%% AASTeX v6.31 LaTeX 2e macros.
%%
%% AASTeX is now based on Alexey Vikhlinin's emulateapj.cls 
%% (Copyright 2000-2015).  See the classfile for details.

%% AASTeX requires revtex4-1.cls and other external packages such as
%% latexsym, graphicx, amssymb, longtable, and epsf.  Note that as of 
%% Oct 2020, APS now uses revtex4.2e for its journals but remember that 
%% AASTeX v6+ still uses v4.1. All of these external packages should 
%% already be present in the modern TeX distributions but not always.
%% For example, revtex4.1 seems to be missing in the linux version of
%% TexLive 2020. One should be able to get all packages from www.ctan.org.
%% In particular, revtex v4.1 can be found at 
%% https://www.ctan.org/pkg/revtex4-1.

%% The first piece of markup in an AASTeX v6.x document is the \documentclass
%% command. LaTeX will ignore any data that comes before this command. The 
%% documentclass can take an optional argument to modify the output style.
%% The command below calls the preprint style which will produce a tightly 
%% typeset, one-column, single-spaced document.  It is the default and thus
%% does not need to be explicitly stated.
%%
%% using aastex version 6.3
\documentclass[twocolumn]{aastex631}

%% The default is a single spaced, 10 point font, single spaced article.
%% There are 5 other style options available via an optional argument. They
%% can be invoked like this:
%%
%% \documentclass[arguments]{aastex631}
%% 
%% where the layout options are:
%%
%%  twocolumn   : two text columns, 10 point font, single spaced article.
%%                This is the most compact and represent the final published
%%                derived PDF copy of the accepted manuscript from the publisher
%%  manuscript  : one text column, 12 point font, double spaced article.
%%  preprint    : one text column, 12 point font, single spaced article.  
%%  preprint2   : two text columns, 12 point font, single spaced article.
%%  modern      : a stylish, single text column, 12 point font, article with
%% 		  wider left and right margins. This uses the Daniel
%% 		  Foreman-Mackey and David Hogg design.
%%  RNAAS       : Supresses an abstract. Originally for RNAAS manuscripts 
%%                but now that abstracts are required this is obsolete for
%%                AAS Journals. Authors might need it for other reasons. DO NOT
%%                use \begin{abstract} and \end{abstract} with this style.
%%
%% Note that you can submit to the AAS Journals in any of these 6 styles.
%%
%% There are other optional arguments one can invoke to allow other stylistic
%% actions. The available options are:
%%
%%   astrosymb    : Loads Astrosymb font and define \astrocommands. 
%%   tighten      : Makes baselineskip slightly smaller, only works with 
%%                  the twocolumn substyle.
%%   times        : uses times font instead of the default
%%   linenumbers  : turn on lineno package.
%%   trackchanges : required to see the revision mark up and print its output
%%   longauthor   : Do not use the more compressed footnote style (default) for 
%%                  the author/collaboration/affiliations. Instead print all
%%                  affiliation information after each name. Creates a much 
%%                  longer author list but may be desirable for short 
%%                  author papers.
%% twocolappendix : make 2 column appendix.
%%   anonymous    : Do not show the authors, affiliations and acknowledgments 
%%                  for dual anonymous review.
%%
%% these can be used in any combination, e.g.
%%
%% \documentclass[twocolumn,linenumbers,trackchanges]{aastex631}
%%
%% AASTeX v6.* now includes \hyperref support. While we have built in specific
%% defaults into the classfile you can manually override them with the
%% \hypersetup command. For example,
%%
%% \hypersetup{linkcolor=red,citecolor=green,filecolor=cyan,urlcolor=magenta}
%%
%% will change the color of the internal links to red, the links to the
%% bibliography to green, the file links to cyan, and the external links to
%% magenta. Additional information on \hyperref options can be found here:
%% https://www.tug.org/applications/hyperref/manual.html#x1-40003
%%
%% Note that in v6.3 "bookmarks" has been changed to "true" in hyperref
%% to improve the accessibility of the compiled pdf file.
%%
%% If you want to create your own macros, you can do so
%% using \newcommand. Your macros should appear before
%% the \begin{document} command.
%%
\usepackage{amsmath}
\usepackage{amssymb}
\usepackage{bm} % italic bold math symbols

\newcommand{\vdag}{(v)^\dagger}
\newcommand\aastex{AAS\TeX}
\newcommand\latex{La\TeX}

\newcommand{\ketju}{\textsc{Ketju}}
\newcommand{\mstar}{\textsc{MSTAR}}
\newcommand{\gadget}{\textsc{Gadget-4}}

\newcommand{\Msun}{\ensuremath{\mathrm{M}_{\sun}}}
\newcommand{\Rvir}{\ensuremath{R_\mathrm{vir}}}

\newcommand{\dd}[1]{\ensuremath{\mathrm{d}#1}}
\newcommand{\dv}[2]{\ensuremath{\frac{\dd{#1}}{\dd{#2}}}}
\newcommand{\vb}[1]{\ensuremath{\bm{#1}}} %vectors

% drafting macros
\newcommand{\drafting}[1]{
{\leavevmode\color[RGB]{224, 77, 24}#1}
}

\newcommand{\phj}[1]{
{\leavevmode\color[RGB]{0,0,255}#1}
}

%% Reintroduced the \received and \accepted commands from AASTeX v5.2
%\received{March 1, 2021}
%\revised{April 1, 2021}
%\accepted{\today}

%% Command to document which AAS Journal the manuscript was submitted to.
%% Adds "Submitted to " the argument.
%\submitjournal{PSJ}

%% For manuscript that include authors in collaborations, AASTeX v6.31
%% builds on the \collaboration command to allow greater freedom to 
%% keep the traditional author+affiliation information but only show
%% subsets. The \collaboration command now must appear AFTER the group
%% of authors in the collaboration and it takes TWO arguments. The last
%% is still the collaboration identifier. The text given in this
%% argument is what will be shown in the manuscript. The first argument
%% is the number of author above the \collaboration command to show with
%% the collaboration text. If there are authors that are not part of any
%% collaboration the \nocollaboration command is used. This command takes
%% one argument which is also the number of authors above to show. A
%% dashed line is shown to indicate no collaboration. This example manuscript
%% shows how these commands work to display specific set of authors 
%% on the front page.
%%
%% For manuscript without any need to use \collaboration the 
%% \AuthorCollaborationLimit command from v6.2 can still be used to 
%% show a subset of authors.
%
%\AuthorCollaborationLimit=2
%
%% will only show Schwarz & Muench on the front page of the manuscript
%% (assuming the \collaboration and \nocollaboration commands are
%% commented out).
%%
%% Note that all of the author will be shown in the published article.
%% This feature is meant to be used prior to acceptance to make the
%% front end of a long author article more manageable. Please do not use
%% this functionality for manuscripts with less than 20 authors. Conversely,
%% please do use this when the number of authors exceeds 40.
%%
%% Use \allauthors at the manuscript end to show the full author list.
%% This command should only be used with \AuthorCollaborationLimit is used.

%% The following command can be used to set the latex table counters.  It
%% is needed in this document because it uses a mix of latex tabular and
%% AASTeX deluxetables.  In general it should not be needed.
%\setcounter{table}{1}

%%%%%%%%%%%%%%%%%%%%%%%%%%%%%%%%%%%%%%%%%%%%%%%%%%%%%%%%%%%%%%%%%%%%%%%%%%%%%%%%
%%
%% The following section outlines numerous optional output that
%% can be displayed in the front matter or as running meta-data.
%%
%% If you wish, you may supply running head information, although
%% this information may be modified by the editorial offices.
\shorttitle{Reviving Stochasticity}
\shortauthors{A. Rawlings et al.}
%%
%% You can add a light gray and diagonal water-mark to the first page 
%% with this command:
%% \watermark{text}
%% where "text", e.g. DRAFT, is the text to appear.  If the text is 
%% long you can control the water-mark size with:
%% \setwatermarkfontsize{dimension}
%% where dimension is any recognized LaTeX dimension, e.g. pt, in, etc.
%%
%%%%%%%%%%%%%%%%%%%%%%%%%%%%%%%%%%%%%%%%%%%%%%%%%%%%%%%%%%%%%%%%%%%%%%%%%%%%%%%%
\graphicspath{{./}{figures/}}
%% This is the end of the preamble.  Indicate the beginning of the
%% manuscript itself with \begin{document}.

\begin{document}

\title{Reviving stochasticity: SMBH-binary eccentricity is inherently random}

%% LaTeX will automatically break titles if they run longer than
%% one line. However, you may use \\ to force a line break if
%% you desire. In v6.31 you can include a footnote in the title.

%% A significant change from earlier AASTEX versions is in the structure for 
%% calling author and affiliations. The change was necessary to implement 
%% auto-indexing of affiliations which prior was a manual process that could 
%% easily be tedious in large author manuscripts.
%%
%% The \author command is the same as before except it now takes an optional
%% argument which is the 16 digit ORCID. The syntax is:
%% \author[xxxx-xxxx-xxxx-xxxx]{Author Name}
%%
%% This will hyperlink the author name to the author's ORCID page. Note that
%% during compilation, LaTeX will do some limited checking of the format of
%% the ID to make sure it is valid. If the "orcid-ID.png" image file is 
%% present or in the LaTeX pathway, the OrcID icon will appear next to
%% the authors name.
%%
%% Use \affiliation for affiliation information. The old \affil is now aliased
%% to \affiliation. AASTeX v6.31 will automatically index these in the header.
%% When a duplicate is found its index will be the same as its previous entry.
%%
%% Note that \altaffilmark and \altaffiltext have been removed and thus 
%% can not be used to document secondary affiliations. If they are used latex
%% will issue a specific error message and quit. Please use multiple 
%% \affiliation calls for to document more than one affiliation.
%%
%% The new \altaffiliation can be used to indicate some secondary information
%% such as fellowships. This command produces a non-numeric footnote that is
%% set away from the numeric \affiliation footnotes.  NOTE that if an
%% \altaffiliation command is used it must come BEFORE the \affiliation call,
%% right after the \author command, in order to place the footnotes in
%% the proper location.
%%
%% Use \email to set provide email addresses. Each \email will appear on its
%% own line so you can put multiple email address in one \email call. A new
%% \correspondingauthor command is available in V6.31 to identify the
%% corresponding author of the manuscript. It is the author's responsibility
%% to make sure this name is also in the author list.
%%
%% While authors can be grouped inside the same \author and \affiliation
%% commands it is better to have a single author for each. This allows for
%% one to exploit all the new benefits and should make book-keeping easier.
%%
%% If done correctly the peer review system will be able to
%% automatically put the author and affiliation information from the manuscript
%% and save the corresponding author the trouble of entering it by hand.

%\correspondingauthor{August Muench}
%\email{greg.schwarz@aas.org, gus.muench@aas.org}

\author[0000-0003-1807-6321]{Alexander Rawlings}
\affiliation{Department of Physics,
Gustaf H\"allstr\"omin katu 2, FI-00014, University of Helsinki, Finland}

\author[0000-0001-5721-9335]{Matias Mannerkoski}
\affiliation{Department of Physics,
Gustaf H\"allstr\"omin katu 2, FI-00014, University of Helsinki, Finland}

\author[0000-0001-8741-8263]{Peter H. Johansson}
\affiliation{Department of Physics,
Gustaf H\"allstr\"omin katu 2, FI-00014, University of Helsinki, Finland}

\author[0000-0002-7314-2558]{Thorsten Naab}
\affiliation{Max-Planck-Institut f\"ur Astrophysik, Karl-Schwarzchild-Str 1,
D-85748 Garching, Germany}

\correspondingauthor{Alexander Rawlings}
\email{alexander.rawlings@helsinki.fi}

%% Note that the \and command from previous versions of AASTeX is now
%% depreciated in this version as it is no longer necessary. AASTeX 
%% automatically takes care of all commas and "and"s between authors names.

%% AASTeX 6.31 has the new \collaboration and \nocollaboration commands to
%% provide the collaboration status of a group of authors. These commands 
%% can be used either before or after the list of corresponding authors. The
%% argument for \collaboration is the collaboration identifier. Authors are
%% encouraged to surround collaboration identifiers with ()s. The 
%% \nocollaboration command takes no argument and exists to indicate that
%% the nearby authors are not part of surrounding collaborations.

%% Mark off the abstract in the ``abstract'' environment. 
\begin{abstract}


%\drafting{
%    Key results:
%    \begin{enumerate}
%        \item Eccentricity of hard binary depends on deflection angle 
%        \item Mapping $\theta_\mathrm{defl} \rightarrow e_\mathrm{h}$ relatively independent of resolution, however low res explores wider domain space 
%        \item Explored eccentricity space can be full $[0,1]$ range for certain systems 
%        \item Exact values of $e$ from simulations meaningless, need to understand the distribution of 
%        \item Current resolutions appropriate for exploring statistics of eccentricity
%    \end{enumerate}
%    Limited to 250 words.
%}

We study the stochasticity of supermassive black hole (SMBH) binary eccentricity in $N$-body simulations of equal-mass galaxy mergers using the \ketju{} code.
In simulations with realistic, high eccentricity galactic merger orbits, the binary eccentricity is found to be a highly non-linear function of the deflection angle in the SMBH orbit during the final, nearly radial close encounter between the SMBHs before they form a bound binary.
This mapping between the deflection angle and the binary eccentricity is captured using a simple toy model, indicating that it is driven by the interplay between an asymmetric stellar potential and dynamical friction acting on the SMBHs, with no apparent resolution dependence.
Due to the non-linearity of this mapping, in certain merger configurations small, parsec-scale variations in the merger orbit can result in binary eccentricities varying in nearly the full possible range between 0 and 1.
By increasing the number of stellar particles, the reduction in Brownian motion means a smaller range of deflection angles is sampled, thus allowing the binary eccentricity produced by a given merger orbit to converge towards a definite value.
We conclude that the sensitivity of the eccentricity to small-scale perturbations of the merger orbit means that the binary eccentricity is effectively a random quantity when considering realistic galaxy mergers, however we stress that the distribution of these effectively random eccentricities can be constrained using even moderate resolution simulations.

\drafting{
    We present a study of collisionless merger simulations using the recently released \ketju{} code to investigate the stochasticity in the supermassive black hole (SMBH) binary eccentricity at the time a hard binary forms.
    We find a mapping between the hard binary eccentricity and the deflection angle the SMBHs undergo prior to becoming bound that is independent of the ratio between SMBH mass and stellar particle mass. 
    Decreasing the number of stellar particles does however sample a wider domain of deflection angles due to Brownian motion compared to increased stellar particle counts; some merger geometries explore the full $[0,1]$ domain of eccentricity as a result.
    We argue that the SMBH binary eccentricity is not a deterministic function of the initial merger eccentricity, but rather should be understood as a distribution of possible outcomes, and is also most likely present in the observable universe.
    We conclude that mass resolutions currently used to explore the SMBH binary merger process, including their merger timescales, are sufficient to capture the observable population, and that stochasticity in the SMBH binary eccentricity is unavoidable.
}

\drafting{Another version:}

\drafting{
We study the stochasticity of the eccentricity of SMBH binaries formed in $N$-body simulations equal-mass galactic mergers.
In simulations with realistic, high eccentricity galactic merger orbits, the binary eccentricity is found to be a highly non-linear function of the deflection angle in the SMBH orbit during the final, nearly radial close encounter between the SMBHs before they form a bound binary.
This mapping between the deflection angle and binary eccentricity is reproduced with a simple toy model, showing that it is caused by an interplay between the asymmetric stellar potential and dynamical friction acting on the SMBHs.
There is also no apparent resolution dependency in this mapping, with lower particle count simulations simply sampling a wider range of deflection angles due to Brownian motion.
Due to the non-linearity of this mapping, in certain merger configurations small, parsec-scale variations in the merger orbit can result in eccentricities varying in nearly the full possible range between 0 and 1.
Higher particle counts reduce the Brownian motion induced variation in the merger orbit in idealised simulations, allowing the eccentricity produced by a given merger orbit to converge towards a definite value.
However, we argue that the sensitivity of the eccentricity on the small-scale perturbations of the merger orbit means that the binary eccentricity is an effectively random quantity when considering realistic galaxy mergers, but also that the distribution of these effectively random eccentricities can be constrained using even moderate resolution simulations.
}

\drafting{Third version:}

\drafting{
We study the stochasticity of the eccentricity of SMBH binaries formed in $N$-body simulations of equal-mass galactic mergers, run with the \ketju{} code. In simulations with realistic, high eccentricity galactic merger orbits, the binary eccentricity is found to be a highly non-linear function of the final hard scattering, as defined by the deflection angle. The 
mapping between the deflection angle and the binary eccentricity can be reproduced with a simple toy model, showing that it is caused by an interplay between the asymmetric stellar potential and dynamical friction acting on the SMBHs, with no apparent resolution dependency.
At lower particle resolutions the increase in Brownian motion result in a wider sampling of the deflection angles, whereas at higher particle counts a given merger orbit converge towards a more definite eccentricity value.
Due to the non-linearity of this mapping, for certain merger configurations small, parsec-scale variations in the merger orbit can result in final SMBH eccentricities that nearly cover the full $e=[0,1]$ eccentricity range. 
We conclude that the sensitivity of the eccentricity on the small-scale perturbations of the merger orbit means that the binary eccentricity is effectively a random quantity when considering realistic galaxy mergers, however we stress that the distribution of these effectively random eccentricities can be constrained using even moderate resolution simulations.
}


\end{abstract}

%% Keywords should appear after the \end{abstract} command. 
%% The AAS Journals now uses Unified Astronomy Thesaurus concepts:
%% https://astrothesaurus.org
%% You will be asked to selected these concepts during the submission process
%% but this old "keyword" functionality is maintained in case authors want
%% to include these concepts in their preprints.
\keywords{Edit keywords}

%% From the front matter, we move on to the body of the paper.
%% Sections are demarcated by \section and \subsection, respectively.
%% Observe the use of the LaTeX \label
%% command after the \subsection to give a symbolic KEY to the
%% subsection for cross-referencing in a \ref command.
%% You can use LaTeX's \ref and \label commands to keep track of
%% cross-references to sections, equations, tables, and figures.
%% That way, if you change the order of any elements, LaTeX will
%% automatically renumber them.
%%
%% We recommend that authors also use the natbib \citep
%% and \citet commands to identify citations.  The citations are
%% tied to the reference list via symbolic KEYs. The KEY corresponds
%% to the KEY in the \bibitem in the reference list below. 


%-------------------------- MAIN TEXT STARTS HERE --------------------------%
% INTRODUCTION
\section{Introduction}
Supermassive black holes (SMBHs) are believed to reside at the centres of all massive galaxies \citep[e.g.][]{kormendy2013}.
In the $\Lambda$CDM model as galaxies grow through gas accretion and mergers \citep[e.g.][]{Volonteri2003,naab2017} their SMBHs 
are also expected to interact in a three-phase merger process \citep{begelman1980}.

Firstly, dynamical friction \citep{chandrasekhar1943} acts to bring the SMBHs from kiloparsec-scales down to parsec-scale separations, after which the SMBHs form a bound binary with a semimajor axis $a$ and eccentricity $e$ \citep[e.g.][]{Milosavljevic2001,Merritt2013book}. In the second phase, the
SMBH binary separation is reduced through sequential slingshot encounters with the surrounding stellar
distribution \citep{hills1980, hills1983, quinlan1996,Rantala2018}. Finally, at small subparsec separations gravitational wave (GW) emission
becomes the dominant mechanism by which the SMBH binary can lose its remaining orbital energy and angular momentum, thus driving the
SMBHs to coalescence \citep[e.g.][]{Mannerkoski2019,Mannerkoski2022}, where the strength of the GW emission strongly depends on the
binary eccentricity \citep{peters1963,peters1964}.

The complex nature of SMBH coalescence in a galaxy merger setting necessitates the use of numerical techniques \citep[e.g.][]{Berentzen2009,Khan2011,Dosopoulou2017,Mannerkoski2019}
to provide quantitative predictions for observational programmes such as ground-based pulsar timing arrays \citep[PTAs, e.g.][]{Lentati2015,Verbiest2016,Arzoumanian2020} and the upcoming Laser Interferometer Space Antenna \citep[LISA, e.g.][]{Amaro-Seoane2023} space mission.
Understanding how faithfully SMBH binary eccentricity is captured in numerical simulations is thus critical in constraining the SMBH merger rate
that the observational GW community aims to detect. 

The SMBH binary merging process and its dependence on eccentricity and resolution has been extensively studied using collisionsless merger simulations \citep[e.g.][]{Berentzen2009,Vasiliev2015,Bortolas2016,gualandris2017,gualandris2022}. 
Recently, \citet{nasim2020} have argued that the scatter in SMBH binary eccentricity observed in gas-free merger simulations is an artefact of poor phase space
discretisation, and that in the real Universe where SMBH masses are far greater than stellar masses $(M_\bullet \gg m_\star)$, SMBH binary eccentricity is a reliably predictable quantity.   

In this Letter, we find that the SMBH binary eccentricity is reliably predictable only for particular
initial conditions of the galaxy merger system, and does not in general hold.
Indeed, we find that for realistic galaxy merger orbits where the SMBHs undergo a final near-radial plunge before becoming bound, stochasticity in the final
%initial orbital eccentricity of the system is $e_0>0.90$, stochasticity in the final
SMBH binary eccentricity is physical and unavoidable. 


%Previous work by \citet{nasim2020} have argued that scatter in SMBH binary eccentricity observed in gas-free merger simulations is an artefact of poor phase space discretisation, and that in the real Universe where SMBH masses are far greater than stellar masses, $M_\bullet \gg m_\star$, SMBH binary eccentricity is a reliably predictable quantity.


%The first phase, dynamical friction \citep{chandrasekhar1943}, acts to bring the SMBHs from kiloparsec-scale separations to parsec-scale separations, after which the SMBHs form a bound binary system with a semimajor axis $a$ and eccentricity $e$.
%Following the binding of the SMBHs, the SMBH binary separation is reduced through sequential slingshot encounters with the surrounding stellar distribution \citep{hills1980, hills1983, quinlan1996}.
%Chaotic interactions drive the SMBH binary to sub-parsec separation: a regime where gravitational wave (GW) emission becomes the dominant mechanism by which the SMBH binary can lose its remaining orbital energy and angular momentum, resulting in the coalescence of the two SMBHs into a single SMBH.

%It is well known that the efficiency by which GW emission can remove orbital energy and angular momentum from a binary system is highly sensitive to the binary eccentricity \citep{peters1964}, with the binary merger timescale being particularly sensitive. 


%The complex nature of SMBH coalescence in a galaxy merger setting necessitates the use of numerical techniques to understand, and to provide quantitative predictions for observational programmes such as pulsar timing arrays \citep[PTAs, e.g.][]{babak2016,falxa2023} and the upcoming Laser Interferometer Space Antenna \citep[LISA, e.g.][]{amaro-seoane2007} mission.
%Understanding how faithfully SMBH binary eccentricity is captured in simulations is thus a critical piece of the puzzle in constraining the SMBH merger rate that the observational GW community aims to detect. 

%Previous work by \citet{nasim2020} have argued that scatter in SMBH binary eccentricity observed in gas-free merger simulations is an artefact of poor phase space discretisation, and that in the real Universe where SMBH masses are far greater than stellar masses, $M_\bullet \gg m_\star$, SMBH binary eccentricity is a reliably predictable quantity.

%In this work, we find that SMBH binary eccentricity is reliably predictable only for particular initial conditions of the galaxy merger system, and does not in general hold.
%Indeed, we find that for realistic galaxy merger orbits where the initial orbital eccentricity of the system is $e_0>0.90$, stochasticity in the final SMBH binary eccentricity is present. 


% NUMERICAL
\section{Numerical Simulations}

\begin{figure*}
    \centering
    \includegraphics{eccentricities.pdf}
    \caption{Eccentricity $e$ (note the non-linear scale) as a function of shifted time $t'=t-t_\mathrm{bound}$ for the $M_\bullet/m_\star=5000$ resolution simulations from the $e_0 = 0.90$ set (blue lines) and $e_0 = 0.99$ set (orange lines). 
    The downwards-pointing arrows indicate the median time when the SMBH binaries becomes hard. 
    In the $e_0=0.90$ set, six simulations have a SMBH binary merger within $50\,\mathrm{Myr}$, shown by the rapid orbit circularisation.}
    \label{fig:eccentricities}
\end{figure*}

We construct a number of idealised galaxy merger simulations, which we evolve with our new version of \ketju{} \citep{mannerkoski2023,rantala2017}, in which the dynamics of
SMBH and stars in a small region around them are integrated with an algorithmically regularised integrator \citep{rantala2020}, whereas the dynamics of the remaining particles
is computed with the \gadget{} \citep{springel2021} fast multiple method (FMM) with second order multipoles. Together with hierarchical time integration this allows for
symmetric interactions and manifest momentum conservation. \ketju{} also includes Post-Newtonian (PN) correction terms up to order 3.5 between each pair of SMBHs \citep{Blanchet2014}. 


Our galaxy models represent the nuclear bulge of a gas-devoid elliptical galaxy, matching the models of \citet{nasim2020}.
Each galaxy consists of a stellar-only Dehnen sphere \citep{dehnen1993} with shape parameter $\gamma=0.5$ and scale radius $a=186\,\mathrm{pc}$, where the Dehnen profile is given by:
\begin{equation}\label{eq:dehnen}
    \rho_\star(r) = \frac{(3-\gamma)M_\star}{4\pi} \frac{a}{r^\gamma (r+a)^{(4-\gamma)}}.
\end{equation}
The total stellar mass is $M_\star=10^{10}\,\Msun$, and at the centre of the model galaxy a SMBH of mass $M_\bullet=10^8\,\Msun$ is placed.
We test six different mass resolutions 
with a varying number of stellar particles: $N_\star = \{1.0\times10^5,2.5\times10^5, 5.0\times10^5, 1.0\times10^6, 2.0\times10^6, 4.0\times10^6\}$, corresponding to $M_\bullet/m_\star= 1000 \textnormal{--} 40000$. \drafting{See if we include even higher res runs.}

We then construct isolated merger initial conditions by placing two galaxies on two different elliptical orbits with eccentricities of $e_0=0.90$ and $e_0=0.99$, at a fixed
initial separation of $D=3.72\,\mathrm{kpc}$ and semimajor axis $a_0=2.79\,\mathrm{kpc}$.
The radial and tangential velocities of the initial merger setup are consistent with the values reported in \citet{gualandris2022} for the $e_0=0.90$ orbit.
For each orbital configuration, we run ten realisations for each mass resolution, to account for stochasticity caused by the discretised phase space. 
Interactions between stellar particles are softened with a softening length of $\epsilon = 2.5\,\mathrm{pc}$.



% RESULTS
\section{Results}\label{sec:results}

\begin{figure}
    \includegraphics{convergence}
    \caption{Convergence of the eccentricity scatter $\sigma_e$ as a function of mass resolution $M_\bullet/m_\star$ and the total number of stellar particles $N_{\star, \mathrm{tot}}$. 
    The expected $1/\sqrt{N_\star}$ scaling is recovered for the $e_0 = 0.90$ mergers. 
    For the $e_0=0.99$ mergers, $\sigma_e$ does not decrease until $M_\bullet/m_\star \geq 20000$.}
    \label{fig:convergence}
\end{figure}


\subsection{Eccentricity scatter in simulations}

The SMBH binary eccentricity for both the $e_0=0.90$ (blue) and $e_0=0.99$ (orange) orbits are shown in \autoref{fig:eccentricities} as a function of shifted time $t' = t - t_\mathrm{bound}$, where $t_\mathrm{bound}$ is the time when the binary orbital energy $E$ becomes permanently negative. 
The median time when each simulation set forms a hard binary is similar between the two sets, as indicated by the arrows in \autoref{fig:eccentricities}.
The $e_0=0.90$ set also demonstrates six SMBH-binary mergers within $50\,\mathrm{Myr}$ of forming a bound binary, seen as a rapid orbit circularisation in \autoref{fig:eccentricities}, which is captured self-consistently using \ketju{}.
The $e_0=0.99$ runs show a wide variation in eccentricity, where $e$ spans almost the entire domain range $[0,1]$.
Even though the $e_0=0.99$ runs have a higher initial merger eccentricity than the $e_0=0.90$ runs, none of the SMBH binaries in the shown $e_0=0.99$ set obtain high enough eccentricities to undergo GW-induced coalescence during the $50\,\mathrm{Myr}$ timespan.



To characterise the scatter in eccentricity, we determine the mean eccentricity $e_\mathrm{h}$ over five orbital periods centred on the orbit within which the SMBH binary has become hard.
The binary hardening radius $a_\mathrm{h}$ is defined as \citep[e.g.][]{merritt2006}
\begin{equation}\label{eq:ahard}
    a_\mathrm{h} = \frac{q}{(1+q)^2} \frac{r_\mathrm{m}}{4} = \frac{r_\mathrm{m}}{16},
\end{equation}
where $r_\mathrm{m}=r(m<2M_{\bullet,1})$ is the influence radius and $q$ is the mass ratio between the SMBHs (for our simulations $q=1.0$). 
We characterise the inter-simulation eccentricity scatter in the mean values of $e_\mathrm{h}$ with the standard deviation, denoted as $\sigma_e$.

We show the dependence of the inter-simulation eccentricity standard deviation on mass resolution for the $e_0=0.90$ orbit in \autoref{fig:convergence} with blue circle markers, and for the $e_0=0.99$ orbit using orange square markers.

For the $e_0=0.90$ orbit, the convergence of $\sigma_e$ scales as $1/\sqrt{N_{\star,\mathrm{tot}}}$, where $N_{\star,\mathrm{tot}}$ is the total number of stellar particles in the merger, in agreement with the results of \citet{nasim2020}. 
The variation in eccentricity does not show the same scaling for the $e_0=0.99$ orbit as the $e_0=0.90$ orbit. 
For mass resolutions $M_\bullet/m_\star<10000$, the value of $\sigma_e$ is almost constant, before significantly dropping at higher mass resolutions. \drafting{Add about 4M runs when done}

\subsection{Binary binding process and the scatter in eccentricity}

To understand the observed scatter in eccentricity, we investigate the binary binding process in each simulation. 
Whilst the two SMBHs are unbound, they undergo a number of hyperbolic orbits influenced by the galactic merger potential.
At each pericentre passage, the two SMBHs scatter off each other with some deflection angle $\theta_\mathrm{defl}$, defined:
\begin{equation}\label{eq:theta}
    \theta_\mathrm{defl} = 2 \arctan \left(\frac{GM}{L\sqrt{2E}} \right)
\end{equation}
where $M=M_{\bullet,1}+M_{\bullet,2}$, and $L$ is the magnitude of the SMBH 
binary angular momentum vector and $E$ the binary orbital energy at the time of the pericentre passage \citep{binney2008}.
The deflection angle is a function of the impact parameter $b$ of the SMBH trajectory. However as $\theta_\mathrm{defl}$ can be more robustly measured from the simulation data, we instead
opt using it for our analysis. 

Each pericentre passage is associated with a deflection angle, however we find that it is the first extreme deflection angle, associated with a hard scattering event, that is important.
To quantify an extreme deflection angle, we set as $\theta_\mathrm{defl}$ the first angle that exceeds $\theta_{\mathrm{defl,min}}=30\degr$. \drafting{Is this detail necessary?}


In general, this corresponds to the second or third actual pericentre passage of the SMBHs in the simulations.
An example of $\theta_\mathrm{defl}$ is shown for two realisations of the $e_0=0.90$ ($e_0=0.99$) orbit in the left (right) panel of \autoref{fig:orbit}, which occurs between the arrow heads along the SMBH trajectory in each inset panel. 
We observe evidence for a relationship between the deflection angle $\theta_\mathrm{defl}$ and the resulting eccentricity at the time the SMBH binary becomes hard, as shown in \autoref{fig:theta_e}.

\begin{figure*}
    \centering
    \includegraphics{orbit}
    \caption{Orbits from $t=0\,\mathrm{Myr}$ to a time shortly after $t_\mathrm{bound}$ of a single SMBH in two representative realisations of the $e_0=0.90$ mergers (left) and the $e_0=0.99$ mergers (right), which by symmetry of the equal mass system is a reflection of the second BH orbit about the $x=-z$ line. 
    Each line is coloured according to the shifted time $t'=t-t_\mathrm{bound}$ of the simulation, however the colours for $|t'|>2\,\mathrm{Myr}$ is constant for visual clarity. % of the deflection-phase of the evolution. 
    In each inset panel, $\theta_\mathrm{defl}$ is the deflection at the pericentre between the arrows.
    The circle marker indicates the SMBH position just before a bound binary is formed.}
    \label{fig:orbit}
\end{figure*}


\begin{figure*}
    \centering
    \includegraphics{theta_e_sim_and_model}
    \caption{
    Scattering deflection angle $\theta_\mathrm{defl}$ and the resulting binary eccentricity
    in the $e_0=0.90$ (left) and $e_0=0.99$ (right) simulations,
    compared to fitted model curves.
    The solid lines show the results from the toy model with $e_\mathrm{s}=0.91$,
    while the semitransparent lines show the results for  $e_\mathrm{s}=0.90$ and $e_\mathrm{s}=0.92$.
    The $e_0=0.90$ model curves have been shifted left by $14\degr$
    to better match the simulation data, with the shift also indicated by an arrow.
    The marginal histograms show kernel density estimates of the simulation data.
    The inset panel for the $e_0=0.90$ data shows a zoom-in with $\theta_\mathrm{defl}=[52\degr, 73\degr]$ and $e_\mathrm{h}=[0.96, 1.00]$ for clarity.
    }
    \label{fig:theta_e}
\end{figure*}


\subsection{Reproducing the eccentricity behaviour with a toy model}\label{ssec:toy}

\begin{figure*}
    \centering
    \includegraphics{sample_model_orbits}
    \caption{
    Top:
    Sample orbits computed with the toy model for different background potential eccentricities $e_\mathrm{s}$,
    with otherwise identical initial conditions.
    The dashed lines show the isopotential contours of the model.
    Only the early part of the orbital evolution is shown for clarity.
    Bottom:
    The orbital eccentricity $e$ of the model orbits (note the non-linear scale).
    The vertical line marks the end of the period shown in the top panels. 
    }
    \label{fig:model_orbits}
\end{figure*}

The dependency of the binary eccentricity on the 
deflection angle $\theta_\mathrm{defl}$ during the binary formation
can be reproduced using a simple toy model,
which includes only the essential components relevant to this process.
Taking the orbit to lie in the $xy$-plane,
the relative motion of the two equal-mass SMBHs in this model follows the 
equation of motion
\begin{equation} \label{eq:toy_model_eom}
\ddot{\vb{x}} = -\frac{2 G M_\bullet}{|\vb{x}|^3} \vb{x} + \vb{a}_\mathrm{bg} + \vb{a}_\mathrm{DF},
\end{equation}
where $\vb{x}=(x,y)$ is the separation vector of the SMBHs,
$M_\bullet = 10^8\,\Msun$ is the mass of a single SMBH,
$\vb{a}_\mathrm{bg}$ the acceleration due to the asymmetric background potential,
and $\vb{a}_\mathrm{DF}$ is the acceleration due to dynamical friction.

The stellar background is modeled as a constant density spheroidal potential
\citep[e.g.][]{binney2008}
\begin{equation}
\Phi_\mathrm{bg}(\vb{x}) = \pi G \rho (A_x x^2 + A_y y^2),
\end{equation}
with the acceleration given by 
\begin{equation}
\vb{a}_\mathrm{bg}(\vb{x}) = - \nabla \Phi_\mathrm{bg}(\vb{x}).
\end{equation}
The $A$ coefficients are related to the eccentricity $e_\mathrm{s}$ of the spheroid as
\begin{align}
A_x &= 2 \left(\frac{1-e_\mathrm{s}^2}{e_\mathrm{s}^2}\right) 
    \left[\frac{1}{2 e_\mathrm{s}} \ln\left(\frac{1+e_\mathrm{s}}{1-e_\mathrm{s}}\right) - 1 \right]\\
A_y &= \frac{1-e_\mathrm{s}^2}{e_\mathrm{s}^2} 
        \left[ \frac{1}{1-e_\mathrm{s}^2} 
        - \frac{1}{2 e_\mathrm{s}} \ln\left(\frac{1+e_\mathrm{s}}{1-e_\mathrm{s}}\right)\right],
\end{align}
with the long axis aligned along the $x$-axis.
The stellar density is set to $\rho = 300 \,\Msun\,\mathrm{pc}^{-3}$,
which approximately matches the values seen in the simulations when the SMBH binary is becoming bound.

The dynamical friction acting on a single BH is modeled using the \citet{chandrasekhar1943} formula assuming
a Maxwellian distribution with a constant velocity dispersion $\sigma_\star$ \citep[e.g.][]{binney2008}:
\begin{equation}\label{eq:chandra_df}
\begin{aligned}
\vb{a}_{1,\mathrm{DF}} = &-\frac{4 \pi G^2 M_\bullet \rho \ln{\Lambda}}{|\vb{v}_{1}|^3}\\
                         &\times\left(\operatorname{erf}(X) - \frac{2X}{\sqrt{\pi}} \exp(-X^2)\right)\vb{v}_{1},
\end{aligned}
\end{equation}
where $\vb{v}_1 = \dot{\vb{x}}/2$ is the velocity of a single BH,
and $X = |\vb{v}_1|/\sqrt{2}\sigma_\star$.
We set ${\sigma_\star=200\,\mathrm{km\,s^{-1}}}$ based on the value measured from the simulations.
The value of the Coulomb logarithm is expected to be $\ln{\Lambda}\sim 4\textnormal{--}5$
based on the size of the stellar system,
and we find that $\ln{\Lambda} = 4.7$ gives results that agree well with the simulations.

The effect of dynamical friction weakens as the SMBH binary orbit shrinks.
To account for this, when the binary is bound 
we multiply the acceleration given by equation~\eqref{eq:chandra_df}
with the smooth cut-off function
\begin{equation}
f(a) = \frac{1}{1 + \exp[(a_\mathrm{c}-a)/d_\mathrm{c}]}.
\end{equation}
Here $a$ is the semi-major axis of a bound binary,
and the cut-off scales are $a_\mathrm{c}=2 a_\mathrm{h}$
and $d_\mathrm{c} = 0.5 a_\mathrm{h}$,
with $a_\mathrm{h}$ defined by equation~\eqref{eq:ahard}.
The total dynamical friction term is then
\begin{equation}
\vb{a}_{\mathrm{DF}} = 2 f(a) \vb{a}_{1,\mathrm{DF}}.
\end{equation}

To mimic the nearly linearly plunging orbits before the scattering event seen in \autoref{fig:orbit},
we specify the initial conditions as 
\begin{align}
\vb{x}_0 &= (25\,\mathrm{pc}, b)\\
\dot{\vb{x}}_0 &= (-v_0, 0).
\end{align}
In order to match the typical SMBH relative velocity seen at this separation in
the simulations we set $v_0 = 450\,\mathrm{km\,s^{-1}}$ for the $e_0=0.90$ case, and
 $v_0 = 560\,\mathrm{km\,s^{-1}}$ for the the $e_0=0.99$ case.
 
The spheroid eccentricity parameter $e_\mathrm{s}$ is not easily measured from the simulations,
due to stellar components that remain tightly bound to the SMBHs and obscure the background
potential relevant for the dynamics,
as well as due to fact that the potential in the simulations is constantly evolving.
However, only sufficiently large values of $e_\mathrm{s}$ allow for low binary
eccentricities to be produced,
as is shown by \autoref{fig:model_orbits}.
Performing calculations with different values of $e_\mathrm{s}$,
we found a good fit to the simulation results for $e_\mathrm{s}\approx 0.9$.

We compute the resulting eccentricities for a range of impact parameters up to $b=20\,\mathrm{pc}$
for the two initial velocities,
by solving the equation of motion \eqref{eq:toy_model_eom} until the binary has become hard using
the error controlled 8th order Runge--Kutta method DOP853 included in the SciPy library \citep{virtanen2020}.
For reference, a $90\degr$ deflection during the first pericentre passage
corresponds to initial impact parameters of $b\approx 5.8\,\mathrm{pc}$ and
$b\approx 3.3\,\mathrm{pc}$ for the $e_0=0.90$ and $e_0=0.99$ cases, respectively.

The resulting model curves are shown together with the simulation data in \autoref{fig:theta_e}.
To correctly match the data in the $e_0=0.90$ case, the model curve has been shifted to the left by $14\degr$.
This shift is required to account for the rotation of the stellar component tilting the background potential
relative to the SMBH trajectories in the simulations, with the effect being only evident in the $e_0=0.90$ case
due to the less radial merger orbit.
In the $e_0=0.99$ case we can also see that the simulation data has significant scatter around
the model curve in the $\theta_\mathrm{defl} \gtrsim 120\degr$ region,
although the simple model does appear to capture the mean behaviour relatively well, apart from the largest values of
$\theta_\mathrm{defl}$.
The behaviour of the model in this region is also quite sensitive to the values of $e_\mathrm{s}$
and $\ln{\Lambda}$,
which might be related to the large scatter seen in the simulation data.

However, in general the behaviour of the simulation data is captured well,
with the model correctly producing the two main eccentricity minima,
as well as the $e\approx 1$ region between them.
The minima occur for trajectories where the torque from the background potential together with the dynamical friction
causes the SMBHs to loop around into a nearly circular orbit,
while highly eccentric binaries are produced when the binary is trapped into nearly radial oscillations along 
the main axes of the potential, as is seen in \autoref{fig:model_orbits}.


% DISCUSSION
\section{Discussion and Conclusions}

We find that the variation in eccentricity $e_\mathrm{h}$ for a given system configuration is tightly correlated with the variation in the deflection angle $\theta_\mathrm{defl}$, independent of the simulation mass resolution.
Together with the simple toy model that reproduces well the behaviour seen in the simulations,
this demonstrates that the sensitivity of the SMBH binary eccentricity to slight changes in the merger orbit is a physical feature of galactic major mergers on highly radial orbits, and not an error due to insufficient numerical resolution.
%at least in major mergers where the SMBHs have nearly equal masses,
%and not an error due to insufficient numerical resolution.

As shown by both the simulations and the toy model, the eccentricity of the SMBH binary can span nearly the full possible range between $e=[0,1]$, depending on comparatively small differences on the scale of a few parsecs in the particular realisation of the galactic merger orbit. On the other hand in minor mergers where the stellar background is less asymmetric when the SMBHs become bound,
the scatter in eccentricity is likely to be relatively low even in the case of a radial merger orbit, since the system is less sensitive to the exact value of the deflection angle,
as can be seen from the $e_\mathrm{s}=0.2$ case in \autoref{fig:model_orbits}.

Lower eccentricity merger orbits can be expected to show less scatter in the binary eccentricity due to the lack of a hard scattering event that is sensitive to slight perturbations in the merger orbit, which was also seen in the study by \citet{gualandris2022}.
However, major mergers are expected to occur on highly radial orbits based on cosmological simulations \citep[e.g.][]{khochfar2006}. In addition, the dynamical friction from the dark matter halo would in general drive even lower eccentricity merger orbits to highly radial final plunges when the initial separation of the galaxies is large enough.

The simulation setup used in this work is idealised,
and covers only a small part of the parameter space of merger configurations since we only consider relatively high-eccentricity galactic merger orbits.
However, similar behaviour can be expected to occur also in more realistic models,
since the relevant dynamics occurs within a few hundred parsecs from the centre of the merged galaxy,
where we do not expect any significant effects from dark matter or the outer parts of the galaxy, which are not included in our present simulations.



In numerical simulations with a discretised phase space, the two primary causes of random variation in the impact parameter are the initial particle sampling variation in the merger initial conditions and the Brownian motion of the SMBH binary \citep{merritt2001,Bortolas2016}.
Both of these effects scale as $1/\sqrt{N_{\star,\mathrm{tot}}}$, which can explain the observed scaling of $\sigma_e$ \citep{nasim2020} when the system falls into the region of parameter space where the relation between $e$ and $b$
is approximately linear.
The deviation from this scaling seen in \autoref{fig:convergence} can then be explained by the fact that the eccentricity is not globally a linear function of the impact parameter. 

While the scatter in the impact parameter between different numerical realisations of the same merger is due to relatively low number of particles compared to real galaxies,
we expect that uncertainty of a similar magnitude is also present in real systems due to various mechanisms, such as perturbations from their stellar environment.
In addition, the SMBHs are also not necesssarily located exactly at the centre of mass of the galaxy,
as e.g.\ work by \citet{batcheldor2010} has suggested that the SMBH in M87 may be displaced from the galactic photo-centre by up to $\sim7\,\mathrm{pc}$ due to jet acceleration or gravitational recoil kicks. Thus, it is not possible to give an exact prediction for the SMBH binary eccentricity produced by a given merger,
and instead we must focus on predicting the distribution of eccentricities that can be produced by a given merger configuration.

Predicting the eccentricity distribution requires the knowledge of the distribution of impact parameters that can result from essentially identical mergers
as well as the mapping between the impact parameter, or deflection angle, and the final eccentricity.
Since the relation between the impact parameter and the eccentricity does not appear to significantly depend on the resolution,
simulations at moderate mass resolutions of $M_\bullet/m_\star \sim 10^3$ can be used
to map out the relation by performing a large number of merger simulations,
possibly augmented by more sophisticated versions of the toy model presented here.
The distribution of impact parameters in real systems is a much more difficult problem to tackle,
since in simulations numerical resolution effects are likely dominant.
%In any case, already from the results shown here it is clear that the distribution has a complex structure.

In conclusion, we have shown the importance of the deflection angle $\theta_\mathrm{defl}$ on the resulting hard binary eccentricity $e_\mathrm{h}$. 
By using a simple, resolution-free toy model of the SMBH scattering process,
we have demonstrated that uncertainty in SMBH binary eccentricity is not caused solely by discretisation effects in galaxy merger simulations,
but is rather due to the physical sensitivity of the system to small changes in the merger orbit,
which can be caused by physical mechanisms in addition to numerical discretisation effects.
The results presented here justify extending the investigation to more realistic galaxy merger scenarios,
in order to quantify the expected range of hard binary eccentricities, which is critical for merger timescale predictions for future gravitational wave observation missions. 


% FINAL BITS
\section*{Acknowledgements}

A.R. acknowledges the support by the University of Helsinki Research Foundation.
A.R., M.M. and P.H.J. acknowledge the support
by the European Research Council via ERC Consolidator Grant KETJU (no. 818930) and the support of the Academy of Finland grant 339127.
TN acknowledges support from the Deutsche Forschungsgemeinschaft (DFG, German Research Foundation) under Germany’s Excellence Strategy - EXC-2094 - 390783311 from the DFG Cluster of Excellence ``ORIGINS''. 

The numerical simulations used computational resources provided by
the CSC -- IT Center for Science, Finland.

We list here the roles and contributions of the authors according to the Contributor Roles Taxonomy (CRediT\footnote{https://credit.niso.org}). 
\textbf{AR}: Conceptualisation, Investigation, Formal analysis, Data curation, Writing -- original draft. 
\textbf{MM}: Conceptualisation, Formal analysis, Writing -- original draft. 
\textbf{PHJ}: Supervision, Writing -- review \& editing.
\textbf{TN}:  Writing -- review \& editing.

\textit{Software:} \ketju{} \citep{mannerkoski2023,rantala2017}, \gadget{} \citep{springel2021}, NumPy \citep{harris2020}, SciPy \citep{virtanen2020}, Matplotlib \citep{hunter2007}, pygad \citep{rottgers2020}. 

%% For this sample we use BibTeX plus aasjournals.bst to generate the
%% the bibliography. The sample631.bib file was populated from ADS. To
%% get the citations to show in the compiled file do the following:
%%
%% pdflatex sample631.tex
%% bibtext sample631
%% pdflatex sample631.tex
%% pdflatex sample631.tex

\bibliography{ref}{}
\bibliographystyle{aasjournal}

%% This command is needed to show the entire author+affiliation list when
%% the collaboration and author truncation commands are used.  It has to
%% go at the end of the manuscript.
%\allauthors

%% Include this line if you are using the \added, \replaced, \deleted
%% commands to see a summary list of all changes at the end of the article.
%\listofchanges

\end{document}

% End of file `sample631.tex'.
