\label{sec:results}

\drafting{
Figures to do
\begin{itemize}
\item Example of forward-modelled semimajor axis and eccentricity
\item Example of forward-modelled density profile
\item Example of forward-modelled velocity dispersion profile
\end{itemize}
Table of galaxy properties,
and maybe another one of BH merger remnant properties.
}

\drafting{
    WHAT ARE THE MAIN OBSERVATIONS WE SEE and WHAT CONCLUSIONS CAN WE DRAW?
    \begin{enumerate}
        \item stability of hardening rate -> semimajor axis can be robustly predicted
        \item wide variance of eccentricity -> N-body models offer no quantitative prediction on likelihood of merger, can only comment on what merger looks like \textit{IF} it happens
        \item variance in observables: trends between observable properties and what observers can actually make use of (don't have this yet), e.g. signatures of SMBH scouring in velocity profile?
    \end{enumerate}
}


\begin{figure*}
    \includegraphics[width=0.7\textwidth]{graham_density_prior_pred_log10_Sigma_mean.png}
    \caption{\drafting{Prior predictive estimate for the core-S\'ersic fit. Darker shaded regions correspond to higher Bayesian Credible Intervals (BCIs). The observed data is overlaid. From the prior predictive estimate, it is seen that the model and the assumed parameter hyperdistributions are able to describe the observed data, in that the observed data lies within the range allowed by the prior predictive estimate. TODO?? This isn't a true prior predictive test, as the independent variable $R$ is not the $R$ of the data but some arbitrary range. Change it?}}
    \label{fig:priorpred}
\end{figure*}
\begin{figure*}
    \includegraphics[width=0.7\textwidth]{graham_density_pair_0.png}
    \caption{\drafting{Corner plot of hierarchical model hyperparameters. Each of the hyperdistributions are modelled as Gamma distributions with shape parameters $\alpha$ and $\beta$. For example, the hyperdistribution on the core radius $r_\mathrm{b}$ is modelled as $r_\mathrm{b} \sim \text{Gamma}(r_{\mathrm{b},\alpha}, r_{\mathrm{b},\beta})$. Critically, the hyperdistributions are unimodal with highest-density regions located some distance from the distribution edges, indicating the HMC sampling has converged well. }}
    \label{fig:corner_params_1}
\end{figure*}
\begin{figure*}
    \includegraphics[width=0.7\textwidth]{graham_density_pair_1.png}
    \caption{\drafting{Same as \ref{fig:corner_params_1}}}
    \label{fig:corner_params_2}
\end{figure*}
\begin{figure*}
    \includegraphics[width=0.7\textwidth]{graham_density_posterior_pred_log10_Sigma_mean.png}
    \caption{\drafting{Posterior predictive estimate for the core-S\'ersic fit. Darker shaded regions correspond to higher Bayesian Credible Intervals (BCIs). The observed data is overlaid, and seen to lie well within the 50\% BCI. }}
    \label{fig:postpred}
\end{figure*}


%\begin{figure*}
%\includegraphics{binary_params_main}
%\includegraphics{binary_params_secondary}
%\includegraphics{binary_params_tertiary}
%\caption{
%\drafting{
%Binary orbital parameters. Dashed line for triplet outer orbit (dashes not very %visible at the moment).
%Lower panels are mergers occurring in separate galaxies.
%Note the use of $1-e$ for the last panel.
%}
%}
%\label{fig:binary_params}
%\end{figure*}
